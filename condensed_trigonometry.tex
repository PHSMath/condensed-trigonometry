% !TEX TS-program = pdflatex
% !TEX encoding = UTF-8 Unicode

\documentclass[10pt,twoside]{article} % for a long document

\usepackage[utf8]{inputenc} % set input encoding to utf8
\usepackage{pgf, pgfarrows,pgfnodes}
\usepackage{geometry} % to change the page dimensions
\usepackage{enumerate}
\usepackage{tikz}
\usetikzlibrary{calc}
\usetikzlibrary{arrows}
\usetikzlibrary{automata}

%%% PAGE DIMENSIONS
\geometry{paperwidth=6in, paperheight=9in, margin=0.75in} % or letterpaper (US) or a5paper or....
% \geometry{margin=2in} % for example, change the margins to 2 inches all round
% \geometry{landscape} % set up the page for landscape
%   read geometry.pdf for detailed page layout information


%use this for printing
%\setlength{\oddsidemargin}{.25in}
%\setlength{\evensidemargin}{-.25in}

%shorthand commands for trig

\newcommand{\pitwo}{\ensuremath{\frac{\pi}{2}}} 
\newcommand{\fivepitwo}{\ensuremath{\frac{5\pi}{2}}} 
\newcommand{\pithree}{\ensuremath{\frac{\pi}{3}}} 
\newcommand{\pifour}{\ensuremath{\frac{\pi}{4}}} 
\newcommand{\pisix}{\ensuremath{\frac{\pi}{6}}} 
\newcommand{\fivepisix}{\ensuremath{\frac{5\pi}{6}}} 
\newcommand{\threepifour}{\ensuremath{\frac{3\pi}{4}}} 
\newcommand{\fivepifour}{\ensuremath{\frac{5\pi}{4}}} 
\newcommand{\sevenpifour}{\ensuremath{\frac{7\pi}{4}}} 
\newcommand{\twopithree}{\ensuremath{\frac{2\pi}{3}}} 
\newcommand{\threepitwo}{\ensuremath{\frac{3\pi}{2}}} 
\newcommand{\sevenpisix}{\ensuremath{\frac{7\pi}{6}}} 
\newcommand{\fourpithree}{\ensuremath{\frac{4\pi}{3}}} 
\newcommand{\fivepithree}{\ensuremath{\frac{5\pi}{3}}} 
\newcommand{\elevenpisix}{\ensuremath{\frac{11\pi}{6}}} 

{\setlength{\parindent}{0cm}

\renewcommand{\arraystretch}{2}

\newcommand{\implies}{\ensuremath{\Rightarrow}}
 	
\newcommand{\degree}{\ensuremath{^\circ}}
\newcommand{\rad}{\ensuremath{^c}}
\newcommand{\tab}{\hspace{24pt}}

\newcommand{\LANGLE}[5]{
\coordinate (A) at (#1);
\coordinate (B) at (0.5*#3+0.5*#4:#2+0.4);
\node[auto, inner sep = 4pt] () at ($(A)+(B)$)  {#5};
\draw[fill=black] (#1) circle (0.025);
\draw[fill=black] (#1) -- ++(#3:#2) circle (0.025);
\draw[fill=black] (#1) -- ++(#4:#2) circle (0.025);
\draw[densely dashed,arrows={-triangle 45},]  (#1) ++(#3:#2) arc(#3:#4:#2);
}

\newcommand{\LLANGLE}[6]{
\coordinate (A) at (#1);
\coordinate (B) at (#4:#2+#6);
\node[auto, inner sep = 4pt] () at ($(A) + (B)$) {#5};
\draw[fill=black] (#1) circle (0.025);
\draw[fill=black] (#1) -- ++(#3:#2) circle (0.025);
\draw[fill=black] (#1) -- ++(#4:#2) circle (0.025);
\draw[densely dashed,arrows={-triangle 45},]  (#1) ++(#3:#2) arc(#3:#4:#2);
}

%1 is start angle
%2 is end angle
%3 is inner radius
%4 is outer radius
%5 is the label
\newcommand{\SPIRAL}[5]{ 
\node[auto,inner sep = 4pt] () at ({(#4+0.25)*cos((#2+#1)/2)},{(#4+0.25)*sin((#1+#2)/2)}) {#5};
\draw[densely dashed,arrows={-triangle 45},domain=#1:#2,variable=\t,samples=((#2-#1)/5)]
      plot ({(((\t - #1)*#4)+((#2 - \t)*#3))*cos(\t)/(#2 - #1)},
           {(((\t - #1)*#4)+((#2 - \t)*#3))*sin(\t)/(#2 - #1)});
}

\newcommand{\ANGLE}[4]{
\draw[fill=black] (#1) circle (0.02);
\draw[fill=black] (#1) -- ++(#3:#2) circle (0.02);
\draw[fill=black] (#1) -- ++(#4:#2) circle (0.02);
\draw[densely dashed,arrows={-triangle 45}]  (#1) ++(#3:#2) arc(#3:#4:#2);
}


\newcommand{\XAXIS}[3]{
\node[inner sep = 3pt, right] () at (#1+#3,#2) {$x$};
\draw[arrows = {angle 60-angle 60}] (#1-#3,#2) -- (#1+#3,#2);
}

\newcommand{\YAXIS}[3]{
\node[inner sep = 3pt, above] () at (#1,#2+#3) {$y$};
\draw[arrows = {angle 60-angle 60}] (#1,#2-#3) -- (#1,#2+#3);
}

\newcommand{\AXES}[4]{
\node[inner sep = 3pt, right] () at (#1+#3,#2) {$x$};
\node[inner sep = 3pt, above] () at (#1,#2+#4) {$y$};
\draw[arrows = {angle 60-angle 60}] (#1-#3,#2) -- (#1+#3,#2);
\draw[arrows = {angle 60-angle 60}] (#1,#2-#4) -- (#1,#2+#4);
}

\newcommand{\PAXES}[4]{
\node[inner sep = 3pt, right] () at (#1+#3,#2) {$x$};
\node[inner sep = 3pt, above] () at (#1,#2+#4) {$y$};
\draw[arrows = {-angle 60}] (#1,#2) -- (#1+#3,#2);
\draw[arrows = {-angle 60}] (#1,#2) -- (#1,#2+#4);
}

\newcommand{\PXAXES}[4]{
\node[inner sep = 3pt, right] () at (#1+#3,#2) {$x$};
\node[inner sep = 3pt, above] () at (#1,#2+#4) {$y$};
\draw[arrows = {-angle 60}] (#1,#2) -- (#1+#3,#2);
\draw[arrows = {angle 60-angle 60}] (#1,#2-#4) -- (#1,#2+#4);
}


\title{Condensed Trigonometry}
\author{James Larsen}
\date{} % Delete this line to display the current date

%%% BEGIN DOCUMENT1
\begin{document}

\maketitle
\begin{figure}[htb]
\center
\begin{tikzpicture}[inner sep=0pt,minimum size=0mm]

\node at (0,5){};

\node at (22.5:4.25) {};
\node at (22.5:3.35) {};
\node at (22.5:2.1) {};

\AXES{0}{0}{4}{3.5}
\LANGLE{0,0}{4}{0}{45}{}
\SPIRAL{0}{45+360}{3.05}{2.8}{}
\SPIRAL{0}{45+360+360}{1.75}{1.25}{}

\node at (-0.35,2) {};
\draw[dotted] (0,4*0.7071) -- (4*0.7071,4*0.7071);
\draw[very thick,->] (0,0) -- (0,4*0.7071);

\node at (2,-0.35) {};
\draw[dotted] (4*0.7071,0) -- (4*0.7071,4*0.7071);
\draw[very thick,->] (0,0) -- (4*0.7071,0);

\end{tikzpicture}
\end{figure}

\newpage

\clearpage
\tableofcontents
\newpage
\listoffigures
\newpage

\section{Introduction}

\subsection{Foreward}

My original plan was to publish and sell Condensed Trigonometry in mid 2014.  Unfortunately, I haven't had the time to deal with traditional publishing.  So, in an effort to have the book make some impact, I've decided to release it under the Creative Commons license.  Under this license, others may read, modify, and redistribute the full text and LaTeX source code it as they see fit.\\

If you look today, you'll often find trigonometry textbooks that cost \$100 or more.  That's quite a premium for a 2000 year old idea!  I hope you find this to be an acceptable alternative, and if not, let me know!  Constructive criticism is welcome, and I'll keep working to improve the text and regularly release updates.\\

\subsection{License}
This work is licensed under the Creative Commons Attribution 4.0 International License. To view a copy of this license, visit \\ http://creativecommons.org/licenses/by/4.0.

\subsection{Instructions}

This book is an efficient introduction to trigonometry   As a result, ideas and definitions are presented rapidly.  Stop as often as you need to reflect on a sentence or equation.  Be sure you understand and memorize the contents of {\bf every} page.  If you don't understand a term or a symbol, stop and look it up before continuing to read.  Re-read a chapter as often as you need to understand it.\\

At the end of every section there is a set of review questions.  These questions are designed to test your understanding of the concepts presented in the section.  Be sure you answer {\bf every} question before moving on.  You should be able to solve every problem without looking at the book.  The solution for every question can be found in the back of the book.  Re-solve every problem as many times as you need.\\

If you can recall the contents of every chapter, and can solve every review problem without difficulty, you can say with confidence that you understand trigonometry.\\

\clearpage
\subsection{Table of Symbols}

\begin{figure}[htb]
\caption{Table of symbols used in the book.}
\label{fig:table_of_symbols}
\begin{center}
\begin{tabular}{ |c| l |}
\hline 
$\theta$ & The Greek letter theta; used as the measure of an angle.\\
\hline 
$l$ & The length of an arc.\\
\hline 
$r$ & The radius of an arc.\\
\hline 
$\approx$ & 'Approximately equal'.\\
\hline 
$\pi$ & The Greek letter pi;  a special constant $\approx 3.14$ .\\
\hline
$rad$ & Shorthand notation for radians.\\
\hline
$^c$ & Shorthand notation for radians.\\
\hline
$deg$ & Shorthand notation for degrees.\\
\hline
$^o$ & Shorthand notation for degrees.\\
\hline
$nan$ & Shorthand for 'not a number'.\\
\hline
$\implies$ & Shorthand for 'implies'.  Used in derivations.\\
\hline
$i$ & An imaginary number.  Shorthand for $sqrt(-1)$\\
\hline
$e$ & Euler's number; a special constant $\approx 2.78$\\
\hline
\end{tabular}
\end{center}
\end{figure}


\section{Angles}
\subsection{Definition of Angles}

Trigonometry is the study of angles.  An {\bf angle} is a measure of 'rotational distance'.  Take a line segment; hold one end down, and move the other.  What does the path look like?\\

The path traced by the moving endpoint is an {\bf arc}.  The formal definition of an angle is the ratio of the length of this arc to the length of its line segment, or radius.

\begin{figure}[htb]
\center
\caption{An arc and its angle.}
\label{fig:arc_and_angle}
\begin{tikzpicture}[inner sep=0pt,minimum size=0mm]
\node (r) at (1,-0.25){$r$};
\node (r) at (0.55,0.3){$\theta$};
\node (eqn) at (4,1){$\theta=\frac{l}{r} \ \ radians$};
\node () at (90:2.25){};

\LANGLE{0,0}{2.0}{0}{57.3}{$l$}

\end{tikzpicture}
\end{figure}


\subsection{Measuring Angles}

The intrinsic unit for measuring angles is the {\bf radian}.  In the previous formula, $\theta = \frac{l}{r}$, $\theta$ is in radians.  An angle of $1$ radian ($\theta = 1$) is the angle where the arc length is equal to the radius($l = r$). For a visual cue of how "big" a radian is, Figure \ref{fig:radians} shows a circle constructed from radians.  The notation for radians in an equation is 'rad' or less commonly a superscript c;  $1\  rad = 1^c = 1 \ radian$.\\

If a line segment makes a complete rotation, the resulting arc is a circle, and the angle is $2\pi$($\approx 6.28$) radians.  $\pi$($\approx 3.1415$) is the ratio of the circumference of a circle to its diameter.  This number is a constant you will see frequently.  You can see in  Figure \ref{fig:radians} that a complete circular rotation is an angle of $2\pi$ radians.\\

Because a circle represents a complete rotation, it is conventient to measure angles in fractions of a circle.  Thus, another common unit of measurement for angles is the {\bf degree}.  There are $360$ degrees in a circle.  The notation for degrees in an equation is 'deg' or a superscript o;  $1\  deg = 1^o = 1 \ degree$.\\

Since there are $2\pi$ radians in a circle, there are $\frac{360}{2\pi}$($\approx 57.3$) degrees in a radian.  Remember this!  To convert an angle from radians to degrees, multiply the radian value by $\frac{180}{\pi}$.\\

\begin{figure}[htb]
\center
\caption{Radians in a circle.}
\label{fig:radians}
\begin{tikzpicture}[inner sep=0pt,minimum size=0mm]

\node () at (90:2){};

\AXES{0}{0}{3}{3}
\LANGLE{0,0}{2.5}{0}{57.3}{$1^c$}
\LANGLE{0,0}{2.5}{57.3}{2*57.3}{$2^c$}
\LANGLE{0,0}{2.5}{2*57.3}{3*57.3}{$3^c$}
\LANGLE{0,0}{2.5}{3*57.3}{4*57.3}{$4^c$}
\LANGLE{0,0}{2.5}{4*57.3}{5*57.3}{$5^c$}
\LANGLE{0,0}{2.5}{5*57.3}{6*57.3}{$6^c$}
\LANGLE{0,0}{2.5}{6*57.3}{360}{$2\pi^c$}

\end{tikzpicture}
\end{figure}

Note that angles, by convention, are drawn as a counterclockwise rotation from the positive x axis.  An angle drawn as a clockwise rotation is considered a negative angle.\\

\begin{figure}[htb]
\center
\caption{Positive and negative angles.}
\label{fig:positive_and_negative}
\begin{tikzpicture}[inner sep=0pt,minimum size=0mm]
\node () at (90:1.25){};

\AXES{0}{0}{2.5}{2.5}
\LANGLE{0,0}{2.0}{0}{57.3}{$\theta$}
\LANGLE{0,0}{2.0}{0}{-57.3}{$-\theta$}

\end{tikzpicture}
\end{figure}

When a line undergoes a complete rotation (a rotation by $2\pi$ radians), it ends up in the same position as where it started.  A consequence of this is that for many problems, an angle $\theta$ can be replaced with an equivalent angle $\theta \pm  (2\pi)n$, where $n= 0, 1, 2...$  So, for an angle of $60^o$, $60^o + 360^o = 420^o$ or $60^o - 360^o = -300^o$ are equivalent angles.\\

\begin{figure}[htb]
\center
\caption{Equivalent angles.}
\label{fig:positive_and_negative}
\begin{tikzpicture}[inner sep=0pt,minimum size=0mm]

\AXES{0}{0}{3.25}{2.5}
\LANGLE{0,0}{2.5}{0}{60}{$60\degree$}
\SPIRAL{0}{420}{1.75}{1.25}{$420\degree$}

\end{tikzpicture}
\end{figure}

Right angles are a common sight in mathematics, and as such have a shorthand notation.  Right angles are signified by drawing a square in the corner of the angle.\\

\begin{figure}[htb]
\center
\caption{A right angle.}
\label{fig:right_angles}
\begin{tikzpicture}[inner sep=0pt,minimum size=0mm]
\node () at (0,2.25){};
\ANGLE{0,0}{2}{0}{90}

\draw (0,0.2) -- (0.2,0.2);
\draw (0.2,0) -- (0.2,0.2);

\end{tikzpicture}
\end{figure}

{\bf Complementary angles} are two angles that can be combined to form a right angle.  They do not have to be adjacent angles, just angles whose measurements add to $\pitwo$ radians.  Similarly, {\bf supplementary angles} are two angles that can be combined to form a straight angle; angles whose measurements add to $\pi$ radians.

\begin{figure}[htb]
\center
\caption{Complementary and supplementary angles.}
\label{fig:comp_and_supp}
\begin{tikzpicture}[inner sep=0pt,minimum size=0mm]
\node () at (0,2.25){};
\node () at (-3.0,-.5){complementary};
\node () at (3.0,-.5){supplementary};

\ANGLE{-3,0}{2}{0}{60}{}
\ANGLE{-3,0}{2}{60}{90}{}

\ANGLE{3,0}{2}{0}{120}{}
\ANGLE{3,0}{2}{120}{180}{}

\end{tikzpicture}
\end{figure}

\newpage

\subsection{Types of Angles}

There are several different classifications of angles: \\

A {\bf  full angle} is the angle made by a complete circle, an angle of $2\pi$ radians($360^o$). \\

A {\bf  straight angle} is the angle made by a semi-circle, an angle of $\pi$ radians($180^o$). \\

A {\bf right angle} is the angle made by a quarter-circle, an angle of $\pitwo$ radians($90^o$). \\

A {\bf reflex angle} is any angle larger than a straight angle and smaller than a full angle.  It has an angle of between $\pi$ and $2\pi$ radians(between $180^o$ and $360^o$). \\

An {\bf obtuse angle} is any angle larger than a right angle and smaller than a straight angle.  It has an angle of between $\pitwo$ and $\pi$ radians(between $90^o$ and $180^o$). \\

An {\bf acute angle} is any angle smaller than a right angle.  It has an angle of between $0$ and $\pitwo$ radians(between $0^o$ and $90^o$). \\

\begin{figure}[htb]
\center
\caption{Types of angles.}
\label{fig:types_of_angles}
\begin{tikzpicture}[inner sep=0pt,minimum size=0mm]
\node[label={[label distance=1.25cm]270:full}] (a) at (-4,3){};
\node[label={[label distance=1.25cm]270:straight}] (b) at (0,3){};
\node[label={[label distance=1.25cm]270:right}] (c) at (4,3){};
\node[label={[label distance=1.25cm]270:reflex}] (d) at (-4,0){};
\node[label={[label distance=1.25cm]270:obtuse}] (e) at (0,0){};
\node[label={[label distance=1.25cm]270:acute}] (f) at (4,0){};
\node () at (0,4.5){};

\ANGLE{a}{1}{0}{360}
\ANGLE{b}{1}{0}{180}
\ANGLE{c}{1}{0}{90}

\ANGLE{d}{1}{0}{225}
\ANGLE{e}{1}{0}{135}
\ANGLE{f}{1}{0}{60}

\end{tikzpicture}
\end{figure}


\newpage

\newpage
\subsection{Review}

\begin{enumerate}
\item {In the beginning of this chapter, I defined an angle as a measure of 'rotational distance', but I didn't state what a rotation actually is.  So, what is a rotation?  Do some research, and come up with a definition for a rotation that you find satisfactory.}

\item{What is the mathematical definition of an angle?  Write this down until you can recall it without referring back to the chapter.}

\item{What is the definition of a radian, and what is the definition of a degree?  Why would we have two different units of measure for an angle?}

\item{What is the conversion ratio for degrees to radians?  For radians to degrees?}

\item{Convert the following values in radians to degrees: \pisix, \pifour, \pithree, \pitwo, \twopithree, \threepifour, \fivepisix, $\pi$, \sevenpisix, \fivepifour, \fourpithree, \threepitwo, \fivepithree, \sevenpifour, \elevenpisix, $2\pi$.}

\item{Convert the following values in degrees to radians: 30, 45, 60, 90, 120, 135, 150, 180, 210, 225, 240, 270, 300, 315, 330, 360.}

\item{Convert the following values in radians to fractions of a circle:  \pisix, \pifour, \pithree, \pitwo, \twopithree, \threepifour, \fivepisix, $\pi$, \sevenpisix, \fivepifour, \fourpithree, \threepitwo, \fivepithree, \sevenpifour, \elevenpisix, $2\pi$.}

\item{Convert the following values from degrees to fractions of a circle: 30, 45, 60, 90, 120, 135, 150, 180, 210, 225, 240, 270, 300, 315, 330, 360.}

\item{Classify each angle as full, straight, right, reflex, obtuse, or acute:  \pisix, \pifour, \pithree, \pitwo, \twopithree, \threepifour, \fivepisix, $\pi$, \sevenpisix, \fivepifour, \fourpithree, \threepitwo, \fivepithree, \sevenpifour, \elevenpisix, $2\pi$.}

\item{Find the complement of each angle in degrees: 30, 45, 60, 90, 120, 135, 150, 180, 210, 225, 240, 270, 300, 315, 330, 360.}

\item{Find the supplement of each angle in radians:  \pisix, \pifour, \pithree, \pitwo, \twopithree, \threepifour, \fivepisix, $\pi$, \sevenpisix, \fivepifour, \fourpithree, \threepitwo, \fivepithree, \sevenpifour, \elevenpisix, $2\pi$.}

\end{enumerate}

\section{Trigonometric Functions}

\subsection{Definition and Projections}

A {\bf trigonometric function}, by definition, is any function of an angle.  However, when one says "trigonometric functions", it is universally understood that one is referring to a specific function known as the {\bf sine}, and its related functions.  These functions are derived from the projection of an angle into a {\bf Cartesean coordinate system}.  A {\bf projection}, for our purposes, can be thought of as the 'shadow' that one line casts on another.  Note that the shadow's edge is at a right angle to the line being shadowed; this is important.\\

\begin{figure}[htb]
\center
\caption{A projection of $r$ onto $x$.}
\label{fig:projection}
\begin{tikzpicture}[inner sep=0pt,minimum size=0mm]
\node () at (0,2){};
\node () at (1.325,-0.5) {$projection$};
\node () at (40:1.5) {$r$};

\LANGLE{0,0}{3}{0}{30}{$\theta$}
\XAXIS{1}{0}{2.5}

\draw[|-|,line width=1pt] (0,-0.25) -- (1.5*1.732,-0.25);

\draw[dashed] (1.5*1.732,0) -- (1.5*1.732,1.5);
\draw (1.5*1.732-0.2,0) -- (1.5*1.732-0.2,0.2) -- (1.5*1.732,0.2) -- (1.5*1.732,0.0);

\end{tikzpicture}
\end{figure}


\subsection{Sine et al.}

So, let's take a line segment of length $r$ and rotate it by angle $\theta$, and let $y$ be the length of the projection of this line segment onto the y axis of our coordinate system.\\

\begin{figure}[htb]
\center
\caption{A projection of $r$ onto the x and y axes.}
\label{fig:projection_onto_axes}
\begin{tikzpicture}[inner sep=0pt,minimum size=0mm]
\node () at (0,2.5){};
\node () at (1.325,-0.85) {$x$};
\node () at (-0.85,0.75) {$y$};

\node () at (40:1.5) {$r$};

\LANGLE{0,0}{3}{0}{30}{$\theta$}
\PAXES{0}{0}{3}{1.7}

\draw[|-|,line width=1pt] (-0.5,0) -- (-0.5,1.5);
\draw[|-|,line width=1pt] (0,-0.5) -- (1.5*1.732,-0.5);
\draw[dashed] (1.5*1.732,1.5) -- (0,1.5);
\draw (0,1.5-0.2) -- (0.2,1.5-0.2) -- (0.2,1.5) -- (0,1.5);
\draw[dashed] (1.5*1.732,0) -- (1.5*1.732,1.5);
\draw (1.5*1.732-0.2,0) -- (1.5*1.732-0.2,0.2) -- (1.5*1.732,0.2) -- (1.5*1.732,0.0);

\end{tikzpicture}
\end{figure}

$y$ must be some function of the line segment's length, and the angle of rotation, $r$ and $\theta$.  By using similar triangles, we can show that $y$ is directly proportional to $r$, meaning that $y = rf(\theta)$, the length multiplied by some unknown function of the angle.  This function $f(\theta)$ is of interest, so we rearrange the equation to isolate it.\\

\tab$y = rf(\theta)$ \ \  \implies  \ \  $f(\theta)=\frac{y}{r}$\\

So, $f(\theta)$ is the ratio of the length of the line segment to its projection on the y axis.  This function is a special trigonometric function called the {\bf sine} of an angle. \\

\tab $f(\theta) = sin(\theta) = \frac{y}{r}$, \ \ $y = r \ sin(\theta)$ \ \ \ (remember these!)\\

The {\bf cosine} of an angle is a similiar function, but uses the projection onto the x axis instead of the y axis.\\

\tab $cos(\theta) = \frac{x}{r}$, \ \ $x = r \ cos(\theta)$ \\

Another useful function is the {\bf tangent} of an angle, which is the sine divided by the cosine.\\

\tab$tan(\theta)=\frac{sin(\theta)}{cos(\theta)} = \frac{y}{r}/\frac{x}{r} = \frac{y}{x}$\\

These three functions are the basic trigonometric functions.  However, there are three other functions, each of which is the {\bf reciprocal}, or multiplicative inverse, of a basic trigonometric function.\\

The {\bf secant} of an angle is the reciprocal of the cosine.\\

\tab$sec(\theta) = \frac{1}{cos(\theta)} = \frac{r}{x}$\\

The {\bf cosecant} of an angle is the reciprocal of the sine.\\

\tab$csc(\theta) = \frac{1}{sin(\theta)} = \frac{r}{y}$\\

The {\bf cotangent} of an angle is the reciprocal of the tangent.\\

\tab$cot(\theta) = \frac{1}{tan(\theta)} = \frac{x}{y}$\\

\newpage

\subsection{Right Triangles}
A more common visual interpretation of trigonometric functions is created by using the sides of a right triangle.  The sides of this right triangle, due to the properties of parallel lines, are equal in size to the x and y projections of $r$, thus sine and cosine are also the ratio of the sides of this triangle.\\

\begin{figure}[htb]
\center
\caption{A right triangle from projections.}
\label{fig:right_triangle_projections}
\begin{tikzpicture}[inner sep=0pt,minimum size=0mm]
\node () at (0,2){};
\node () at (1.325,-0.85) {$x$};
\node () at (3.85,0.75) {$y$};
\node () at (0.75,0.2) {$\theta$};

\node () at (40:1.5) {$r$};

\LANGLE{0,0}{3}{0}{30}{}

\draw[|-|,line width=1pt] (3.65,0) -- (3.65,1.5);
\draw[|-|,line width=1pt] (0,-0.5) -- (1.5*1.732,-0.5);
\draw[dashed] (1.5*1.732,0) -- (1.5*1.732,1.5);
\draw (1.5*1.732-0.2,0) -- (1.5*1.732-0.2,0.2) -- (1.5*1.732,0.2) -- (1.5*1.732,0.0);

\end{tikzpicture}
\end{figure}

From this, we can again derive the trigonometric functions.\\

\tab$sin(\theta) = \frac{y}{r}$ \ \ \ \ $csc(\theta) = \frac{r}{y}$\\

\tab$cos(\theta) = \frac{x}{r}$ \ \ \ \ $sec(\theta) = \frac{r}{x}$\\

\tab$tan(\theta) = \frac{y}{x}$ \ \ \ \ $cot(\theta) = \frac{x}{y}$\\

\newpage
\subsection{Review}

\begin{enumerate}

\item{What is the definition of a trigonometric function?}

\item{What is the definition of a projection?}

\item{What is the definition of sine?  of cosine?}

\item{Which axis is the sine projected on?  The cosine?  Don't forget this!}

\item{Draw an angle and its projections, then define the sine and cosine of that angle.}

\item{What is the definition of tangent?  of secant?  of cosecant?  of cotangent?}

\item{Is secant the reciprocal of sine or cosine?  Don't forget this!}

\item{Why can you also use a right triangle to define sine and cosine?}

\item{Draw a right triangle, and write the length of the sides in terms of one angle and the length of the hypotenuse.}

\item{From the previous question, how do you know which side corresponds with sine, and which with cosine?}

\end{enumerate}


\section{Simple Angles}

There is a set of angles around the unit circle, which we refer to as the {\bf simple angles}, due to the simplicity in form of the sine and cosine at these angles.  It will be expected for you to recreate these values on demand in the future, make sure you study them well.\\

\begin{figure}[htb]
\center
\caption{Simple angles in degrees.}
\label{fig:simple_angles_degrees}
\begin{tikzpicture}[inner sep=0pt,minimum size=0mm]

\LLANGLE{0,0}{2}{0}{30}{$30^o$}{0.5}
\LLANGLE{0,0}{2}{30}{45}{$45^o$}{0.5}
\LLANGLE{0,0}{2}{45}{60}{$60^o$}{0.5}
\LLANGLE{0,0}{2}{60}{90}{$90^o$}{0.5}

\LLANGLE{0,0}{2}{90}{120}{$120^o$}{0.5}
\LLANGLE{0,0}{2}{120}{135}{$135^o$}{0.5}
\LLANGLE{0,0}{2}{135}{150}{$150^o$}{0.5}
\LLANGLE{0,0}{2}{150}{180}{$180^o$}{0.5}

\LLANGLE{0,0}{2}{180}{210}{$210^o$}{0.5}
\LLANGLE{0,0}{2}{210}{225}{$225^o$}{0.5}
\LLANGLE{0,0}{2}{225}{240}{$240^o$}{0.5}
\LLANGLE{0,0}{2}{240}{270}{$270^o$}{0.5}

\LLANGLE{0,0}{2}{270}{300}{$300^o$}{0.5}
\LLANGLE{0,0}{2}{300}{315}{$315^o$}{0.5}
\LLANGLE{0,0}{2}{315}{330}{$330^o$}{0.5}
\LLANGLE{0,0}{2}{330}{360}{$0^o, 360^o$}{.75}

\end{tikzpicture}
\end{figure}

\begin{figure}[htb]
\center
\caption{Simple angles in radians.}
\label{fig:simple_angles_radians}
\begin{tikzpicture}[inner sep=0pt,minimum size=0mm]

\LLANGLE{0,0}{2}{0}{30}{$\pisix^c$}{0.5}
\LLANGLE{0,0}{2}{30}{45}{$\pifour^c$}{0.5}
\LLANGLE{0,0}{2}{45}{60}{$\pithree^c$}{0.5}
\LLANGLE{0,0}{2}{60}{90}{$\pitwo^c$}{0.5}

\LLANGLE{0,0}{2}{90}{120}{$\twopithree^c$}{0.5}
\LLANGLE{0,0}{2}{120}{135}{$\threepifour^c$}{0.5}
\LLANGLE{0,0}{2}{135}{150}{$\fivepisix^c$}{0.5}
\LLANGLE{0,0}{2}{150}{180}{$\pi^c$}{0.5}

\LLANGLE{0,0}{2}{180}{210}{$\sevenpisix^c$}{0.5}
\LLANGLE{0,0}{2}{210}{225}{$\fivepifour^c$}{0.5}
\LLANGLE{0,0}{2}{225}{240}{$\fourpithree^c$}{0.5}
\LLANGLE{0,0}{2}{240}{270}{$\threepitwo^c$}{0.5}

\LLANGLE{0,0}{2}{270}{300}{$\fivepithree^c$}{0.5}
\LLANGLE{0,0}{2}{300}{315}{$\sevenpifour^c$}{0.5}
\LLANGLE{0,0}{2}{315}{330}{$\elevenpisix^c$}{0.5}
\LLANGLE{0,0}{2}{330}{360}{$0^c, 2\pi^c$}{0.75}

\end{tikzpicture}
\end{figure}

These figures can be confusing, so we will split the unit circle into quadrants and memorize a piece at a time.\\

\subsection{First Quadrant}

We will start with the simple angles in the first quadrant, or the quarter of our circle where both the x and y projections are $\geq 0$.  The simple angles in the first quadrant are $0$, $\pisix$, $\pifour$, $\pithree$, and $\pitwo$.  Looking at these angles graphically, you can see that $0$ and $\pitwo$ bound the quarter circle in the first quadrant, $\pifour$ splits the quarter circle into halves, and $\pisix$ and $\pifour$ split it into thirds.\\

\begin{figure}[htb]
\center
\caption{Sine and cosine of the first quadrant.}
\label{fig:sine_and_cosine_of_the_first_quadrant}
\begin{tikzpicture}[inner sep=0pt,minimum size=0mm]

\node at (0,7.5) {};

\node[inner sep = 4pt] at (3, -1.25) {$cos$};
\node[inner sep = 4pt] at (-1.25,3) {$sin$};


\node[inner sep = 4pt] at (6.0*1,-0.5) {$1$};
\node[inner sep = 4pt] at (6.0*0.866,-0.5) {$\frac{\sqrt{3}}{2}$};
\node[inner sep = 4pt] at (6.0*0.7071,-0.5) {$\frac{\sqrt{2}}{2}$};
\node[inner sep = 4pt] at (6.0*0.5,-0.5) {$\frac{1}{2}$};
\node[inner sep = 4pt] at (6.0*0.0,-0.5) {$0$};

\node[inner sep = 4pt] at (-0.5,6.0*1) {$1$};
\node[inner sep = 4pt] at (-0.5,6.0*0.866) {$\frac{\sqrt{3}}{2}$};
\node[inner sep = 4pt] at (-0.5,6.0*0.7071) {$\frac{\sqrt{2}}{2}$};
\node[inner sep = 4pt] at (-0.5,6.0*0.5) {$\frac{1}{2}$};
\node[inner sep = 4pt] at (-0.5,6.0*0.0) {$0$};


\LLANGLE{0,0}{6}{0}{0}{$0^o$}{0.5}
\LLANGLE{0,0}{6}{0}{30}{$30^o$}{0.5}
\LLANGLE{0,0}{6}{30}{45}{$45^o$}{0.5}
\LLANGLE{0,0}{6}{45}{60}{$60^o$}{0.5}
\LLANGLE{0,0}{6}{60}{90}{$90^o$}{0.5}

\draw[dashed] (6*0.866,6*0.0) -- (6*0.866,6*0.5);
\draw[dashed] (6*0.70701,6*0.0) -- (6*0.7071,6*0.7071);
\draw[dashed] (6*0.5,6*0.0) -- (6*0.5,6*0.866);

\draw[dashed] (6*0.0,6*0.866) -- (6*0.5,6*0.866);
\draw[dashed] (6*0.0,6*0.7071) -- (6*0.7071,6*0.7071);
\draw[dashed] (6*0.0,6*0.5) -- (6*0.866,6*0.5);

\end{tikzpicture}
\end{figure}

The following table shows the sine and cosine of angles in the first quadrant.  These values are also shown graphically.\\

\begin{figure}[htb]
\caption{Sine and cosine of simple angles in the first quadrant.}
\label{fig:simple_angles_first_quadrant}
\begin{center}
\begin{tabular}{ | c | c | c | c | c | c | }
\hline 
$\theta$ & $0^c , 0^o$ & $\frac{\pi}{6}^c , 30^o$ & $\frac{\pi}{4}^c , 45^o$ & $\frac{\pi}{3}^c , 60^o$ & $\pitwo^c , 90^o$\\
\hline 
$sin(\theta)$ & $\sqrt{\frac{0}{4}}$ & $\sqrt{\frac{1}{4}}$ & $\sqrt{\frac{2}{4}}$ & $\sqrt{\frac{3}{4}}$ & $\sqrt{\frac{4}{4}}$\\
\hline 
$cos(\theta)$ & $\sqrt{\frac{4}{4}}$ & $\sqrt{\frac{3}{4}}$ & $\sqrt{\frac{2}{4}}$ & $\sqrt{\frac{1}{4}}$ & $\sqrt{\frac{0}{4}}$\\
\hline 
$tan(\theta)$ & $\sqrt{\frac{0}{4}}$ & $\sqrt{\frac{1}{3}}$ & $\sqrt{\frac{2}{2}}$ & $\sqrt{\frac{3}{1}}$ & $\sqrt{\frac{4}{0}}$\\
\hline
\end{tabular}
\end{center}
\end{figure}

\newpage

A first useful observation is that sine and cosine have the same values for these angles, but in reverse orders.  A second observation is that all the values of sine and cosine at these angles are the square root of some fraction of fourths.  This is an example of why these angles are useful to memorize; not only are the measures of these angles simple expressions in radians or degrees, but the sine and cosine of these angles are also simple expressions.

\begin{figure}[htb]
\caption{Simplified sine and cosine of the first quadrant.}
\label{fig:simplified_first_quadrant}
\begin{center}
\begin{tabular}{ | c | c | c | c | c | c | }
\hline 
$\theta$ & $0^c , 0^o$ & $\frac{\pi}{6}^c , 30^o$ & $\frac{\pi}{4}^c , 45^o$ & $\frac{\pi}{3}^c , 60^o$ & $\pitwo^c , 90^o$\\
\hline 
$sin(\theta)$ & $0$ & $\frac{1}{2}$ & $\frac{\sqrt{2}}{2}$ & $\frac{\sqrt{3}}{2}$ & $1$\\
\hline 
$cos(\theta)$ & $1$ & $\frac{\sqrt{3}}{2}$ & $\frac{\sqrt{2}}{2}$ & $\frac{1}{2}$ & $0$\\
\hline 
$tan(\theta)$ & $0$ & $\frac{\sqrt{3}}{3}$ & $1$ & $\sqrt{3}$ & $nan$\\
\hline
\end{tabular}
\end{center}
\end{figure}

Another important thing to note is the tangent of $90^o$ has a zero denominator.  The tangent of $90^o$ does not exists, and thus is not a number (nan).  This will be explained in depth later.

\clearpage
\subsection{Second Quadrant}

The simple angles for the second quadrant($x \leq 0 , y \geq 0$) can be found by adding \pitwo to each simple angle in the first quadrant.  This is equivalent to a rotation by a quarter-circle.  As well, this quadrant is a reflection of the first quadrant across the y axis.  Thus, the x projections (cosine) become negative, but the y projections (sine) remain positive.\\

\begin{figure}[htb]
\center
\caption{Sine and cosine of the second quadrant.}
\label{fig:sine_and_cosine_of_the_second_quadrant}
\begin{tikzpicture}[inner sep=0pt,minimum size=0mm]

\node at (0,7.5) {};
\node[inner sep = 4pt] at (-3, -1) {$cos$};
\node[inner sep = 4pt] at (1.25,3) {$sin$};

\node[inner sep = 4pt] at (-6.0*1,-0.5) {$-1$};
\node[inner sep = 4pt] at (-6.0*0.866,-0.5) {$-\frac{\sqrt{3}}{2}$};
\node[inner sep = 4pt] at (-6.0*0.7071,-0.5) {$-\frac{\sqrt{2}}{2}$};
\node[inner sep = 4pt] at (-6.0*0.5,-0.5) {$-\frac{1}{2}$};
\node[inner sep = 4pt] at (-6.0*0.0,-0.5) {$0$};

\node[inner sep = 4pt] at (0.5,6.0*1) {$1$};
\node[inner sep = 4pt] at (0.5,6.0*0.866) {$\frac{\sqrt{3}}{2}$};
\node[inner sep = 4pt] at (0.5,6.0*0.7071) {$\frac{\sqrt{2}}{2}$};
\node[inner sep = 4pt] at (0.5,6.0*0.5) {$\frac{1}{2}$};
\node[inner sep = 4pt] at (0.5,6.0*0.0) {$0$};


\LLANGLE{0,0}{6}{90}{90}{$90^o$}{0.5}
\LLANGLE{0,0}{6}{90}{120}{$120^o$}{0.5}
\LLANGLE{0,0}{6}{120}{135}{$135^o$}{0.5}
\LLANGLE{0,0}{6}{135}{150}{$150^o$}{0.5}
\LLANGLE{0,0}{6}{150}{180}{$180^o$}{0.5}

\draw[dashed] (-6*0.866,6*0.0) -- (-6*0.866,6*0.5);
\draw[dashed] (-6*0.70701,6*0.0) -- (-6*0.7071,6*0.7071);
\draw[dashed] (-6*0.5,6*0.0) -- (-6*0.5,6*0.866);

\draw[dashed] (-6*0.0,6*0.866) -- (-6*0.5,6*0.866);
\draw[dashed] (-6*0.0,6*0.7071) -- (-6*0.7071,6*0.7071);
\draw[dashed] (-6*0.0,6*0.5) -- (-6*0.866,6*0.5);

\end{tikzpicture}

\begin{center}
\begin{tabular}{ | c | c | c | c | c | c | }
\hline 
$\theta$ & $\pitwo$ & $\fivepisix$ & $\threepifour$ & $\fivepisix$ & $\pi$\\
\hline 
$sin(\theta)$ & $\sqrt{\frac{4}{4}}$ & $\sqrt{\frac{3}{4}}$ & $\sqrt{\frac{2}{4}}$ & $\sqrt{\frac{1}{4}}$ & $\sqrt{\frac{0}{4}}$\\
\hline 
$cos(\theta)$ & $-\sqrt{\frac{0}{4}}$ & $-\sqrt{\frac{1}{4}}$ & $-\sqrt{\frac{2}{4}}$ & $-\sqrt{\frac{3}{4}}$ & $-\sqrt{\frac{4}{4}}$\\
\hline 
$tan(\theta)$ & $-\sqrt{\frac{4}{0}}$ & $-\sqrt{\frac{3}{1}}$ & $-\sqrt{\frac{2}{2}}$ & $-\sqrt{\frac{1}{3}}$ & $-\sqrt{\frac{0}{4}}$\\
\hline
\end{tabular}
\end{center}
\end{figure}

\clearpage
\subsection{Third Quadrant}

The simple angles in the third qudrant are $\pi$, $\sevenpisix$, $\fivepifour$, $\fourpithree$, and $\threepitwo$.  An additional reflection across the x axis implies that both x and y projections will be negated.\\

\begin{figure}[htb]
\center
\caption{Sine and cosine of the third quadrant.}
\label{fig:sine_and_cosine_of_the_third_quadrant}
\begin{tikzpicture}[inner sep=0pt,minimum size=0mm]

\node at (0,1.5) {};
\node[inner sep = 4pt] at (-3, 1) {$cos$};
\node[inner sep = 4pt] at (1.25,-3) {$sin$};

\node[inner sep = 4pt] at (-6.0*1,0.5) {$-1$};
\node[inner sep = 4pt] at (-6.0*0.866,0.5) {$-\frac{\sqrt{3}}{2}$};
\node[inner sep = 4pt] at (-6.0*0.7071,0.5) {$-\frac{\sqrt{2}}{2}$};
\node[inner sep = 4pt] at (-6.0*0.5,0.5) {$-\frac{1}{2}$};
\node[inner sep = 4pt] at (-6.0*0.0,0.5) {$0$};

\node[inner sep = 4pt] at (0.5,-6.0*1) {$-1$};
\node[inner sep = 4pt] at (0.5,-6.0*0.866) {$-\frac{\sqrt{3}}{2}$};
\node[inner sep = 4pt] at (0.5,-6.0*0.7071) {$-\frac{\sqrt{2}}{2}$};
\node[inner sep = 4pt] at (0.5,-6.0*0.5) {$-\frac{1}{2}$};
\node[inner sep = 4pt] at (0.5,-6.0*0.0) {$0$};


\LLANGLE{0,0}{6}{180}{180}{$180^o$}{0.5}
\LLANGLE{0,0}{6}{180}{210}{$210^o$}{0.5}
\LLANGLE{0,0}{6}{210}{225}{$225^o$}{0.5}
\LLANGLE{0,0}{6}{225}{240}{$240^o$}{0.5}
\LLANGLE{0,0}{6}{240}{270}{$270^o$}{0.5}

\draw[dashed] (-6*0.866,-6*0.0) -- (-6*0.866,-6*0.5);
\draw[dashed] (-6*0.70701,-6*0.0) -- (-6*0.7071,-6*0.7071);
\draw[dashed] (-6*0.5,-6*0.0) -- (-6*0.5,-6*0.866);

\draw[dashed] (-6*0.0,-6*0.866) -- (-6*0.5,-6*0.866);
\draw[dashed] (-6*0.0,-6*0.7071) -- (-6*0.7071,-6*0.7071);
\draw[dashed] (-6*0.0,-6*0.5) -- (-6*0.866,-6*0.5);

\end{tikzpicture}

\begin{center}
\begin{tabular}{ | c | c | c | c | c | c | }
\hline 
$\theta$ & $\pi$ & $\sevenpisix$ & $\fivepifour$ & $\fourpithree$ & $\threepitwo$\\
\hline 
$sin(\theta)$ & $-\sqrt{\frac{0}{4}}$ & $-\sqrt{\frac{1}{4}}$ & $-\sqrt{\frac{2}{4}}$ & $-\sqrt{\frac{3}{4}}$ & $-\sqrt{\frac{4}{4}}$\\
\hline 
$cos(\theta)$ & $-\sqrt{\frac{4}{4}}$ & $-\sqrt{\frac{3}{4}}$ & $-\sqrt{\frac{2}{4}}$ & $-\sqrt{\frac{1}{4}}$ & $-\sqrt{\frac{0}{4}}$\\
\hline 
$tan(\theta)$ & $\sqrt{\frac{0}{4}}$ & $\sqrt{\frac{1}{3}}$ & $\sqrt{\frac{2}{2}}$ & $\sqrt{\frac{3}{1}}$ & $\sqrt{\frac{4}{0}}$\\
\hline
\end{tabular}
\end{center}
\end{figure}


\clearpage
\subsection{Fourth Quadrant}
The simple angles in the fourth qudrant are $\threepitwo$, $\fivepithree$, $\sevenpifour$, $\elevenpisix$, and $2\pi$.  This quadrant is equivalent to the reflection of the first quadrant across the x axis, and so the y projections (sine) will be negative, but the x projections (cosine) will be positive. 

\begin{figure}[htb]
\center
\caption{Sine and cosine of the fourth quadrant.}
\label{fig:sine_and_cosine_of_the_fourth_quadrant}
\begin{tikzpicture}[inner sep=0pt,minimum size=0mm]

\node at (0,1.5) {};
\node[inner sep = 4pt] at (3, 1) {$cos$};
\node[inner sep = 4pt] at (-1.25,-3) {$sin$};

\node[inner sep = 4pt] at (6.0*1,0.5) {$1$};
\node[inner sep = 4pt] at (6.0*0.866,0.5) {$\frac{\sqrt{3}}{2}$};
\node[inner sep = 4pt] at (6.0*0.7071,0.5) {$\frac{\sqrt{2}}{2}$};
\node[inner sep = 4pt] at (6.0*0.5,0.5) {$\frac{1}{2}$};
\node[inner sep = 4pt] at (6.0*0.0,0.5) {$0$};

\node[inner sep = 4pt] at (-0.5,-6.0*1) {$-1$};
\node[inner sep = 4pt] at (-0.5,-6.0*0.866) {$-\frac{\sqrt{3}}{2}$};
\node[inner sep = 4pt] at (-0.5,-6.0*0.7071) {$-\frac{\sqrt{2}}{2}$};
\node[inner sep = 4pt] at (-0.5,-6.0*0.5) {$-\frac{1}{2}$};
\node[inner sep = 4pt] at (-0.5,-6.0*0.0) {$0$};

\LLANGLE{0,0}{6}{270}{270}{$270^o$}{0.5}
\LLANGLE{0,0}{6}{270}{300}{$300^o$}{0.5}
\LLANGLE{0,0}{6}{300}{315}{$315^o$}{0.5}
\LLANGLE{0,0}{6}{315}{330}{$330^o$}{0.5}
\LLANGLE{0,0}{6}{330}{360}{$360^o$}{0.5}

\draw[dashed] (6*0.866,-6*0.0) -- (6*0.866,-6*0.5);
\draw[dashed] (6*0.70701,-6*0.0) -- (6*0.7071,-6*0.7071);
\draw[dashed] (6*0.5,-6*0.0) -- (6*0.5,-6*0.866);

\draw[dashed] (6*0.0,-6*0.866) -- (6*0.5,-6*0.866);
\draw[dashed] (6*0.0,-6*0.7071) -- (6*0.7071,-6*0.7071);
\draw[dashed] (6*0.0,-6*0.5) -- (6*0.866,-6*0.5);

\end{tikzpicture}

\begin{center}
\begin{tabular}{ | c | c | c | c | c | c | }
\hline 
$\theta$ & $\threepitwo$ & $\fivepithree$ & $\sevenpifour$ & $\elevenpisix$ & $2\pi$\\
\hline 
$sin(\theta)$ & $-\sqrt{\frac{4}{4}}$ & $-\sqrt{\frac{3}{4}}$ & $-\sqrt{\frac{2}{4}}$ & $-\sqrt{\frac{1}{4}}$ & $-\sqrt{\frac{0}{4}}$\\
\hline 
$cos(\theta)$ & $\sqrt{\frac{0}{4}}$ & $\sqrt{\frac{1}{4}}$ & $\sqrt{\frac{2}{4}}$ & $\sqrt{\frac{3}{4}}$ & $\sqrt{\frac{4}{4}}$\\
\hline 
$tan(\theta)$ & $-\sqrt{\frac{4}{0}}$ & $-\sqrt{\frac{3}{1}}$ & $-\sqrt{\frac{2}{2}}$ & $-\sqrt{\frac{1}{3}}$ & $-\sqrt{\frac{0}{4}}$\\
\hline

\end{tabular}
\end{center}
\end{figure}

\clearpage
\subsection{Review}

Practice drawing this chart, and filling in the angles.  For every angle, write the value in degrees, in radians, and the sine, cosine, and tangent.  You should be able to draw and fill out the entire chart in under five minutes.\\

\begin{figure}[htb]
\center
\caption{Review of simple angles.}
\label{fig:review of simple angles}
\begin{tikzpicture}[inner sep=0pt,minimum size=0mm]

\node at (0,6) {};

\LLANGLE{0,0}{4}{0}{30}{}{0.5}
\LLANGLE{0,0}{4}{30}{45}{}{0.5}
\LLANGLE{0,0}{4}{45}{60}{}{0.5}
\LLANGLE{0,0}{4}{60}{90}{}{0.5}

\LLANGLE{0,0}{4}{90}{120}{}{0.5}
\LLANGLE{0,0}{4}{120}{135}{}{0.5}
\LLANGLE{0,0}{4}{135}{150}{}{0.5}
\LLANGLE{0,0}{4}{150}{180}{}{0.5}

\LLANGLE{0,0}{4}{180}{210}{}{0.5}
\LLANGLE{0,0}{4}{210}{225}{}{0.5}
\LLANGLE{0,0}{4}{225}{240}{}{0.5}
\LLANGLE{0,0}{4}{240}{270}{}{0.5}

\LLANGLE{0,0}{4}{270}{300}{}{0.5}
\LLANGLE{0,0}{4}{300}{315}{}{0.5}
\LLANGLE{0,0}{4}{315}{330}{}{0.5}
\LLANGLE{0,0}{4}{330}{360}{}{.5}

\end{tikzpicture}
\end{figure}


\section{Graphs of Functions}

\subsection{Sine}

\begin{figure}[htb]
\center
\caption{Graph of sine.}
\label{fig:graph of sine}
\begin{tikzpicture}[inner sep=0pt,minimum size=0mm, scale = 0.7]

\node at (3.1415, -2.5) {$\pi$};
\node at (3/2*3.1415, -2.5) {$\frac{3\pi}{2}$};
\node at (2*3.1415, -2.5) {$2\pi$};
\node at (1/2*3.1415, -2.5) {$\frac{\pi}{2}$};
\node at (0, -2.5) {$0$};
\node at (1/2*-3.1415, -2.5) {-$\frac{\pi}{2}$};
\node at (-3.1415, -2.5) {$-\pi$};
\node at (3/2*-3.1415, -2.5) {-$\frac{3\pi}{2}$};
\node at (2*-3.1415, -2.5) {$-2\pi$};

\node[left] at (-7,2) {$2$};
\node[left] at (-7,1) {$1$};
\node[left] at (-7,0) {$0$};
\node[left] at (-7,-1) {$-1$};
\node[left] at (-7,-2) {$-2$};

\draw[xstep=3.1415/2, densely dashed] (-2*3.1415,-2) grid (2*3.1415,2);
\AXES{0}{0}{2*3.1415}{2}
\draw[thick, variable = \t, domain=2*-3.1415:2*3.1415,samples=250] plot ({\t},{sin(180*\t/3.1415)});

\end{tikzpicture}
\end{figure}

Several important properties of sine can be seen by looking at a graph.  The foremost is that sine is a {\bf periodic function}, or a function that repeats itself.  This is a direct consequence of the circular nature of angles;  adding a full rotation ($2\pi$) to an angle has no effect on it, thus any angle not in the range $[0,2\pi]$ is merely an equivalent angle to one in the range.  As a result, sine has a {\bf period}, or repeat distance, of $2\pi$.\\

Another important property to note is that the sine of any angle is always in the range $[-1,1]$.  Remember that sine is the ratio of a line to its projection.  A projection can be at most the length of the line casting it, so sine can be no greater in magnitude than $1$.\\



\clearpage
\subsection{Cosine}

\begin{figure}[htb]
\center
\caption{Graph of cosine.}
\label{fig:graph of cosine}
\begin{tikzpicture}[inner sep=0pt,minimum size=0mm, scale = 0.7]

\node at (3.1415, -2.5) {$\pi$};
\node at (3/2*3.1415, -2.5) {$\frac{3\pi}{2}$};
\node at (2*3.1415, -2.5) {$2\pi$};
\node at (1/2*3.1415, -2.5) {$\frac{\pi}{2}$};
\node at (0, -2.5) {$0$};
\node at (1/2*-3.1415, -2.5) {-$\frac{\pi}{2}$};
\node at (-3.1415, -2.5) {$-\pi$};
\node at (3/2*-3.1415, -2.5) {-$\frac{3\pi}{2}$};
\node at (2*-3.1415, -2.5) {$-2\pi$};

\node[left] at (-7,2) {$2$};
\node[left] at (-7,1) {$1$};
\node[left] at (-7,0) {$0$};
\node[left] at (-7,-1) {$-1$};
\node[left] at (-7,-2) {$-2$};

\draw[xstep=3.1415/2, densely dashed] (-2*3.1415,-2) grid (2*3.1415,2);
\AXES{0}{0}{2*3.1415}{2}
\draw[thick, variable = \t, domain=2*-3.1415:2*3.1415,samples=250] plot ({\t},{cos(180*\t/3.1415)});

\end{tikzpicture}
\end{figure}

There are obvious similarities between the graph of sine and the graph of cosine.  They have the same period, the same shape, and the same range.  In fact, sine and cosine are actually the same function, but with a difference in phase.  {\bf Phase} is an offset of the input of a function.  For example, if you have a function $f(x)$, $x$ is  the input variable.  $f(x+2)$ offset from $f(x)$ by a phase of $2$.\\

The phase offset for sine and cosine is $\pitwo$.  In other words $sin(x+\pitwo) = cos(x)$.  Remember, sine is a projection onto the y axis, and cosine is a projection onto the x axis.  The angle between the x and y axis is \pitwo, so it should make sense that the phase difference between sine and cosine is also \pitwo.

\begin{figure}[htb]
\center
\caption{Phase offset between sine and cosine.}
\label{fig:phase offset}
\begin{tikzpicture}[inner sep=0pt,minimum size=0mm, scale = 0.7]

\node at (3.1415, -2.5) {$\pi$};
\node at (3/2*3.1415, -2.5) {$\frac{3\pi}{2}$};
\node at (2*3.1415, -2.5) {$2\pi$};
\node at (1/2*3.1415, -2.5) {$\frac{\pi}{2}$};
\node at (0, -2.5) {$0$};
\node at (1/2*-3.1415, -2.5) {-$\frac{\pi}{2}$};
\node at (-3.1415, -2.5) {$-\pi$};
\node at (3/2*-3.1415, -2.5) {-$\frac{3\pi}{2}$};
\node at (2*-3.1415, -2.5) {$-2\pi$};

\node[left] at (-7,2) {$2$};
\node[left] at (-7,1) {$1$};
\node[left] at (-7,0) {$0$};
\node[left] at (-7,-1) {$-1$};
\node[left] at (-7,-2) {$-2$};

\draw[xstep=3.1415/2, dashed] (-2*3.1415,-2) grid (2*3.1415,2);
\AXES{0}{0}{2*3.1415}{2}
\draw[variable = \t, domain=2*-3.1415:2*3.1415,samples=250] plot ({\t},{sin(180*\t/3.1415)});
\draw[thick, variable = \t, domain=2*-3.1415:2*3.1415,samples=250] plot ({\t},{cos(180*\t/3.1415)});

\node at (2.25,1.25) {$sin$};
\node at (-.75,1.25) {$cos$};

\end{tikzpicture}
\end{figure}

\clearpage
\subsection{Tangent}


\begin{figure}[htb]
\center
\caption{Graph of tangent.}
\label{fig:graph of tan}
\begin{tikzpicture}[inner sep=0pt,minimum size=0mm, scale = 0.7]

\node at (3.1415, -4.5) {$\pi$};
\node at (3/2*3.1415, -4.5) {$\frac{3\pi}{2}$};
\node at (2*3.1415, -4.5) {$2\pi$};
\node at (1/2*3.1415, -4.5) {$\frac{\pi}{2}$};
\node at (0, -4.5) {$0$};
\node at (1/2*-3.1415, -4.5) {-$\frac{\pi}{2}$};
\node at (-3.1415, -4.5) {$-\pi$};
\node at (3/2*-3.1415, -4.5) {-$\frac{3\pi}{2}$};
\node at (2*-3.1415, -4.5) {$-2\pi$};

\node[left] at (-7,4) {$4$};
\node[left] at (-7,3) {$3$};
\node[left] at (-7,2) {$2$};
\node[left] at (-7,1) {$1$};
\node[left] at (-7,0) {$0$};
\node[left] at (-7,-1) {$-1$};
\node[left] at (-7,-2) {$-2$};
\node[left] at (-7,-3) {$-3$};
\node[left] at (-7,-4) {$-4$};

\draw[xstep=3.1415/2, dashed] (-2*3.1415,-4) grid (2*3.1415,4);
\AXES{0}{0}{2*3.1415}{4}

\begin{scope}
    \clip(2*-3.1415,-4) rectangle (2*3.1415,4);

\draw[thick,
 variable = \t, 
 domain=4/2*-3.1415+0.01:3/2*-3.1415-0.01,
 samples=30] 
plot ({\t},{tan(180*\t/3.1415)});


\draw[thick,
 variable = \t, 
 domain=3/2*-3.1415+0.01:1/2*-3.1415-0.01,
 samples=30] 
plot ({\t},{tan(180*\t/3.1415)});

\draw[thick,
 variable = \t, 
 domain=1/2*-3.1415+0.01:1/2*3.1415-0.01,
 samples=30] 
plot ({\t},{tan(180*\t/3.1415)});

\draw[thick,
 variable = \t, 
 domain=1/2*3.1415+0.01:3/2*3.1415-0.01,
 samples=30] 
plot ({\t},{tan(180*\t/3.1415)});

\draw[thick,
 variable = \t, 
 domain=3/2*3.1415+0.01:4/2*3.1415-0.01,
 samples=30] 
plot ({\t},{tan(180*\t/3.1415)});

\end{scope}

\end{tikzpicture}
\end{figure}

Tangent is periodic, like sine and cosine, but unlike sine and cosine it is not continuous.  A {\bf continuous function} is a function that always remains 'connected'.  You can see in the above figure that tangent has an {\bf asymptote} at \pitwo, \threepitwo, \fivepitwo, etc.  An {\bf asymptote} is a line that a function approaches, but never touches.  In this case, the asymptotes are vertical lines at \pitwo, \threepitwo, \fivepitwo, etc.  As an angle approaches \pitwo, the tangent of the angle approaches infinity from the left, and negative infinity from the right.  As a result, the tangent has a perdiod of $\pi$, a range of $(-\infty,\infty)$, and asymptotes at $\pm\frac{\pi}{2}, \pm\frac{3\pi}{2}, \pm\frac{5\pi}{2} ...$ .

\clearpage
\subsection{Secant}

\begin{figure}[htb]
\center
\caption{Graph of secant.}
\label{fig:graph of sec}
\begin{tikzpicture}[inner sep=0pt,minimum size=0mm, scale = 0.7]

\node at (3.1415, -4.5) {$\pi$};
\node at (3/2*3.1415, -4.5) {$\frac{3\pi}{2}$};
\node at (2*3.1415, -4.5) {$2\pi$};
\node at (1/2*3.1415, -4.5) {$\frac{\pi}{2}$};
\node at (0, -4.5) {$0$};
\node at (1/2*-3.1415, -4.5) {-$\frac{\pi}{2}$};
\node at (-3.1415, -4.5) {$-\pi$};
\node at (3/2*-3.1415, -4.5) {-$\frac{3\pi}{2}$};
\node at (2*-3.1415, -4.5) {$-2\pi$};

\node[left] at (-7,4) {$4$};
\node[left] at (-7,3) {$3$};
\node[left] at (-7,2) {$2$};
\node[left] at (-7,1) {$1$};
\node[left] at (-7,0) {$0$};
\node[left] at (-7,-1) {$-1$};
\node[left] at (-7,-2) {$-2$};
\node[left] at (-7,-3) {$-3$};
\node[left] at (-7,-4) {$-4$};

\draw[xstep=3.1415/2, dashed] (-2*3.1415,-4) grid (2*3.1415,4);
\AXES{0}{0}{2*3.1415}{4}

\begin{scope}
    \clip(2*-3.1415,-4) rectangle (2*3.1415,4);

\draw[thick,
 variable = \t, 
 domain=4/2*-3.1415+0.01:3/2*-3.1415-0.01,
 samples=30] 
plot ({\t},{1/cos(180*\t/3.1415)});

\draw[thick,
 variable = \t, 
 domain=3/2*-3.1415+0.01:1/2*-3.1415-0.01,
 samples=30] 
plot ({\t},{1/cos(180*\t/3.1415)});

\draw[thick,
 variable = \t, 
 domain=1/2*-3.1415+0.01:1/2*3.1415-0.01,
 samples=30] 
plot ({\t},{1/cos(180*\t/3.1415)});

\draw[thick,
 variable = \t, 
 domain=1/2*3.1415+0.01:3/2*3.1415-0.01,
 samples=30] 
plot ({\t},{1/cos(180*\t/3.1415)});

\draw[thick,
 variable = \t, 
 domain=3/2*3.1415+0.01:5/2*3.1415-0.01,
 samples=30] 
plot ({\t},{1/cos(180*\t/3.1415)});

\end{scope}

\end{tikzpicture}
\end{figure}

Secant ($\frac{1}{cos(x)}$) has a period of $2\pi$, and asymptotic at multiples of $\pm\frac{\pi}{2}, \pm\frac{3\pi}{2}, \pm\frac{5\pi}{2} ...$.  Note that secant is never in the range $(-1,1)$, so the range is $(-\infty,-1)$ and $(1,\infty)$.\\


\clearpage
\subsection{Cosecant}


\begin{figure}[htb]
\center
\caption{Graph of cosecant.}
\label{fig:graph of csc}
\begin{tikzpicture}[inner sep=0pt,minimum size=0mm, scale = 0.7]

\node at (3.1415, -4.5) {$\pi$};
\node at (3/2*3.1415, -4.5) {$\frac{3\pi}{2}$};
\node at (2*3.1415, -4.5) {$2\pi$};
\node at (1/2*3.1415, -4.5) {$\frac{\pi}{2}$};
\node at (0, -4.5) {$0$};
\node at (1/2*-3.1415, -4.5) {-$\frac{\pi}{2}$};
\node at (-3.1415, -4.5) {$-\pi$};
\node at (3/2*-3.1415, -4.5) {-$\frac{3\pi}{2}$};
\node at (2*-3.1415, -4.5) {$-2\pi$};

\node[left] at (-7,4) {$4$};
\node[left] at (-7,3) {$3$};
\node[left] at (-7,2) {$2$};
\node[left] at (-7,1) {$1$};
\node[left] at (-7,0) {$0$};
\node[left] at (-7,-1) {$-1$};
\node[left] at (-7,-2) {$-2$};
\node[left] at (-7,-3) {$-3$};
\node[left] at (-7,-4) {$-4$};

\draw[xstep=3.1415/2, dashed] (-2*3.1415,-4) grid (2*3.1415,4);
\AXES{0}{0}{2*3.1415}{4}

\begin{scope}
    \clip(2*-3.1415,-4) rectangle (2*3.1415,4);

\draw[thick,
 variable = \t, 
 domain=4/2*-3.1415+0.01:2/2*-3.1415-0.01,
 samples=30] 
plot ({\t},{1/sin(180*\t/3.1415)});

\draw[thick,
 variable = \t, 
 domain=2/2*-3.1415+0.01:0/2*-3.1415-0.01,
 samples=30] 
plot ({\t},{1/sin(180*\t/3.1415)});

\draw[thick,
 variable = \t, 
 domain=0/2*-3.1415+0.01:2/2*3.1415-0.01,
 samples=30] 
plot ({\t},{1/sin(180*\t/3.1415)});

\draw[thick,
 variable = \t, 
 domain=2/2*3.1415+0.01:4/2*3.1415-0.01,
 samples=30] 
plot ({\t},{1/sin(180*\t/3.1415)});

\end{scope}

\end{tikzpicture}
\end{figure}

Just as sine and cosine are the same function out of phase, secant and cosecant are the same function out of phase.  Cosecant ($\frac{1}{sin(x)}$) also has a period of $2\pi$, and a range of $(-\infty,-1)$ and $(1,\infty)$.  However, due to the phase offset of $\frac{\pi}{2}$, cosecant has asymptotes at $0, \pm\pi, \pm2\pi, \pm3\pi ...$.\\

\clearpage
\subsection{Cotangent}

\begin{figure}[htb]
\center
\caption{Graph of cotangent.}
\label{fig:graph of cot}
\begin{tikzpicture}[inner sep=0pt,minimum size=0mm, scale = 0.7]

\node at (3.1415, -4.5) {$\pi$};
\node at (3/2*3.1415, -4.5) {$\frac{3\pi}{2}$};
\node at (2*3.1415, -4.5) {$2\pi$};
\node at (1/2*3.1415, -4.5) {$\frac{\pi}{2}$};
\node at (0, -4.5) {$0$};
\node at (1/2*-3.1415, -4.5) {-$\frac{\pi}{2}$};
\node at (-3.1415, -4.5) {$-\pi$};
\node at (3/2*-3.1415, -4.5) {-$\frac{3\pi}{2}$};
\node at (2*-3.1415, -4.5) {$-2\pi$};

\node[left] at (-7,4) {$4$};
\node[left] at (-7,3) {$3$};
\node[left] at (-7,2) {$2$};
\node[left] at (-7,1) {$1$};
\node[left] at (-7,0) {$0$};
\node[left] at (-7,-1) {$-1$};
\node[left] at (-7,-2) {$-2$};
\node[left] at (-7,-3) {$-3$};
\node[left] at (-7,-4) {$-4$};

\draw[xstep=3.1415/2, dashed] (-2*3.1415,-4) grid (2*3.1415,4);
\AXES{0}{0}{2*3.1415}{4}

\begin{scope}
    \clip(2*-3.1415,-4) rectangle (2*3.1415,4);

\draw[thick,
 variable = \t, 
 domain=4/2*-3.1415+0.01:2/2*-3.1415-0.01,
 samples=30] 
plot ({\t},{1/tan(180*\t/3.1415)});

\draw[thick,
 variable = \t, 
 domain=2/2*-3.1415+0.01:0/2*-3.1415-0.01,
 samples=30] 
plot ({\t},{1/tan(180*\t/3.1415)});

\draw[thick,
 variable = \t, 
 domain=0/2*-3.1415+0.01:2/2*3.1415-0.01,
 samples=30] 
plot ({\t},{1/tan(180*\t/3.1415)});

\draw[thick,
 variable = \t, 
 domain=2/2*3.1415+0.01:4/2*3.1415-0.01,
 samples=30] 
plot ({\t},{1/tan(180*\t/3.1415)});
\end{scope}

\end{tikzpicture}
\end{figure}

Cotangent ($\frac{cos(x)}{sin(x)}$) is the inverse of tangent.  It also has a period of $\pi$, and a range of $(-\infty,\infty)$.  It has asymptotes at  $0, \pm\pi, \pm2\pi, \pm3\pi ...$.\\

\clearpage
\subsection{Summary}

Sine is a periodic function with a period of $2\pi$.  It oscillates between $-1$, and $1$. \\

Cosine is a phase-offset version of sine.  It is offset from sine by $\frac{\pi}{2}$.  It also has a period of $2\pi$, and a range of $[-1,1]$.\\

Tangent is a periodic function with a period of $\pi$.  Tangent has an inifinite range; $tan(\theta)$ can range from $(-\infty,\infty)$.  Tangent is a discontinuous function.  It has vertical asymptotes at $\theta = \pm \frac{\pi}{2}, \pm \frac{3\pi}{2}, \pm \frac{5\pi}{2},...$.\\

Secant is the reciprocal of cosine.  It ranges from $[1,\infty)$ and $(-\infty,1]$.  Secant is a discontinuous function, that has vertical asymptotes at $\theta = \pm \frac{\pi}{2}, \pm \frac{3\pi}{2}, \pm \frac{5\pi}{2},...$.  Secant has a period of $2\pi$.\\

Cosecant is the reciprocal of sine.  Cosecant is phase-offset from secant by $\frac{\pi}{2}$.  It has the same range as secant, and is also discontinuous, with vertical asymptotes at $\theta = 0, \pm \pi, \pm 2\pi,...$.\\

Cotangent is the reciprocal of tangent.  It also has an infinite range, and a period of $\pi$.  It is also discontinuous, with vertical asymptotes at $\theta = 0, \pm \pi, \pm 2\pi,...$.\\

\begin{figure}[htb]
\caption{Table of properties of trig. functions.}
\label{fig:table_of_properties}
\begin{center}
\begin{tabular}{|c|c|c|c|}
\hline 
function & period & range & asymptotes\\
\hline 
$sin(x)$ & $2\pi$ & $[-1,1]$ & none\\
\hline 
$cos(x)$ & $2\pi$ & $[-1,1]$ & none\\
\hline 
$tan(x)$ & $\pi$ & $[-\infty,\infty]$ & $\pm\frac{\pi}{2}, \pm\frac{3\pi}{2}, \pm\frac{5\pi}{2} ...$\\
\hline
$sec(x)$ & $2\pi$ & $(-\infty,-1)$ and $(1,\infty)$ & $\pm\frac{\pi}{2}, \pm\frac{3\pi}{2}, \pm\frac{5\pi}{2} ...$\\
\hline
$csc(x)$ & $2\pi$ & $(-\infty,-1)$ and $(1,\infty)$ & $0, \pm\pi, \pm2\pi, \pm3\pi ...$\\
\hline
$cot(x)$ & $\pi$ & $(-\infty,\infty)$ & $0, \pm\pi, \pm2\pi, \pm3\pi ...$\\
\hline
\end{tabular}
\end{center}
\end{figure}




\clearpage
\subsection{Review}

Draw and label graphs for the six trig. functions: sine, cosine, tangent, secant, cosecant, and cotangent.  Be sure to specify the range, and any asymptotes.  You should be able to draw all of these from memory, without referring back to the text.\\


\section{Identities}

\subsection{Periodicity}
From both the definition of trigonometric functions, and their graphs, we can see that they are inherently {\bf periodic functions}.  As the angle of rotation increases without limit, every trigonometric function repeats itself.  The number of rotations are irrelevant to sine and cosine;  they depend only on the final rotational position.  Because one complete rotation of an angle is $2\pi$ radians, we can deduce that the {\bf period} of these functions is $2\pi$.  That is,\\

\tab$sin(\theta) = sin(\theta \pm 2\pi n), \ \ n=0,1,2,...$\\

\tab$cos(\theta) = sin(\theta \pm 2\pi n), \ \ n=0,1,2,...$\\

From these, we can derive that the tangent, cotangent, secant, and cosecant also periodic.

\begin{figure}[htb]
\center
\caption{Periodicity of trig. functions.}
\label{fig:preiodicity of trig. functions}
\begin{tikzpicture}[inner sep=0pt,minimum size=0mm]

\node at (0,5){};

\node at (22.5:4.25) {$\frac{\pi}{4}$};
\node at (22.5:3.35) {$\frac{9\pi}{4}$};
\node at (22.5:2.1) {$\frac{17\pi}{4}$};

\AXES{0}{0}{4}{3.5}
\LANGLE{0,0}{4}{0}{45}{}
\SPIRAL{0}{45+360}{3.05}{2.8}{}
\SPIRAL{0}{45+360+360}{1.75}{1.25}{}

\node at (-0.35,2) {$sin$};
\draw[dotted] (0,4*0.7071) -- (4*0.7071,4*0.7071);
\draw[very thick,->] (0,0) -- (0,4*0.7071);

\node at (2,-0.35) {$cos$};
\draw[dotted] (4*0.7071,0) -- (4*0.7071,4*0.7071);
\draw[very thick,->] (0,0) -- (4*0.7071,0);

\end{tikzpicture}
\end{figure}


\subsection{Negative Angles}

We can graphically show how trigonometric functions respond to negative angles.  For a negative angle, the y projection of the angle is negated, while the x projection is not.  Thus,\\

\tab$sin(-\theta) = - sin(\theta)$\\

\tab$cos(-\theta) = cos(\theta)$\\

Another way to describe this is by calling sine an {\bf odd function}, and cosine an {\bf even function}.  The definition of an odd function is $f(-x) = -f(x)$, and the definition of an even function is $f(-x)=f(x)$.  Graphically, an even function is symmetric about the y axis, and an odd function has rotational symmetry about the origin.  Keep in mind that a function may be neither even nor odd.\\

\begin{figure}[htb]
\center
\caption{Negative angles and trig. functions.}
\label{fig:negative angles and trig. functions}
\begin{tikzpicture}[inner sep=0pt,minimum size=0mm]

\PXAXES{0}{0}{4}{3.25}
\LANGLE{0,0}{4}{0}{45}{$\theta$}
\LANGLE{0,0}{4}{0}{-45}{$-\theta$}

\node at (-0.75,1.5) {$sin(\theta)$};
\draw[dotted] (0,4*0.7071) -- (4*0.7071,4*0.7071);
\draw[very thick,->] (0,0) -- (0,4*0.7071);

\node at (-0.75,-1.5) {$sin(-\theta)$};
\draw[dotted] (0,-4*0.7071) -- (4*0.7071,-4*0.7071);
\draw[very thick,->] (0,0) -- (0,-4*0.7071);


\node at (2,0.35) {$cos(\theta))$};
\node at (2,-0.35) {$cos(-\theta)$};
\draw[dotted] (4*0.7071,0) -- (4*0.7071,4*0.7071);
\draw[dotted] (4*0.7071,0) -- (4*0.7071,-4*0.7071);
\draw[very thick,->] (0,0) -- (4*0.7071,0);

\end{tikzpicture}
\end{figure}

By referring to the graphs in the previous chapter, we can see that the secant is an even function, and the cosecant, tangent, and cotangent are odd functions.\\

\subsection{Rearrangements}

From the definitions of the standard trigonomtetric functions, we can derive many simple relationships. \\

\tab$tan(\theta) = \frac{sin(\theta)}{\cos(\theta)} = sin(\theta)sec(\theta)$\\

\tab$cot(\theta) = \frac{cos(\theta)}{\sin(\theta)} = cos(\theta)csc(\theta)$\\

\tab$cos(\theta)tan(\theta)=sin(\theta)$\\

\tab$sin(\theta)cot(\theta)=cos(\theta)$\\

It is often recommended to memorize many of these simple rearrangements of trigonometric identities, but not necessary, as any combination of trigonometric functions can be rewritten in terms of sine and cosine, and then simplified.\\

\subsection{The Pythagorean Identity}

Let us look at a rotation by an angle theta of a line with length $r$, and the resulting projections $x$ and $y$.\\

\begin{figure}[htb]
\center
\caption{Angle and projections.}
\label{fig:angle and projections}
\begin{tikzpicture}[inner sep=0pt,minimum size=0mm]
\node at (0,3.5) {};

\PAXES{0}{0}{4.5}{2.5}
\LANGLE{0,0}{4}{0}{30}{$\theta$}

\node at (1.5,1.25) {$r$};

\node at (-0.25,1) {$y$};
\draw[dashed] (0,4*0.5) -- (4*0.866,4*0.5);
\draw[very thick,->] (0,0) -- (0,4*0.5);

\node at (1.75,-0.25) {$x$};
\draw[dashed] (4*0.866,0) -- (4*0.866,4*0.5);
\draw[very thick,->] (0,0) -- (4*0.866,0);


\end{tikzpicture}
\end{figure}


By the Pythagorean Theorem (or the Cartesean distance formula), we can see that \\

\tab$x^2 + y^2 = r^2$\\

\tab$\implies \frac{x^2}{r^2} + \frac{y^2}{r^2} = 1$\\

\tab$\implies (\frac{x}{r})^2 + (\frac{y}{r})^2 = 1$\\

We can substitute the definition of sine and cosine into this equation, to yield\\

\tab$sin(\theta) = \frac{y}{r}, \ \ cos(\theta) = \frac{x}{r}$\\

\tab$\implies sin^2(\theta) + cos^2(\theta) = 1$\\

This is known as the {\bf Pythagorean Identity}.  It shows the fundamental relationship between the sine and cosine of an angle.\\

Through rearrangement, we can derive other equivalent pythagoren identities, but again, these need not be memorized.\\

\tab$sin^2(\theta) + cos^2(\theta) = 1$\\

\tab$\implies \frac{sin^2(\theta)}{cos^2(\theta)} + \frac{cos^2(\theta)}{cos^2(\theta)} = \frac{1}{cos^2(\theta)}$\\

\tab$\implies tan^2(\theta) + 1 = sec^2(\theta)$\\

Or,\\

\tab$sin^2(\theta) + cos^2(\theta) = 1$\\

\tab$\implies \frac{sin^2(\theta)}{sin^2(\theta)} + \frac{cos^2(\theta)}{sin^2(\theta)} = \frac{1}{sin^2(\theta)}$\\

\tab$\implies 1 + cot^2(\theta) = csc^2(\theta)$\\

\subsection{Review}

\begin{enumerate}

\item{Find 3 equivalent angles for each of the following:\\

\tab a) $30^o$\\

\tab b) $\frac{7\pi}{6}^c$\\

\tab c) $115^o$\\

\tab d) $\frac{-\pi}{2}$\\

\tab e) $0^o$\\}

\item{Write whether each of the following trig. functions are even, odd, or neither, and prove it:\\

\tab a) $tan(\theta)$\\

\tab b) $cot(\theta)$\\

\tab c) $sec(\theta)$\\

\tab d) $csc(\theta)$\\

\tab e) $sin(\theta) - cos(\theta)$\\}

\item{Simplify the following expressions:\\

\tab a) $cos^2(\theta)tan(\theta)$\\

\tab b) $csc(\theta) - cos(\theta)cot(\theta)$\\

\tab c) $1 + cot^2(\theta)$\\

\tab d) $\frac{sin^2(\theta)tan^2(\theta) + sin^2(\theta)}{tan^2(\theta)}$\\

\tab e) $\frac{(sec(\theta) + tan(\theta))(sec(\theta) - tan(\theta))}{(csc(\theta) - cot(\theta))(csc(\theta) + cot(\theta))}$\\}

\item{Given that the sine of an angle is 0.73, what is the cosine?  the tangent?\\}

\item{Given that the tangent of an angle is 0.5 and the angle is in the first quadrant, what is the cosine?}

\end{enumerate}


\section{Inverse Functions}

\subsection{Definition}

Let $f$ be some function of a variable $x$, that is $f=f(x)$.  The {\bf inverse function} of $f$, is some function $g$ such that $g(f(x))=x$.  The inverse of a function acts to "undo" whatever a function does.\\

To find the inverse of a function $f(x)$, write out the full equation, and substitute every instance of $x$ with $f^{-1}(x)$, and every instance of $f(x)$ with $x$.  You should be left with an equation in terms of $f^{-1}(x)$ and $x$.  Work to isolate $f^{-1}(x)$, and the result will be an equation for the inverse of $f(x)$.  For example,\\

\tab$f(x) = x^2 + 1$\\

\tab$\implies x = (f^{-1}(x))^2 + 1$\\

\tab$\implies x - 1 = (f^{-1}(x))^2$\\

\tab$\implies f^{-1}(x) = \sqrt{(x - 1)}$\\

Keep in mind it is not always possible to find the inverse of a function.  While the inverse of a function is always a {\bf relation}, it is not always a function, in which case it would be impossible to isolate $f^-1(x)$.\\

To find the inverse of a function graphically, take a graph of the function, and switch the x and y axes.  The inverse of a function maps its output back to its input, thus the x and y coordinates of every point in the function should be switched.\\

\subsection{Inverse Sinse}

As mentioned before, the sine is a function that maps an angle to a ratio.\\

\tab$sin(\theta) = \frac{y}{r}$\\

The inverse sine, also called the {\bf arc-sine}, is the function where\\

\tab$arcsin(\frac{y}{r}) = \theta$\\

or a function that takes in a ratio, and outputs an angle.\\

\begin{figure}[htb!]
\center
\caption{Graph of arc-sine.}
\label{fig:graph of arc-sine}
\begin{tikzpicture}[inner sep=0pt,minimum size=0mm, scale = 0.7]

\node at (-4.6,3.1415) {$\pi$};
\node at (-4.6,3/2*3.1415) {$\frac{3\pi}{2}$};
\node at (-4.6,2*3.1415) {$2\pi$};
\node at (-4.6,1/2*3.1415) {$\frac{\pi}{2}$};
\node at (-4.6,0) {$0$};
\node at (-4.6,1/2*-3.1415) {-$\frac{\pi}{2}$};
\node at (-4.6,-3.1415) {$-\pi$};
\node at (-4.6,3/2*-3.1415) {-$\frac{3\pi}{2}$};
\node at (-4.6,2*-3.1415) {$-2\pi$};

\node[left] at (4, -7) {$4$};
\node[left] at (3, -7) {$3$};
\node[left] at (2, -7) {$2$};
\node[left] at (1, -7) {$1$};
\node[left] at (0, -7) {$0$};
\node[left] at (-1, -7) {$-1$};
\node[left] at (-2, -7) {$-2$};
\node[left] at (-3, -7) {$-3$};
\node[left] at (-4, -7) {$-4$};

\draw[ystep=3.1415/2, densely dashed] (-4,-2*3.1415) grid (4,2*3.1415);
\AXES{0}{0}{4}{2*3.1415}
\draw[thick, variable = \t, domain=2*-3.1415:2*3.1415,samples=250] plot ({sin(180*\t/3.1415)},{\t});

\end{tikzpicture}
\end{figure}

Note the difference in domain and range:  The sine is a function that takes in an angle, and maps it to a ratio, whereas the arcsine takes in a ratio and maps it to an angle.\\

There are also an inverse cosine, the {\bf arc-cosine}, and and inverse tangent, the {\bf arc-tangent}.\\

\tab$arccos(\frac{x}{r}) = \theta$\\

\tab$arctan(\frac{y}{x}) = \theta$\\

\begin{figure}[htb!]
\center
\caption{Graph of arc-cosine.}
\label{fig:graph of arc-cosine}
\begin{tikzpicture}[inner sep=0pt,minimum size=0mm, scale = 0.7]

\node at (-4.6,3.1415) {$\pi$};
\node at (-4.6,3/2*3.1415) {$\frac{3\pi}{2}$};
\node at (-4.6,2*3.1415) {$2\pi$};
\node at (-4.6,1/2*3.1415) {$\frac{\pi}{2}$};
\node at (-4.6,0) {$0$};
\node at (-4.6,1/2*-3.1415) {-$\frac{\pi}{2}$};
\node at (-4.6,-3.1415) {$-\pi$};
\node at (-4.6,3/2*-3.1415) {-$\frac{3\pi}{2}$};
\node at (-4.6,2*-3.1415) {$-2\pi$};

\node[left] at (4, -7) {$4$};
\node[left] at (3, -7) {$3$};
\node[left] at (2, -7) {$2$};
\node[left] at (1, -7) {$1$};
\node[left] at (0, -7) {$0$};
\node[left] at (-1, -7) {$-1$};
\node[left] at (-2, -7) {$-2$};
\node[left] at (-3, -7) {$-3$};
\node[left] at (-4, -7) {$-4$};

\draw[ystep=3.1415/2, densely dashed] (-4,-2*3.1415) grid (4,2*3.1415);
\AXES{0}{0}{4}{2*3.1415}
\draw[thick, variable = \t, domain=2*-3.1415:2*3.1415,samples=250] plot ({cos(180*\t/3.1415)},{\t});

\end{tikzpicture}
\end{figure}

\begin{figure}[htb!]
\center
\caption{Graph of arc-tangent.}
\label{fig:graph of arc-tan}
\begin{tikzpicture}[inner sep=0pt,minimum size=0mm, scale = 0.7]

\node at (-4.6, 3.1415) {$\pi$};
\node at (-4.6, 3/2*3.1415) {$\frac{3\pi}{2}$};
\node at (-4.6, 2*3.1415) {$2\pi$};
\node at (-4.6, 1/2*3.1415) {$\frac{\pi}{2}$};
\node at (-4.6, 0) {$0$};
\node at (-4.6, 1/2*-3.1415) {-$\frac{\pi}{2}$};
\node at (-4.6, -3.1415) {$-\pi$};
\node at (-4.6, 3/2*-3.1415) {-$\frac{3\pi}{2}$};
\node at (-4.6, 2*-3.1415) {$-2\pi$};

\node[left] at (4, -7) {$4$};
\node[left] at (3, -7) {$3$};
\node[left] at (2, -7) {$2$};
\node[left] at (1, -7) {$1$};
\node[left] at (0, -7) {$0$};
\node[left] at (-1, -7) {$-1$};
\node[left] at (-2, -7) {$-2$};
\node[left] at (-3, -7) {$-3$};
\node[left] at (-4, -7) {$-4$};

\draw[ystep=3.1415/2, dashed] (-4,-2*3.1415) grid (4,2*3.1415);
\AXES{0}{0}{4}{2*3.1415}

\begin{scope}
    \clip(-4, 2*-3.1415) rectangle (4, 2*3.1415);

\draw[thick,
 variable = \t, 
 domain=4/2*-3.1415+0.01:3/2*-3.1415-0.01,
 samples=30] 
plot ({tan(180*\t/3.1415)}, {\t});


\draw[thick,
 variable = \t, 
 domain=3/2*-3.1415+0.01:1/2*-3.1415-0.01,
 samples=30] 
plot ({tan(180*\t/3.1415)}, {\t});

\draw[thick,
 variable = \t, 
 domain=1/2*-3.1415+0.01:1/2*3.1415-0.01,
 samples=30] 
plot ({tan(180*\t/3.1415)}, {\t});

\draw[thick,
 variable = \t, 
 domain=1/2*3.1415+0.01:3/2*3.1415-0.01,
 samples=30] 
plot ({tan(180*\t/3.1415)}, {\t});

\draw[thick,
 variable = \t, 
 domain=3/2*3.1415+0.01:4/2*3.1415-0.01,
 samples=30] 
plot ({tan(180*\t/3.1415)}, {\t});

\end{scope}

\end{tikzpicture}
\end{figure}

\subsection{Principal Angles}

However, due to the periodicity of the trigonometric functions, the inverse trigonometric functions are not strictly functions.  Remember that\\

\tab$sin(\theta) = sin(\theta \pm 2\pi n), \ \ n=0,1,2,...$\\

From this we can say that\\

\tab$sin(\theta \pm 2\pi n) = \frac{y}{r}$\\

Taking the inverse of both sides leaves us with\\

\tab$arcsin(sin(\theta + 2\pi n)) = arcsin(\frac{y}{r})$\\

\tab$\theta + 2\pi n = arcsin(\frac{y}{r})$\\

The arcsine is a {\bf relation}, but is not strictly a {\bf function}, since for every input value $\frac{y}{r}$, there is an infinite set of possible output values $\theta + 2\pi n$.  The typical approach to handling this shortcoming is to constrain the inverse trig functions to a certain range, called the {\bf principal-value range}.  By doing this, we create equivalent inverses that are functions.\\

\begin{figure}[htb!]
\center
\caption{Arc-sine constrained to its principal-value range.}
\label{fig:principal value arc sine}
\begin{tikzpicture}[inner sep=0pt,minimum size=0mm, scale = 0.7]

\node at (-4.6,3.1415) {$\pi$};
\node at (-4.6,1/2*3.1415) {$\frac{\pi}{2}$};
\node at (-4.6,0) {$0$};
\node at (-4.6,1/2*-3.1415) {-$\frac{\pi}{2}$};
\node at (-4.6,-3.1415) {$-\pi$};

\node[left] at (4, -4) {$4$};
\node[left] at (3, -4) {$3$};
\node[left] at (2, -4) {$2$};
\node[left] at (1, -4) {$1$};
\node[left] at (0, -4) {$0$};
\node[left] at (-1, -4) {$-1$};
\node[left] at (-2, -4) {$-2$};
\node[left] at (-3, -4) {$-3$};
\node[left] at (-4, -4) {$-4$};

\draw[ystep=3.1415/2, densely dashed] (-4,-1*3.1415) grid (4,1*3.1415);
\AXES{0}{0}{4}{1*3.1415}
\draw[thick, variable = \t, domain=1/2*-3.1415:1/2*3.1415,samples=250] plot ({sin(180*\t/3.1415)},{\t});

\end{tikzpicture}
\end{figure}

\begin{figure}[htb!]
\center
\caption{Arc-cosine constrained to its principal-value range.}
\label{fig:principal value arc-cosine }
\begin{tikzpicture}[inner sep=0pt,minimum size=0mm, scale = 0.7]

\node at (-4.6,3.1415) {$\pi$};
\node at (-4.6,1/2*3.1415) {$\frac{\pi}{2}$};
\node at (-4.6,0) {$0$};
\node at (-4.6,1/2*-3.1415) {-$\frac{\pi}{2}$};
\node at (-4.6,-3.1415) {$-\pi$};

\node[left] at (4, -4) {$4$};
\node[left] at (3, -4) {$3$};
\node[left] at (2, -4) {$2$};
\node[left] at (1, -4) {$1$};
\node[left] at (0, -4) {$0$};
\node[left] at (-1, -4) {$-1$};
\node[left] at (-2, -4) {$-2$};
\node[left] at (-3, -4) {$-3$};
\node[left] at (-4, -4) {$-4$};

\draw[ystep=3.1415/2, densely dashed] (-4,-1*3.1415) grid (4,1*3.1415);
\AXES{0}{0}{4}{1*3.1415}
\draw[thick, variable = \t, domain=0:3.1415,samples=250] plot ({cos(180*\t/3.1415)},{\t});

\end{tikzpicture}
\end{figure}


\begin{figure}[htb!]
\center
\caption{Arc-tangent contrained to its principal-value range.}
\label{fig:principal value arc-tan}
\begin{tikzpicture}[inner sep=0pt,minimum size=0mm, scale = 0.7]

\node at (-4.6, 3.1415) {$\pi$};
\node at (-4.6, 1/2*3.1415) {$\frac{\pi}{2}$};
\node at (-4.6, 0) {$0$};
\node at (-4.6, 1/2*-3.1415) {-$\frac{\pi}{2}$};
\node at (-4.6, -3.1415) {$-\pi$};

\node[left] at (4, -4) {$4$};
\node[left] at (3, -4) {$3$};
\node[left] at (2, -4) {$2$};
\node[left] at (1, -4) {$1$};
\node[left] at (0, -4) {$0$};
\node[left] at (-1, -4) {$-1$};
\node[left] at (-2, -4) {$-2$};
\node[left] at (-3, -4) {$-3$};
\node[left] at (-4, -4) {$-4$};

\draw[ystep=3.1415/2, dashed] (-4,-3.1415) grid (4,3.1415);
\AXES{0}{0}{4}{3.1415}

\begin{scope}
    \clip(-4, -3.1415) rectangle (4, 3.1415);

\draw[thick,
 variable = \t, 
 domain=1/2*-3.1415+0.01:1/2*3.1415-0.01,
 samples=30] 
plot ({tan(180*\t/3.1415)}, {\t});

\end{scope}

\end{tikzpicture}
\end{figure}

The following table lists the principal-value range for the inverse trigonometric functions.\\

\begin{figure}[htb!]
\caption{Principal-value range for inverse trigonometric functions.}
\label{fig:principal_value_range}
\begin{center}
\begin{tabular}{ | c | c | }
\hline 
function & principal-value range\\
\hline 
$arcsin(\theta)$ & $-\pitwo \leq \theta \leq \pitwo$ \\
\hline 
$arccos(\theta)$ & $0 \leq \theta \leq \pi$ \\
\hline 
$arctan(\theta)$ & $-\pitwo < \theta < \pitwo$ \\
\hline 
$arcsec(\theta)$ & $0 \leq \theta \leq \pi, \theta \neq \pitwo$ \\
\hline 
$arccsc(\theta)$ & $-\pitwo \leq \theta \leq \pitwo, \theta \neq 0$ \\
\hline 
$arccot(\theta)$ & $0 < \theta < \pi$ \\
\hline
\end{tabular}
\end{center}
\end{figure}

\clearpage
\subsection{Review}

\begin{enumerate}

\item{What is the definition of the inverse of a function?}\\

\item{Write the inverse of each of the following functions:}\\

\tab a) $f(x) = x + 1$\\

\tab b) $f(x) = x^2$\\

\tab c) $f(x) = \frac{1}{x}$\\

\tab d) $f(x) = sin(x)$\\

\tab e) $f(x) = 1$\\

\item{What is the difference between a relation and a function?}\\

\item{Why do inverse trig. functions have a principal-value range?}\\

\item{Draw and label a graph of the arc-sine, arc-cosine, and arc-tangent, without referring back to the text.}\\

\end{enumerate}

\section{Two Angles}

\subsection{Sum of two angles}

A relationship exists between the sine of the sum of angles.  Memorize these formulae, as they are very useful.  There are methods to derive this relationship using geometry, but a more succint proof can be found using complex trigonometry, which will be shown in Chapter \ref{sec:complex_trigonometry}.\\

\tab$sin(\alpha \pm \beta) = sin(\alpha)cos(\beta) \pm cos(\alpha)sin(\beta)$\\

\tab$cos(\alpha \pm \beta) = cos(\alpha)cos(\beta) \mp sin(\alpha)sin(\beta)$\\

\tab$tan(\alpha \pm \beta) = \frac{tan(\alpha) \pm tan(\beta)}{1 \mp tan(\alpha)tan(\beta)}$\\

\subsection{Double angles}

From the previous formulae, we can easily find equations for the sine, cosine, and tangent of twice some angle, by letting $\alpha = \beta$.\\

\tab$sin(2\alpha) = 2sin(\alpha)cos(\alpha)$\\

\tab$cos(2\alpha) = cos^2(\alpha) - sin^2(\alpha)$\\

\tab$tan(2\alpha) = \frac{2tan(\alpha)}{1-tan^2(\alpha)}$

\subsection{Half-angles}

Using the double angle formulae for the cosine, and the Pythagorean identitiy, we can also derive equations for the half-angle of the sine and cosine.\\

\tab$cos(2\alpha) = cos^2(\alpha) - sin^2(\alpha)$\\

\tab$\implies cos(\alpha) = cos^2(\frac{\alpha}{2}) - sin^2(\frac{\alpha}{2})$\\

\tab$\implies cos(\alpha) = cos^2(\frac{\alpha}{2}) - (1 - cos^2(\frac{\alpha}{2}))$\\

\tab$\implies cos(\alpha) = 2cos^2(\frac{\alpha}{2}) -  1$\\

\tab$\implies \frac{1 + cos(\alpha)}{2} = cos^2(\frac{\alpha}{2})$\\

\tab$\implies cos(\frac{\alpha}{2}) = (\frac{1 + cos(\alpha)}{2})^{\frac{1}{2}}$\\

Likewise, for the half-angle sine:\\

\tab$cos(2\alpha) = cos^2(\alpha) - sin^2(\alpha)$\\

\tab$\implies cos(\alpha) = cos^2(\frac{\alpha}{2}) - sin^2(\frac{\alpha}{2})$\\

\tab$\implies cos(\alpha) = (1 - sin^2(\frac{\alpha}{2})) - sin^2(\frac{\alpha}{2})$\\

\tab$\implies cos(\alpha) = 1- 2sin^2(\frac{\alpha}{2})$\\

\tab$\implies \frac{1 - cos(\alpha)}{2} = sin^2(\frac{\alpha}{2})$\\

\tab$\implies sin(\frac{\alpha}{2}) = (\frac{1 - cos(\alpha)}{2})^{\frac{1}{2}}$\\

Frome these two, we can derive the half-angle formulae for tangent.\\

\tab$tan(\frac{\alpha}{2}) = \frac{sin(\frac{\alpha}{2})}{cos(\frac{\alpha}{2})}$\\

\tab$ = \frac{ (\frac{1 - cos(\alpha)}{2})^{\frac{1}{2}}}{ (\frac{1 + cos(\alpha)}{2})^{\frac{1}{2}}}$\\

\tab$ = (\frac{\frac{1 - cos(\alpha)}{2}}{\frac{1 + cos(\alpha)}{2}})^{\frac{1}{2}}$\\

\tab$ = (\frac{1 - cos(\alpha)}{1 + cos(\alpha)})^{\frac{1}{2}}$\\

\subsection{Review}

\begin{enumerate}

\item{Write down the formulae for the sine, cosine, and tangent of the sum of two angles until you've committed them to memory.}\\

\item{Using the previous formulae, derive the double angle formulae for sine, cosine, and tangent}

\item{Using the previous formulae, derive the half angle formulae for sine, cosine, and tangent}

\end{enumerate}

\section{Polar Coordinates}

\subsection{Definition}

In a rectangular coordinate system, every point in space can be represented by a list of two numbers: the x-coordinate and the y-coordinate of the point.  Put another way, for any point in a 2-dimensional space, we can represent that point by the tuple $(x,y)$.  A {\bf tuple} is an ordered list of numbers.\\

It is also possible to represent every point in a 2-dimensional space using angles.  Let us choose some point in space.  Make a line between this point and the origin (this figure should look familiar).  The length $r$ of the line, and the angle $\theta$ of the line contain enough information to identify that point in space.  Put another way, for any point in a 2-dimensional space, we can represent that point by the tuple $(r,\theta)$.  This is the {\bf polar coordinate system}, where every point is represented by a distance and an angle instead of an x and a y.\\

\begin{figure}[htb]
\center
\caption{Rectangular and polar coordinates.}
\label{fig:rectangular and polar coordinates}
\begin{tikzpicture}[inner sep=0pt,minimum size=0mm]
\node at (0,4.5) {};

\PAXES{0}{0}{4}{3.25}
\LANGLE{0,0}{4}{0}{45}{$\theta$}

\node at (-0.25,1.5) {$y$};
\draw[dotted] (0,4*0.7071) -- (4*0.7071,4*0.7071);
\draw[very thick,->] (0,0) -- (0,4*0.7071);

\node at (1.5,-0.25) {$x$};
\draw[dotted] (4*0.7071,0) -- (4*0.7071,4*0.7071);
\draw[very thick,->] (0,0) -- (4*0.7071,0);

\node at (1.5,1.8) {$r$};

\node at (4, 3.5) {rectangular: $(x,y)$};
\node at (4, 3) {polar: $(r,\theta)$};

\end{tikzpicture}
\end{figure}

\subsection{Polar to rectangular}

Converting polar coordinates to rectangular coordinates is simple using sines and cosines.  By definition, \\

\tab$sin(\theta) = \frac{y}{r}$  \ \ \ and \ \ \  $cos(\theta)=\frac{x}{r}$\\

Through rearrangement,\\

\tab$y = rsin(\theta)$ \ \ \ and \ \ \ $x = rcos(\theta)$

\begin{figure}[htb!]
\center
\caption{Polar coordinates to rectangular coordinates.}
\label{fig:Polar coordinates to rectangular coordinates.}
\begin{tikzpicture}[inner sep=0pt,minimum size=0mm]
\node at (0,4) {};

\PAXES{0}{0}{4}{3.25}
\LANGLE{0,0}{4}{0}{45}{$\theta$}

\node at (-0.75,1.5) {$r sin(\theta)$};
\draw[dotted] (0,4*0.7071) -- (4*0.7071,4*0.7071);
\draw[very thick,->] (0,0) -- (0,4*0.7071);

\node at (1.5,-0.25) {$r cos(\theta)$};
\draw[dotted] (4*0.7071,0) -- (4*0.7071,4*0.7071);
\draw[very thick,->] (0,0) -- (4*0.7071,0);

\node at (1.5,1.8) {$r$};

\end{tikzpicture}
\end{figure}

\subsection{Rectangular to polar}

Converting rectangular coordinates is also simple.  Using the Pythagorean theorem, \\

\tab$r = \sqrt{x^2 + y^2}$\\

and using our results from deriving inverse trigonometric functions, \\

\tab$\theta = arcsin(\frac{y}{r})$ \ \ or \ \ $\theta = arccos(\frac{x}{r})$ \ \ or \ \ $\theta = arctan(\frac{y}{x})$

\begin{figure}[htb!]
\center
\caption{Rectangular coordinates to polar coordinates.}
\label{fig:Rectangular coordinates to polar coordinates.}
\begin{tikzpicture}[inner sep=0pt,minimum size=0mm]
\node at (0,4) {};

\PAXES{0}{0}{4}{3.25}
\LANGLE{0,0}{4}{0}{45}{}
\node at (30:5) {$arcsin(\frac{y}{r})$};
\node at (20:5) {$arccos(\frac{x}{r})$};
\node at (10:5) {$arctan(\frac{y}{x})$};

\node at (-0.25,1.5) {$y$};
\draw[dotted] (0,4*0.7071) -- (4*0.7071,4*0.7071);
\draw[very thick,->] (0,0) -- (0,4*0.7071);

\node at (1.5,-0.25) {$x$};
\draw[dotted] (4*0.7071,0) -- (4*0.7071,4*0.7071);
\draw[very thick,->] (0,0) -- (4*0.7071,0);

\node at (1,2) {$\sqrt{x^2+y^2}$};

\end{tikzpicture}
\end{figure}

\subsection{Review}

\begin{enumerate}

\item{Write the conversions from rectangular to polar coordinates, and the conversions from polar to rectangular coordinates, until you've committed them to memory.}\\

\item{Convert the following coordinates from polar to rectangular:}\\

\tab a) $(1,0^c)$\\

\tab b) $(2,\frac{\pi}{4}^c)$\\

\tab c) $(1,115^o)$\\

\tab d) $(2,-30^o)$\\

\tab e) $(0,0^o)$\\

\item{Convert the following coordinates from rectangular to polar:}\\

\tab a) $(1,0)$\\

\tab b) $(1,1)$\\

\tab c) $(0,1)$\\

\tab d) $(2,3)$\\

\tab e) $(0,0)$\\

\end{enumerate}


\section{Triangles}

Intuition should suggest that there is a fundamental relationship between the lengths of the sides of a triangle, and its interior angles.  There simply aren't enough degrees of freedom in a triangle to allow every side and every angle to be independent, so there must be some relationship between the two.  Using trigonometry, we find that there are indeed two quite useful relationships, known as the law of sines and the law of cosines.\\

\subsection{Properties of triangles}

A quick review of geometric properties of trianges:\\

\begin{enumerate}

\item{A triangle has three sides, and three interior angles.}\\

\item{A the sum of the interior angles of a triangle is $180\degree$, or $\pi\rad$.}\\

\item{The length of any side of a triangle, is less than the sum of the lengths of the other two sides.  For a triangle with sides of lengths $A, B, C$:\\$A < B + C$, $B < A + C$, and $C < A + B$.}\\

\item{A triangle with one interior angle of $90\degree$, or $\frac{\pi}{2}\rad$ is called a {\bf right triangle}.}\\

\item{A triangle with three equal interior angles is called an {\bf equilateral triangle}.}\\

\item{A triangle with two equal interior angles is called an {\bf isocoles triangle}.}\\

\item{A triangle with no equal interior angles is called a {\bf scalene triangle}.}\\

\end{enumerate}

\subsection{Law of sines}

Let us examine a triangle with sides of length $A$, $B$, and $C$, with opposing interior angles $a$, $b$, and $c$.\\

\begin{figure}[htb]
\center
\caption{A triangle.}
\label{fig:A triangle}
\begin{tikzpicture}[inner sep=0pt,minimum size=0mm]

\node () at (0,3.5) {};

\draw[] (0,0) -- (2,1);
\draw[] (2,1) -- (1,3);
\draw[] (1,3) -- (0,0);

\node () at (0.4,0.5){$a$};
\node () at (1.6,1.15) {$b$};
\node () at (1.05,2.4) {$c$};

\node () at (1.9,2.1){$A$};
\node () at (0.15,1.5) {$B$};
\node () at (1.25,0.25) {$C$};

\end{tikzpicture}
\end{figure}


The {\bf law of sines} states that for a triangle like this, with sides of length $A$, $B$, and $C$, and opposing interior angles $a$, $b$, and $c$, the ratios of the lengths of a side to the sine of the corresponding angle are all equal.  That is:\\

\tab$\frac{sin(a)}{A} = \frac{sin(b)}{B} = \frac{sin(c)}{C}$\\

This holds true for all angles of a triangle, whether acute, right, or obtuse.  A derivation of the Law of Sines can be found in the appendix.\\

\subsection{Law of cosines}

\begin{figure}[htb]
\center
\caption{A triangle.}
\label{fig:A triangle}
\begin{tikzpicture}[inner sep=0pt,minimum size=0mm]

\node () at (0,3.5) {};

\draw[] (0,0) -- (2,1);
\draw[] (2,1) -- (1,3);
\draw[] (1,3) -- (0,0);

\node () at (0.4,0.5){$a$};
\node () at (1.6,1.15) {$b$};
\node () at (1.05,2.4) {$c$};

\node () at (1.9,2.1){$A$};
\node () at (0.15,1.5) {$B$};
\node () at (1.25,0.25) {$C$};

\end{tikzpicture}
\end{figure}

The {\bf law of cosines} is equally as useful as the law of sines.  The law  of cosines allows you to, given two sides of a triangle and their interior angle, find the length of the third side.  For a triangle of sides $A$, $B$, and $C$, and interior angles of $a$, $b$, and $c$, the law of cosines states that\\

\tab$A^2 = B^2 + C^2 - 2BCcos(a)$\\

This equation should look familiar.  When a is an angle of $\pitwo$ (a right triangle), the cosine term is zero, and the equation reduces to \\

\tab$A^2 = B^2 + C^2$\\

In fact, the law of cosines is a gereralized form of the Pythagorean theorem, for triangles of any angle.  A derivation of the law of cosines can also be found in the appendix.\\

\subsection{Review}

{\begin{figure}[htb]
\center
\caption{A triangle.}
\label{fig:A triangle}
\begin{tikzpicture}[inner sep=0pt,minimum size=0mm]

\node () at (0,2.5) {};

\draw[] (0,0) -- (2,0);
\draw[] (0,0) -- (1,1.866);
\draw[] (2,0) -- (1,1.866);

\node () at (0.25,0.125){$a$};
\node () at (1.0,1.5) {$b$};
\node () at (1.75,0.125) {$c$};

\node () at (1.75,0.9){$A$};
\node () at (0.25,0.9) {$C$};
\node () at (1.0,-0.25) {$B$};

\end{tikzpicture}
\end{figure}
}

\begin{enumerate}

\item{Given $a=b=c=60\degree$ and $A=1$, find $B$ and $C$.}\\

\item{Given $a=35\degree$, $b=55\degree$, and $B=2$, find $c$, $B$, and $C$.}\\

\item{Given $a=45\degree$, $B=2$, and $C=3$, find $A$, $b$, and $c$.}\\

\item{Given $A=5$, $B=7$, $C=13$, find $a$, $b$, and $c$.}\\

\end{enumerate}



\section{Complex Numbers}
\label{sec:complex_numbers}

\subsection{Imaginary numbers}

Does a negative number really exist?  Intuitively, it is difficult to imagine any physical quantity that is truly negative.  You can hold one or two rocks in your hand, or even zero.  All of these are real physical qualities, but holding $-1$ rocks seems implausible.  You can't walk less than zero miles, or hold your breath for fewer than zero seconds.\\

I posit that negative quantities do not really exist, but are instead just an artifact of addition.  Negative numbers exist as book-keeping, helping us describe the mathematics of addition completely.\\

Likewise, there is another book-keeping artifact known as and {\bf imaginary number}.  An imaginary number is defined as the square root of a negative number.  This is a seemingly impossible operation, but using some careful rearrangement we can simplify the expression.\\

\tab$\sqrt{-16}$\\

\tab$ = \sqrt{16 * -1}$\\

\tab$ = \sqrt{16} * \sqrt{-1}$\\

\tab$ = 4\sqrt{-1}$\\

Using this same technique, we can rewrite the square root of any negative number as the product of a real number and the $\sqrt{-1}$.  This is so commonplace that the symbol $i$ is standard notation for $\sqrt{-1}$.  Thus, \\

\tab$\sqrt{-16} = 4i$ \ \  where \ \  $i = \sqrt{-1}$\\

"Imaginary" is a bit misleading, as again, they are no more imaginary than negative numbers, but the term "imaginary" does illustrate an important point:  You will never see a real, physical quantity that contains an imaginary number.\\

\subsection{Imaginary arithmetic}

Imaginary numbers can be added together, or subtracted from one another, and the result is imaginary.\\

\tab$2i + 4i = 6i$ \ \ and \ \ $2i - 4i = -2i$\\ 

Imaginary numbers can be multiplied together, and the result will be real, but negated.\\

\tab$ i * i = \sqrt{-1}\sqrt{-1} = -1 $\\

\tab$2i * 4i = 2*4*(i*i) = 2*4*(-1) = -8$\\

\tab$-2i * 4i = -2*4*(i*i) = -2*4*(-1) = 8$\\

Imaginary numbers may also be divided, they will reduce accordingly\\

\tab$\frac{4i}{2i} = 2$\\

Interestingly enough, we can also multiply and divide a real number by an imaginary number.  Multiplication is trivial.\\

\tab$4*2i = 8i$\\

To understand division by an imaginary number, we must look again at the definition of $i$.\\

\tab$i*i = -1$\\

\tab$\implies \frac{i*i}{i} = \frac{-1}{i}$\\

\tab$\implies i = \frac{-1}{i}$\\

\tab$ \implies \frac{1}{i} = -i$\\

Thus, dividing by $i$ is equivalent to multiplying by $-i$.\\

\subsection{Complex numbers}


While we can simplify the result of multiplying or dividing a real number by an imaginary number, there is no meaningful way to add a real number to an imaginary number, and simplify the result.  Therefore, the simplest notation for the sum of a real number and an imaginary number is $a + bi$.  This is known as a {\bf complex number}, a number with both a real component and an imaginary component.\\

\subsection{Complex arithmetic}

  We can perform all the basic arithmetic operations on complex numbers.  Addition and subtraction are simple, we just add or subtract the corresponding real and imaginary components.\\

\tab$(a + bi) + (c + di) = (a+c) + (b+d)i$\\

\tab$(a + bi) - (c + di) = (a-c) + (b-d)i$\\

Multiplying two complex numbers is more involved; since there are two components in each number, we must use the distributive property.\\

\tab$(a + bi) * (c + di)$\\

\tab$= a*(c + di ) + bi * (c + di)$\\

\tab$= ac + adi + bci +bd(i*i)$\\

\tab$= (ac - bd) + (ad + bc)i$\\

Note that if $ad + bc = 0$, the imaginary component is eliminated and we are left with a real result.  This is an important property because, given a complex number $z_1 = a + bi$, we can alwasy construct another complex number $z_2$, such that the product $z_1z_2$ is real.  If $a = c$ and $d = -b$, the imaginary parts will cancel and the result will be real.  This specific number is called the {\bf complex conjugate} of a complex number.\\

For any complex number $a+bi$, the exists its complex conjugate $a-bi$, and $(a+bi)(a-bi)$ is real.  In fact $(a+bi)(a-bi) = a^2 + b^2$.\\

The complex conjugate is particularly useful is simplifying the quotient of two complex numbers.  By multiplying the numberator and denominator by the complex conjugate of the denominator, we can reduce the denominator to a real number, and simplify.\\

\tab$\frac{a+bi}{c+di}$\\

\tab$= (\frac{a+bi}{c+di})(\frac{c-di}{c-di})$\\

\tab$= \frac{(a+bi)(c-di)}{(c+di)(c-di)}$\\

\tab$= \frac{(a+bi)(c-di)}{c^2+d^2}$\\

\tab$= \frac{(ac+bd) + (bc-ad)i}{c^2+d^2}$\\

\tab$= \frac{ac+bd}{c^2+d^2} + \frac{bc-ad}{c^2+d^2}i$\\

In summary, we can see that the sum, difference, product, or quotient of two complex numbers can always be simplified to a simple complex number of the form $X + Yi$.\\

\subsection{Polar form}

It's convenient that complex numbers have two components, the real part and the imaginary part, and that they act as independent variables inside the number.  In fact, if we create a rectangular coordinate system where the x axis is treated as the "real" axis, and the y axis is treated as the "imaginary" axis, we can represent every complex number as a point in a 2-dimensional space.\\

Here's where it gets interesting:  just because a point is in a 2d space doesn't mean we have to use a rectangular coordinate system.  We previously derived the polar coordinate system, and it turns out we can also represent complex numbers in a polar form.  Using the conversion from rectangular to polar, we can find the "radius" $r$ of a complex number $X + Yi$ is $r = \sqrt{X^2 + Y^2}$, and the "angle" $\theta$ of a complex number is $\theta = arctan(\frac{Y}{X})$.\\

The mathematical term for $r$ is the {\bfseries modulus} of a complex number, and $\theta$ is the {\bfseries argument} of a complex number.  A complex number is typically written in polar form as $r\angle\theta$.\\

Keep in mind that a point doesn't move when we change coordinate systems, it is at the same location regardless of whether we map it to a rectangular coordinate system or a polar coordinate system.  Likewise when we convert a complex number from rectangular to polar form, or vice versa, we are not changing the {\bf value} of the number, just how we describe it.\\

\subsection{Polar arithmetic}

Complex numbers in polar form are not easy to add or subtract;  to add or subtract two complex numbers, you should convert them to rectangular form first.  However, multiplying and dividing polar complex numbers is significantly easier in polar form:\\

\tab$r_1\angle\theta_1 * r_2\angle\theta_2 = (r_1r_2)\angle(\theta_1 + \theta_2)$ \\

\tab$\frac{r_1\angle\theta_1}{r_2\angle\theta_2} = (\frac{r_1}{r_2})\angle(\theta_1 - \theta_2)$ \\

\subsection{Summary}

In summary:\\

To convert a complex number from rectangular to polar form:\\

\tab$a+bi = (\sqrt{a^2+b^2})\angle (arctan(\frac{b}{a}))$\\

To convert a complex number from polar to rectangular form:\\

\tab$r\angle\theta = rcos(\theta) + r i sin(\theta)$\\

To add or subtract two complex numbers, convert them to rectangular form, and add or subtract the real and imaginary parts.\\

\tab$(a+bi) \pm (c+di) = (a \pm c) + (b \pm d)i$\\

To multiply two complex numbers in rectangular form, distribute the product and simplify.\\

\tab$(a + bi) * (c + di)$\\

\tab$= a * (c + di) + bi * (c + di)$\\

\tab$= ac + adi + bci + bdi^2$\\

\tab$= (ac - bd) + (ad + bc)i$\\

To multiply two complex numbers in polar form, multiply the moduli and add the arguments.\\

\tab$r_1\angle\theta_1 * r_2\angle\theta_2 = (r_1r_2)\angle(\theta_1+\theta_2)$\\

To divide one complex number by another in rectangular form, multiply both the numberator and denominator by the complex conjugate of the denominator, and simplify.\\

\tab$\frac{a+bi}{c+di}$\\

\tab$= (\frac{a+bi}{c+di})(\frac{c-di}{c-di})$\\

\tab$= \frac{(a+bi)(c-di)}{(c+di)(c-di)}$\\

\tab$= \frac{(a+bi)(c-di)}{c^2+d^2}$\\

\tab$= \frac{(ac+bd) + (bc-ad)i}{c^2+d^2}$\\

\tab$= \frac{ac+bd}{c^2+d^2} + \frac{bc-ad}{c^2+d^2}i$\\

To divide one complex number by another in polar form, divide the moduli and subtract the arguments.\\

\tab$\frac{r_1\angle\theta_1}{r_2\angle\theta_2} = (\frac{r_1}{r_2})\angle(\theta_1 - \theta_2)$ \\

\subsection{Review}

\begin{enumerate}

\item{Convert these complex numbers to polar form: \\ $1,\  i ,\  -1,\  -i,\  1+i,\  i-1,\  -i-1,\  1-i,\  2+i,\  2+2i$}

\item{Convert these complex numbers to rectangular form: \\ $1\angle0,\   1\angle90\degree,\  2\angle\frac{\pi}{6}\rad,\   3\angle-60\degree,\  1\angle\pi\rad,\  -2\angle\frac{\pi}{4},\  0\angle90\degree,\  0\angle20\degree$}

\item{Simplify the following: \\ $(1+i)+(2+2i),\ (3+3i)-(1-i),\ (1\angle\frac{3\pi}{4}\rad)-i,\  (1\angle45\degree)+(1\angle-45\degree), $}

\item{Simplify the following: \\ $(1+i)(1-i),\ (2+i)(3+3i),\ (1+i)(1\angle\frac{\pi}{4}\rad),\ (2\angle135\degree)(2\angle210\degree)$}

\item{Simplify the following: \\ $\frac{1+i}{1-i},\ \frac{2+2i}{-3-i},\ \frac{1\angle\frac{\pi}{3}\rad}{1+i},\ \frac{1\angle30\degree}{1\angle-60\degree}$}

\end{enumerate}
\section{Complex Trigonometry}
\label{sec:complex_trigonometry}

\subsection{Euler's formula}

{\bf Euler's formula} (pronounced Oiler) is the pinnacle of trigonometry.  It provides an intuitive method for understanding nearly everything covered in this book.  Unfortunatly, every useful proof of this formula requires calculus, so you will have to take it on faith that it is true, for now.  Euler's formula is:\\

\tab$e^{i\theta} = cos(\theta) + isin(\theta)$\\

That's it.  A single formula that encapsulates almost every idea in trigonometry.  If there's one thing to walk away from this book knowing, its Euler's formula.\\

\subsection{Phasors}

If you look closely at Euler's formula, you should notice some familiarities from earlier sections.  The right hand side of the equation looks an awful lot like a polar complex number converted to rectangular form, right?\\

\tab$r\angle\theta = r*(cos(\theta) + isin(\theta))$\\

In fact, if we multiplied Euler's formula by a modulus, we would see

\tab$re^{i\theta} = rcos(\theta) + r i sin(\theta)$\\

What this implies, interestingly enough, is that there is an actual mathematical formula for writing a complex number in polar form.  This is called the {\bf phasor form} of a complex number, or a {\bf phasor}.\\

Put another way, $r\angle\theta$ is another way of writing $re^{i\theta}$, and Euler's formula is another way of saying that a complex number in rectangular form is the same number in polar form, just written a different way.\\

\subsection{Another look at polar arithmetic}

Using Euler's formula and phasors, we can prove many previous properties of trigonometry and complex numbers. Let's look at multiplying two complex numbers in phasor form:\\

\tab$r_1e^{i\theta_1} * r_2e^{i\theta_2}$\\

\tab$= r_1*r_2 * e^{i\theta_1} * e^{i\theta_2}$\\

\tab$= r_1*r_2 * e^{i\theta_1 + i\theta_2}$\\

\tab$= r_1r_2e^{i(\theta_1 + \theta_2)}$\\

This should look familiar, this is how we described multiplying complex numbers in polar form previously.\\


\subsection{Another look at negative angles}

How about negative angles?  What happens if we put a netagive angle into Euler's formula?\\

\tab$e^{i(-\theta)} = cos(-\theta) + isin(-\theta)$\\

But, from the properties of exponents, we know that\\

\tab$e^{i(-\theta)} = e^{i\theta * -1} = (e^{i\theta})^{-1} = \frac{1}{e^{i\theta}}$\\

and\\

\tab$\frac{1}{e^{i\theta}} = \frac{1}{cos(\theta) + isin(\theta)}$\\

Note that this fraction consists of complex numbers in rectangular form.  We can simplify this expression by multiplying the numberator and denominator by the complex conjugate of the denominator.\\

\tab$\frac{1}{cos(\theta) + isin(\theta)}$\\

\tab$=  (\frac{1}{cos(\theta) + isin(\theta)})(\frac{cos(\theta) - isin(\theta)}{cos(\theta) - isin(\theta)})$\\

\tab$=  \frac{cos(\theta) - isin(\theta)}{(cos(\theta) + isin(\theta))(cos(\theta) - isin(\theta))}$\\

\tab$=  \frac{cos(\theta) - isin(\theta)}{cos^2(\theta) + sin^2(\theta)}$\\

\tab$= cos(\theta) - isin(\theta)$ , by the Pythagorean identity.\\

From this, we can see\\

\tab$cos(\theta) - isin(\theta) = \frac{1}{e^{i\theta}} = e^{i(-\theta)} = cos(-\theta) + isin(-\theta)$\\

Thus\\

\tab$cos(-\theta) + isin(-\theta) = cos(\theta) - isin(\theta)$\\

We know that if two complex numbers are equal, their real and imaginary parts are equal, thus we can say:\\

\tab$cos(-\theta) = cos(\theta)$ \ \ \ and  \ \ \ $sin(-\theta) = -sin(\theta)$\\

\subsection{Another look at the Pythagorean Identity}

We can also rearrange the previous proof to derive the Pythagorean identity from Euler's formula.  Now we know\\

\tab$e^{i\theta} = cos(\theta) + isin(\theta)$\\

and\\

\tab$e^{-i\theta} = cos(\theta) - isin(\theta)$\\

From this,\\

\tab$e^{i\theta}e^{-i\theta} = (cos(\theta) + isin(\theta))(cos(\theta) - isin(\theta))$\\

\tab$e^{i\theta - i\theta} = cos^2(\theta) + sin^2(\theta)$\\

\tab$e^0 = cos^2(\theta) + sin^2(\theta)$\\

\tab$sin^2(\theta) + cos^2(\theta) = 1$\\

We can also derive formulae for the sine and cosine of an angle in terms of phasors.\\

\tab$e^{i\theta} + e^{-i\theta}$\\

\tab$= cos(\theta) + isin(\theta) +  cos(\theta) - isin(\theta)$\\

\tab$= 2cos(\theta)$\\

therefore $cos(\theta) = \frac{1}{2}(e^{i\theta} + e^{-i\theta})$

Likewise, for sine\\

\tab$e^{i\theta} - e^{-i\theta}$\\

\tab$= cos(\theta) + isin(\theta) -  cos(\theta) + isin(\theta)$\\

\tab$= 2isin(\theta)$\\

therefore $sin(\theta) = \frac{1}{2i}(e^{i\theta} - e^{-i\theta})$\\

\subsection{Another look at two angles}

Using the phasor forms of sine and cosine, we can also derive the equations  sine and cosine of the sum or difference of two angles.\\

\tab$sin(\theta_1 + \theta_2) = \frac{1}{2i}(e^{i(\theta_1+\theta_2)} - e^{-i(\theta_1+\theta_2)})$\\

\tab$= \frac{1}{2i}(e^{i\theta_1}e^{i\theta_2} - e^{-i\theta_1}e^{-i\theta_2})$\\

by substitution,\\

\tab$e^{i\theta_1}e^{i\theta_2} = (cos(\theta_1)+isin(\theta_1)(cos(\theta_2) + isin(\theta_2))$\\

\tab$= cos(\theta_1)cos(\theta_2) - sin(\theta_1)sin(\theta2) + icos(\theta_1)sin(\theta_2) + isin(\theta_1)cos(\theta_2)$\\

and\\

\tab$e^{-i\theta_1}e^{-i\theta_2} = (cos(\theta_1)-isin(\theta_1)(cos(\theta_2) - isin(\theta_2))$\\

\tab$= cos(\theta_1)cos(\theta_2) - sin(\theta_1)sin(\theta2) - icos(\theta_1)sin(\theta_2) - isin(\theta_1)cos(\theta_2)$\\

thus\\

\tab$e^{i\theta_1}e^{i\theta_2} - e^{-i\theta_1}e^{-i\theta_2} = 2isin(\theta_1)cos(\theta_2) + 2icos(\theta_1)sin(\theta_2)$\\

and\\

\tab$\frac{1}{2i}(e^{i\theta_1}e^{i\theta_2} - e^{-i\theta_1}e^{-i\theta_2}) = sin(\theta_1)cos(\theta_2) + cos(\theta_1)sin(\theta_2)$\\

\tab$sin(\theta_1 + \theta_2) = sin(\theta_1)cos(\theta_2) + cos(\theta_1)sin(\theta_2)$\\

\subsection{Conclusion}

The above sections illustrate a very important point:  {\bf You can use Euler's formula to derive all of Trigonometry.}  Forgot the double-angle formula for sine?  Use Euler's formula.  Can't remember the phase offset between sine and cosine?  Use Euler's formula.\\

It's rare that an entire subject can be succinctly described with one equation, so take advantage of it when you can.  Don't forget Euler's formula.

\subsection{Review}

\begin{enumerate}

\item{What is Euler's formula?}

\item{Use Euler's formula to derive the double-angle sine formula: \\ $sin(2\theta)=?$}

\item{Use Euler's formula to find the phase offset between sine and cosine: \\ $sin(\theta + x) = cos(\theta) \implies ?$}

\end{enumerate}
\section{Appendix A: Solutions to Problems}

\subsection{Chapter 2}


\begin{enumerate}
\item {{\bf In the beginning of this chapter, I defined an angle as a measure of 'rotational distance', but I didn't state what a rotation actually is.  So, what is a rotation?  Do some research, and come up with a definition for a rotation that you find satisfactory.\\}
{There are several definitions for a rotation, but all include two important properties:  A rotation is {\bf centered around a point}; this point remains unchanged by the rotation.  The other important property is that a rotation is {\bf distance-preserving}.  Any two points will be the same distance apart after the rotation as they were before the rotation.}}



\item{{\bf What is the mathematical definition of an angle?  Write this down until you can recall it without referring back to the chapter.\\}
{An angle is the ratio of an arc length to its radius.  $\theta = \frac{l}{r}$.}
}

\item{{\bf What is the definition of a radian, and what is the definition of a degree?  Why would we have two different units of measure for an angle?\\}
{A radian is the natural unit of measure of an angle.  It is the ratio of an arc length to its radius.  A degree is $\frac{1}{360}$ of a circle.  The key diference between the two is that the radian is derived from a ratio, and the degree is derived from the circle.  One unit may be more convenient to use than the other, depending on the problem.}}

\item{{\bf What is the conversion ratio for degrees to radians?  For radians to degrees?\\}
{There are $\frac{180}{\pi}$ degrees in a radian, and $\frac{\pi}{180}$ radians in a degree.}}

\item{{\bf Convert the following values in radians to degrees: \pisix, \pifour, \pithree, \pitwo, \twopithree, \threepifour, \fivepisix, $\pi$, \sevenpisix, \fivepifour, \fourpithree, \threepitwo, \fivepithree, \sevenpifour, \elevenpisix, $2\pi$.\\}{See Figure \ref{fig:table_of_angles}.}}

\item{{\bf Convert the following values in degrees to radians: 30, 45, 60, 90, 120, 135, 150, 180, 210, 225, 240, 270, 300, 315, 330, 360.}\\
{See Figure \ref{fig:table_of_angles}.}}

\item{{\bf Convert the following values in radians to fractions of a circle:  \pisix, \pifour, \pithree, \pitwo, \twopithree, \threepifour, \fivepisix, $\pi$, \sevenpisix, \fivepifour, \fourpithree, \threepitwo, \fivepithree, \sevenpifour, \elevenpisix, $2\pi$.\\}
{See Figure \ref{fig:table_of_angles}.}}

\item{{\bf Convert the following values from degrees to fractions of a circle: $30$, $45$, $60$, $90$, $120$, $135$, $150$, $180$, $210$, $225$, $240$, $270$, $300$, $315$, $330$, $360$.\\}{See Figure \ref{fig:table_of_angles}.}}

\item{{\bf Classify each angle as full, straight, right, reflex, obtuse, or acute:  \pisix, \pifour, \pithree, \pitwo, \twopithree, \threepifour, \fivepisix, $\pi$, \sevenpisix, \fivepifour, \fourpithree, \threepitwo, \fivepithree, \sevenpifour, \elevenpisix, $2\pi$.\\}{See Figure \ref{fig:table_of_angles}.}}

\item{{\bf Find the complement of each angle in degrees: $30$, $45$, $60$, $90$, $120$, $135$, $150$, $180$, $210$, $225$, $240$, $270$, $300$, $315$, $330$, $360$.\\}{See Figure \ref{fig:table_of_angles}.\\}
Note that this brings up an interesting question:  Can angles greater than $90^o$ have a complement?  Can a negative angle have a complement?  It is implied in many sources that only angles between $0^o$ and $90^o$ can have a complement, but the strict definition(two angles whose sum is $90^o$) does not impose any such restriction.}

\item{{\bf Find the supplement of each angle in radians:  \pisix, \pifour, \pithree, \pitwo, \twopithree, \threepifour, \fivepisix, $\pi$, \sevenpisix, \fivepifour, \fourpithree, \threepitwo, \fivepithree, \sevenpifour, \elevenpisix, $2\pi$.\\}{See Figure \ref{fig:table_of_angles}.}}

\end{enumerate}
\begin{figure}[htb]
\caption{Table of angle conversions.}
\label{fig:table_of_angles}
\begin{center}
\begin{tabular}{ |c | c | c | c | c | c |}
\hline 
rad & deg & circle & type & comp. & supp. \\
\hline 
$\frac{\pi}{6}^c$ & $30^o$ & $\frac{1}{12}$ & acute & $60^o$ & $150^o$\\
\hline
$\frac{\pi}{4}^c$ & $45^o$ & $\frac{1}{8}$ & acute & $45^o$ & $135^o$\\
\hline
$\frac{\pi}{3}^c$ & $60^o$ & $\frac{1}{6}$ & acute & $30^o$ & $120^o$\\
\hline
$\frac{\pi}{2}^c$ & $90^o$ & $\frac{1}{4}$ & right & $0^o$ & $90^o$\\
\hline
$\frac{2\pi}{3}^c$ & $120^o$ & $\frac{1}{3}$ & obtuse & $-30^o$ & $60^o$\\
\hline
$\frac{3\pi}{4}^c$ & $135^o$ & $\frac{3}{8}$ & obtuse & $-45^o$ & $45^o$\\
\hline
$\frac{5\pi}{6}^c$ & $150^o$ & $\frac{5}{12}$ & obtuse & $-60^o$ & $30^o$\\
\hline
$\pi^c$ & $180^o$ & $\frac{1}{2}$ & straight & $-90^o$ & $0^o$\\
\hline
$\frac{7\pi}{6}^c$ & $210^o$ & $\frac{7}{12}$ & reflex & $-120^o$ & $-30^o$\\
\hline
$\frac{5\pi}{4}^c$ & $225^o$ & $\frac{5}{8}$ & reflex & $-135^o$ & $-45^o$\\
\hline
$\frac{4\pi}{3}^c$ & $240^o$ & $\frac{2}{3}$ & reflex & $-150^o$ & $-60^o$\\
\hline
$\frac{3\pi}{2}^c$ & $270^o$ & $\frac{3}{4}$ & reflex & $-180^o$ & $-90^o$\\
\hline
$\frac{5\pi}{3}^c$ & $300^o$ & $\frac{5}{6}$ & reflex & $-210^o$ & $-120^o$\\
 \hline
$\frac{7\pi}{4}^c$ & $315^o$ & $\frac{7}{8}$ & reflex & $-225^o$ & $-135^o$\\
\hline
$\frac{11\pi}{6}^c$ & $330^o$ & $\frac{11}{12}$ & reflex & $-240^o$ & $-150^o$\\
\hline
$2\pi^c$ & $360^o$ & $1$ & full & $-270^o$ & $-180^o$\\
\hline

\end{tabular}
\end{center}
\end{figure}






\clearpage
\subsection{Chapter 3}

\begin{enumerate}

\item{{\bf What is the definition of a trigonometric function?}\\
{A trigonometric function is any function of an angle.  In common usage, the term refers to functions derived from the projection of an angle.}}

\item{{\bf What is the definition of a projection?}\\
{A projection can be thought of as the "shadow" one line casts on another.  A more formal definition is the transformation of points from one line to another using parallel lines.}}

\item{{\bf What is the definition of sine?  of cosine?}\\
{Sine is the projection of an angle onto the y axis of a rectangular coordinate system.  Rotate a line by an angle, and sine is the ratio of the y-axis projection of that line to the length of the line.  Cosine is the projection of an angle onto the x axis.}}

\item{{\bf Which axis is the sine projected on?  The cosine?  Don't forget this!}\\
{Sine is the projection onto the y axis, and cosine is the projection onto the x axis.  Seriously, don't forget this.  Write it down 20 times on a scrap of paper.}}

\item{{\bf Draw an angle and its projections, then define the sine and cosine of that angle.}\\
{}}

\begin{figure}[htb!]
\center
\label{fig:projection_onto_axes}
\begin{tikzpicture}[inner sep=0pt,minimum size=0mm]
\node () at (0,2){};
\node () at (1.325,-0.85) {$x$};
\node () at (-0.8,0.75) {$y$};

\node () at (40:1.5) {$r$};

\LANGLE{0,0}{3}{0}{30}{$\theta$}
\AXES{0}{0}{3}{1.7}

\draw[|-|,line width=1pt] (-0.5,0) -- (-0.5,1.5);
\draw[|-|,line width=1pt] (0,-0.5) -- (1.5*1.732,-0.5);
\draw[dashed] (1.5*1.732,1.5) -- (0,1.5);
\draw (0,1.5-0.2) -- (0.2,1.5-0.2) -- (0.2,1.5) -- (0,1.5);
\draw[dashed] (1.5*1.732,0) -- (1.5*1.732,1.5);
\draw (1.5*1.732-0.2,0) -- (1.5*1.732-0.2,0.2) -- (1.5*1.732,0.2) -- (1.5*1.732,0.0);

\end{tikzpicture}
\end{figure}

\tab$sin(\theta) = \frac{y}{r}$ \tab$cos(\theta) = \frac{x}{r}$ \tab$tan(\theta) = \frac{y}{x}$\\


\item{{\bf What is the definition of tangent?  of secant?  of cosecant?  of cotangent?}\\
{Tangent is the ratio of sine to coisine.  Secant is the reciprocal of cosine.  Cosecant is the reciprocal of sine.  Cotangent is the reciprocal of tangent.}}

\item{{\bf Is secant the reciprocal of sine or cosine?  Don't forget this!}\\
{Secant is the reciprocal of cosine.}}

\item{{\bf Why can you also use a right triangle to define sine and cosine?}\\
{If the right triangle is aligned to the x and y axes, then the projections of the hypotenuse of the triangle are equal in magnitude to the other two lines.  This can be shown using parallel lines.}}

\item{{\bf Draw a right triangle, and write the length of the sides in terms of one angle and the length of the hypotenuse.}\\
{}}

\begin{figure}[htb!]
\center
\label{fig:projection_onto_axes}
\begin{tikzpicture}[inner sep=0pt,minimum size=0mm]
\node () at (0,1.5){};
\node () at (1.325,-0.3) {$r \ cos(\theta)$};
\node () at (3.3,0.75) {$r \ sin(\theta)$};

\node () at (40:1.5) {$r$};

\LANGLE{0,0}{1}{0}{30}{$\theta$}


\draw (0,0) -- (30:3);
\draw (0,0) -- (3*0.866,0);
\draw (3*0.866,0) -- (3*0.866,3*0.5);

\draw (1.5*1.732-0.2,0) -- (1.5*1.732-0.2,0.2) -- (1.5*1.732,0.2) -- (1.5*1.732,0.0);

\end{tikzpicture}
\end{figure}

\item{{\bf From the previous question, how do you know which side corresponds with sine, and which with cosine?}\\
{The leg of the triangle opposite the angle corresponds with sine, and the leg of the triangle adjacent to the angle corresponds with cosine.}}

\end{enumerate}







\clearpage
\subsection{Chapter 4}

{\bf Practice drawing this chart, and filling in the angles.  For every angle, write the value in degrees, in radians, and the sine, cosine, and tangent.  You should be able to draw and fill out the entire chart in under five minutes.}\\

\begin{figure}[htb]
\center
\caption{Simple angles in degrees.}
\label{fig:review of simple angles}
\begin{tikzpicture}[inner sep=0pt,minimum size=0mm]

\node at (0,2.75) {};

\node at (0:2.5) {$0^o, 360^o$};
\node at (30:2.25) {$30^o$};
\node at (45:2.25) {$45^o$};
\node at (60:2.25) {$60^o$};

\node at (90:2.25) {$90^o$};
\node at (120:2.25) {$120^o$};
\node at (135:2.25) {$135^o$};
\node at (150:2.25) {$150^o$};

\node at (180:2.25) {$180^o$};
\node at (210:2.25) {$210^o$};
\node at (225:2.25) {$225^o$};
\node at (240:2.25) {$240^o$};

\node at (270:2.25) {$270^o$};
\node at (300:2.25) {$300^o$};
\node at (315:2.25) {$315^o$};
\node at (330:2.25) {$330^o$};

\LLANGLE{0,0}{1.75}{0}{30}{}{0.5}
\LLANGLE{0,0}{1.75}{30}{45}{}{0.5}
\LLANGLE{0,0}{1.75}{45}{60}{}{0.5}
\LLANGLE{0,0}{1.75}{60}{90}{}{0.5}

\LLANGLE{0,0}{1.75}{90}{120}{}{0.5}
\LLANGLE{0,0}{1.75}{120}{135}{}{0.5}
\LLANGLE{0,0}{1.75}{135}{150}{}{0.5}
\LLANGLE{0,0}{1.75}{150}{180}{}{0.5}

\LLANGLE{0,0}{1.75}{180}{210}{}{0.5}
\LLANGLE{0,0}{1.75}{210}{225}{}{0.5}
\LLANGLE{0,0}{1.75}{225}{240}{}{0.5}
\LLANGLE{0,0}{1.75}{240}{270}{}{0.5}

\LLANGLE{0,0}{1.75}{270}{300}{}{0.5}
\LLANGLE{0,0}{1.75}{300}{315}{}{0.5}
\LLANGLE{0,0}{1.75}{315}{330}{}{0.5}
\LLANGLE{0,0}{1.75}{330}{360}{}{.5}

\end{tikzpicture}
\end{figure}

\begin{figure}[htb]
\center
\caption{Simple angles in radians.}
\label{fig:review of simple angles}
\begin{tikzpicture}[inner sep=0pt,minimum size=0mm]

\node at (0,2.75) {};

\node at (0:2.5) {$0^c, 2\pi^c$};
\node at (30:2.25) {$\pisix^c$};
\node at (45:2.25) {$\pifour^c$};
\node at (60:2.25) {$\pithree^c$};

\node at (90:2.25) {$\pitwo^c$};
\node at (120:2.25) {$\twopithree^c$};
\node at (135:2.25) {$\threepifour^c$};
\node at (150:2.25) {$\fivepisix^c$};

\node at (180:2.25) {$\pi^c$};
\node at (210:2.25) {$\sevenpisix^c$};
\node at (225:2.25) {$\fivepifour^c$};
\node at (240:2.25) {$\fourpithree^c$};

\node at (270:2.25) {$\threepitwo^c$};
\node at (300:2.25) {$\fivepithree^c$};
\node at (315:2.25) {$\sevenpifour^c$};
\node at (330:2.25) {$\elevenpisix^c$};

\LLANGLE{0,0}{1.75}{0}{30}{}{0.5}
\LLANGLE{0,0}{1.75}{30}{45}{}{0.5}
\LLANGLE{0,0}{1.75}{45}{60}{}{0.5}
\LLANGLE{0,0}{1.75}{60}{90}{}{0.5}

\LLANGLE{0,0}{1.75}{90}{120}{}{0.5}
\LLANGLE{0,0}{1.75}{120}{135}{}{0.5}
\LLANGLE{0,0}{1.75}{135}{150}{}{0.5}
\LLANGLE{0,0}{1.75}{150}{180}{}{0.5}

\LLANGLE{0,0}{1.75}{180}{210}{}{0.5}
\LLANGLE{0,0}{1.75}{210}{225}{}{0.5}
\LLANGLE{0,0}{1.75}{225}{240}{}{0.5}
\LLANGLE{0,0}{1.75}{240}{270}{}{0.5}

\LLANGLE{0,0}{1.75}{270}{300}{}{0.5}
\LLANGLE{0,0}{1.75}{300}{315}{}{0.5}
\LLANGLE{0,0}{1.75}{315}{330}{}{0.5}
\LLANGLE{0,0}{1.75}{330}{360}{}{.5}

\end{tikzpicture}
\end{figure}

\begin{figure}[htb]
\center
\caption{Sine of simple angles.}
\label{fig:review of simple angles}
\begin{tikzpicture}[inner sep=0pt,minimum size=0mm]

\node at (0,2.5) {};

\node at (0:2.25) {$0$};
\node at (30:2.25) {$\frac{1}{2}$};
\node at (45:2.25) {$\frac{\sqrt{2}}{2}$};
\node at (60:2.25) {$\frac{\sqrt{3}}{2}$};

\node at (90:2.25) {$1$};
\node at (120:2.25) {$\frac{\sqrt{3}}{2}$};
\node at (135:2.25) {$\frac{\sqrt{2}}{2}$};
\node at (150:2.25) {$\frac{1}{2}$};

\node at (180:2.25) {$0$};
\node at (210:2.25) {$-\frac{1}{2}$};
\node at (225:2.25) {$-\frac{\sqrt{2}}{2}$};
\node at (240:2.25) {$-\frac{\sqrt{3}}{2}$};

\node at (270:2.25) {$-1$};
\node at (300:2.25) {$-\frac{\sqrt{3}}{2}$};
\node at (315:2.25) {$-\frac{\sqrt{2}}{2}$};
\node at (330:2.25) {$-\frac{1}{2}$};

\LLANGLE{0,0}{1.75}{0}{30}{}{0.5}
\LLANGLE{0,0}{1.75}{30}{45}{}{0.5}
\LLANGLE{0,0}{1.75}{45}{60}{}{0.5}
\LLANGLE{0,0}{1.75}{60}{90}{}{0.5}

\LLANGLE{0,0}{1.75}{90}{120}{}{0.5}
\LLANGLE{0,0}{1.75}{120}{135}{}{0.5}
\LLANGLE{0,0}{1.75}{135}{150}{}{0.5}
\LLANGLE{0,0}{1.75}{150}{180}{}{0.5}

\LLANGLE{0,0}{1.75}{180}{210}{}{0.5}
\LLANGLE{0,0}{1.75}{210}{225}{}{0.5}
\LLANGLE{0,0}{1.75}{225}{240}{}{0.5}
\LLANGLE{0,0}{1.75}{240}{270}{}{0.5}

\LLANGLE{0,0}{1.75}{270}{300}{}{0.5}
\LLANGLE{0,0}{1.75}{300}{315}{}{0.5}
\LLANGLE{0,0}{1.75}{315}{330}{}{0.5}
\LLANGLE{0,0}{1.75}{330}{360}{}{.5}

\end{tikzpicture}
\end{figure}

\begin{figure}[htb]
\center
\caption{Cosine of simple angles.}
\label{fig:review of simple angles}
\begin{tikzpicture}[inner sep=0pt,minimum size=0mm]

\node at (0,2.5) {};

\node at (0:2.25) {$1$};
\node at (30:2.25) {$\frac{\sqrt{3}}{2}$};
\node at (45:2.25) {$\frac{\sqrt{2}}{2}$};
\node at (60:2.25) {$\frac{1}{2}$};

\node at (90:2.25){$0$};
\node at (120:2.25) {$-\frac{1}{2}$};
\node at (135:2.25) {$-\frac{\sqrt{2}}{2}$};
\node at (150:2.25) {$-\frac{\sqrt{3}}{2}$};

\node at (180:2.25) {$-1$};
\node at (210:2.25) {$-\frac{\sqrt{3}}{2}$};
\node at (225:2.25) {$-\frac{\sqrt{2}}{2}$};
\node at (240:2.25) {$-\frac{1}{2}$};

\node at (270:2.25) {$0$};
\node at (300:2.25) {$\frac{1}{2}$};
\node at (315:2.25) {$\frac{\sqrt{2}}{2}$};
\node at (330:2.25) {$\frac{\sqrt{3}}{2}$};

\LLANGLE{0,0}{1.75}{0}{30}{}{0.5}
\LLANGLE{0,0}{1.75}{30}{45}{}{0.5}
\LLANGLE{0,0}{1.75}{45}{60}{}{0.5}
\LLANGLE{0,0}{1.75}{60}{90}{}{0.5}

\LLANGLE{0,0}{1.75}{90}{120}{}{0.5}
\LLANGLE{0,0}{1.75}{120}{135}{}{0.5}
\LLANGLE{0,0}{1.75}{135}{150}{}{0.5}
\LLANGLE{0,0}{1.75}{150}{180}{}{0.5}

\LLANGLE{0,0}{1.75}{180}{210}{}{0.5}
\LLANGLE{0,0}{1.75}{210}{225}{}{0.5}
\LLANGLE{0,0}{1.75}{225}{240}{}{0.5}
\LLANGLE{0,0}{1.75}{240}{270}{}{0.5}

\LLANGLE{0,0}{1.75}{270}{300}{}{0.5}
\LLANGLE{0,0}{1.75}{300}{315}{}{0.5}
\LLANGLE{0,0}{1.75}{315}{330}{}{0.5}
\LLANGLE{0,0}{1.75}{330}{360}{}{.5}

\end{tikzpicture}
\end{figure}

\begin{figure}[htb]
\center
\caption{Tangent of simple angles.}
\label{fig:review of simple angles}
\begin{tikzpicture}[inner sep=0pt,minimum size=0mm]

\node at (0,2.5) {};

\node at (0:2.25) {$0$};
\node at (30:2.25) {$\frac{1}{\sqrt{3}}$};
\node at (45:2.25) {$1$};
\node at (60:2.25) {$\sqrt{3}$};

\node at (90:2.25) {$nan$};
\node at (120:2.25) {$-\sqrt{3}$};
\node at (135:2.25) {$-1$};
\node at (150:2.25) {$-\frac{1}{\sqrt{3}}$};

\node at (180:2.25) {$0$};
\node at (210:2.25) {$\frac{1}{\sqrt{3}}$};
\node at (225:2.25) {$1$};
\node at (240:2.25) {$\sqrt{3}$};

\node at (270:2.25) {$nan$};
\node at (300:2.25) {$-\sqrt{3}$};
\node at (315:2.25) {$-1$};
\node at (330:2.25) {$-\frac{1}{\sqrt{3}}$};

\LLANGLE{0,0}{1.75}{0}{30}{}{0.5}
\LLANGLE{0,0}{1.75}{30}{45}{}{0.5}
\LLANGLE{0,0}{1.75}{45}{60}{}{0.5}
\LLANGLE{0,0}{1.75}{60}{90}{}{0.5}

\LLANGLE{0,0}{1.75}{90}{120}{}{0.5}
\LLANGLE{0,0}{1.75}{120}{135}{}{0.5}
\LLANGLE{0,0}{1.75}{135}{150}{}{0.5}
\LLANGLE{0,0}{1.75}{150}{180}{}{0.5}

\LLANGLE{0,0}{1.75}{180}{210}{}{0.5}
\LLANGLE{0,0}{1.75}{210}{225}{}{0.5}
\LLANGLE{0,0}{1.75}{225}{240}{}{0.5}
\LLANGLE{0,0}{1.75}{240}{270}{}{0.5}

\LLANGLE{0,0}{1.75}{270}{300}{}{0.5}
\LLANGLE{0,0}{1.75}{300}{315}{}{0.5}
\LLANGLE{0,0}{1.75}{315}{330}{}{0.5}
\LLANGLE{0,0}{1.75}{330}{360}{}{.5}

\end{tikzpicture}
\end{figure}









\clearpage
\subsection{Chapter 5}

{\bf Draw and label graphs for the six trig. functions: sine, cosine, tangent, secant, cosecant, and cotangent.  You should be able to draw all of these from memory, without referring back to the text.}\\

\begin{figure}[htb]
\center
\caption{Graph of sine.}
\label{fig:graph of sine}
\begin{tikzpicture}[inner sep=0pt,minimum size=0mm, scale = 0.7]

\node at (3.1415, -2.5) {$\pi$};
\node at (3/2*3.1415, -2.5) {$\frac{3\pi}{2}$};
\node at (2*3.1415, -2.5) {$2\pi$};
\node at (1/2*3.1415, -2.5) {$\frac{\pi}{2}$};
\node at (0, -2.5) {$0$};
\node at (1/2*-3.1415, -2.5) {-$\frac{\pi}{2}$};
\node at (-3.1415, -2.5) {$-\pi$};
\node at (3/2*-3.1415, -2.5) {-$\frac{3\pi}{2}$};
\node at (2*-3.1415, -2.5) {$-2\pi$};

\node[left] at (-7,2) {$2$};
\node[left] at (-7,1) {$1$};
\node[left] at (-7,0) {$0$};
\node[left] at (-7,-1) {$-1$};
\node[left] at (-7,-2) {$-2$};

\draw[xstep=3.1415/2, densely dashed] (-2*3.1415,-2) grid (2*3.1415,2);
\AXES{0}{0}{2*3.1415}{2}
\draw[thick, variable = \t, domain=2*-3.1415:2*3.1415,samples=250] plot ({\t},{sin(180*\t/3.1415)});

\end{tikzpicture}
\end{figure}


\begin{figure}[htb]
\center
\caption{Graph of cosine.}
\label{fig:graph of cosine}
\begin{tikzpicture}[inner sep=0pt,minimum size=0mm, scale = 0.7]

\node at (3.1415, -2.5) {$\pi$};
\node at (3/2*3.1415, -2.5) {$\frac{3\pi}{2}$};
\node at (2*3.1415, -2.5) {$2\pi$};
\node at (1/2*3.1415, -2.5) {$\frac{\pi}{2}$};
\node at (0, -2.5) {$0$};
\node at (1/2*-3.1415, -2.5) {-$\frac{\pi}{2}$};
\node at (-3.1415, -2.5) {$-\pi$};
\node at (3/2*-3.1415, -2.5) {-$\frac{3\pi}{2}$};
\node at (2*-3.1415, -2.5) {$-2\pi$};

\node[left] at (-7,2) {$2$};
\node[left] at (-7,1) {$1$};
\node[left] at (-7,0) {$0$};
\node[left] at (-7,-1) {$-1$};
\node[left] at (-7,-2) {$-2$};

\draw[xstep=3.1415/2, densely dashed] (-2*3.1415,-2) grid (2*3.1415,2);
\AXES{0}{0}{2*3.1415}{2}
\draw[thick, variable = \t, domain=2*-3.1415:2*3.1415,samples=250] plot ({\t},{cos(180*\t/3.1415)});

\end{tikzpicture}
\end{figure}


\begin{figure}[htb]
\center
\caption{Graph of tangent.}
\label{fig:graph of tan}
\begin{tikzpicture}[inner sep=0pt,minimum size=0mm, scale = 0.7]

\node at (3.1415, -4.5) {$\pi$};
\node at (3/2*3.1415, -4.5) {$\frac{3\pi}{2}$};
\node at (2*3.1415, -4.5) {$2\pi$};
\node at (1/2*3.1415, -4.5) {$\frac{\pi}{2}$};
\node at (0, -4.5) {$0$};
\node at (1/2*-3.1415, -4.5) {-$\frac{\pi}{2}$};
\node at (-3.1415, -4.5) {$-\pi$};
\node at (3/2*-3.1415, -4.5) {-$\frac{3\pi}{2}$};
\node at (2*-3.1415, -4.5) {$-2\pi$};

\node[left] at (-7,4) {$4$};
\node[left] at (-7,3) {$3$};
\node[left] at (-7,2) {$2$};
\node[left] at (-7,1) {$1$};
\node[left] at (-7,0) {$0$};
\node[left] at (-7,-1) {$-1$};
\node[left] at (-7,-2) {$-2$};
\node[left] at (-7,-3) {$-3$};
\node[left] at (-7,-4) {$-4$};

\draw[xstep=3.1415/2, dashed] (-2*3.1415,-4) grid (2*3.1415,4);
\AXES{0}{0}{2*3.1415}{4}

\begin{scope}
    \clip(2*-3.1415,-4) rectangle (2*3.1415,4);

\draw[thick,
 variable = \t, 
 domain=4/2*-3.1415+0.01:3/2*-3.1415-0.01,
 samples=30] 
plot ({\t},{tan(180*\t/3.1415)});


\draw[thick,
 variable = \t, 
 domain=3/2*-3.1415+0.01:1/2*-3.1415-0.01,
 samples=30] 
plot ({\t},{tan(180*\t/3.1415)});

\draw[thick,
 variable = \t, 
 domain=1/2*-3.1415+0.01:1/2*3.1415-0.01,
 samples=30] 
plot ({\t},{tan(180*\t/3.1415)});

\draw[thick,
 variable = \t, 
 domain=1/2*3.1415+0.01:3/2*3.1415-0.01,
 samples=30] 
plot ({\t},{tan(180*\t/3.1415)});

\draw[thick,
 variable = \t, 
 domain=3/2*3.1415+0.01:4/2*3.1415-0.01,
 samples=30] 
plot ({\t},{tan(180*\t/3.1415)});

\end{scope}

\end{tikzpicture}
\end{figure}



\begin{figure}[htb]
\center
\caption{Graph of secant.}
\label{fig:graph of sec}
\begin{tikzpicture}[inner sep=0pt,minimum size=0mm, scale = 0.7]

\node at (3.1415, -4.5) {$\pi$};
\node at (3/2*3.1415, -4.5) {$\frac{3\pi}{2}$};
\node at (2*3.1415, -4.5) {$2\pi$};
\node at (1/2*3.1415, -4.5) {$\frac{\pi}{2}$};
\node at (0, -4.5) {$0$};
\node at (1/2*-3.1415, -4.5) {-$\frac{\pi}{2}$};
\node at (-3.1415, -4.5) {$-\pi$};
\node at (3/2*-3.1415, -4.5) {-$\frac{3\pi}{2}$};
\node at (2*-3.1415, -4.5) {$-2\pi$};

\node[left] at (-7,4) {$4$};
\node[left] at (-7,3) {$3$};
\node[left] at (-7,2) {$2$};
\node[left] at (-7,1) {$1$};
\node[left] at (-7,0) {$0$};
\node[left] at (-7,-1) {$-1$};
\node[left] at (-7,-2) {$-2$};
\node[left] at (-7,-3) {$-3$};
\node[left] at (-7,-4) {$-4$};

\draw[xstep=3.1415/2, dashed] (-2*3.1415,-4) grid (2*3.1415,4);
\AXES{0}{0}{2*3.1415}{4}

\begin{scope}
    \clip(2*-3.1415,-4) rectangle (2*3.1415,4);

\draw[thick,
 variable = \t, 
 domain=4/2*-3.1415+0.01:3/2*-3.1415-0.01,
 samples=30] 
plot ({\t},{1/cos(180*\t/3.1415)});

\draw[thick,
 variable = \t, 
 domain=3/2*-3.1415+0.01:1/2*-3.1415-0.01,
 samples=30] 
plot ({\t},{1/cos(180*\t/3.1415)});

\draw[thick,
 variable = \t, 
 domain=1/2*-3.1415+0.01:1/2*3.1415-0.01,
 samples=30] 
plot ({\t},{1/cos(180*\t/3.1415)});

\draw[thick,
 variable = \t, 
 domain=1/2*3.1415+0.01:3/2*3.1415-0.01,
 samples=30] 
plot ({\t},{1/cos(180*\t/3.1415)});

\draw[thick,
 variable = \t, 
 domain=3/2*3.1415+0.01:5/2*3.1415-0.01,
 samples=30] 
plot ({\t},{1/cos(180*\t/3.1415)});

\end{scope}

\end{tikzpicture}
\end{figure}


\begin{figure}[htb]
\center
\caption{Graph of cosecant.}
\label{fig:graph of csc}
\begin{tikzpicture}[inner sep=0pt,minimum size=0mm, scale = 0.7]

\node at (3.1415, -4.5) {$\pi$};
\node at (3/2*3.1415, -4.5) {$\frac{3\pi}{2}$};
\node at (2*3.1415, -4.5) {$2\pi$};
\node at (1/2*3.1415, -4.5) {$\frac{\pi}{2}$};
\node at (0, -4.5) {$0$};
\node at (1/2*-3.1415, -4.5) {-$\frac{\pi}{2}$};
\node at (-3.1415, -4.5) {$-\pi$};
\node at (3/2*-3.1415, -4.5) {-$\frac{3\pi}{2}$};
\node at (2*-3.1415, -4.5) {$-2\pi$};

\node[left] at (-7,4) {$4$};
\node[left] at (-7,3) {$3$};
\node[left] at (-7,2) {$2$};
\node[left] at (-7,1) {$1$};
\node[left] at (-7,0) {$0$};
\node[left] at (-7,-1) {$-1$};
\node[left] at (-7,-2) {$-2$};
\node[left] at (-7,-3) {$-3$};
\node[left] at (-7,-4) {$-4$};

\draw[xstep=3.1415/2, dashed] (-2*3.1415,-4) grid (2*3.1415,4);
\AXES{0}{0}{2*3.1415}{4}

\begin{scope}
    \clip(2*-3.1415,-4) rectangle (2*3.1415,4);

\draw[thick,
 variable = \t, 
 domain=4/2*-3.1415+0.01:2/2*-3.1415-0.01,
 samples=30] 
plot ({\t},{1/sin(180*\t/3.1415)});

\draw[thick,
 variable = \t, 
 domain=2/2*-3.1415+0.01:0/2*-3.1415-0.01,
 samples=30] 
plot ({\t},{1/sin(180*\t/3.1415)});

\draw[thick,
 variable = \t, 
 domain=0/2*-3.1415+0.01:2/2*3.1415-0.01,
 samples=30] 
plot ({\t},{1/sin(180*\t/3.1415)});

\draw[thick,
 variable = \t, 
 domain=2/2*3.1415+0.01:4/2*3.1415-0.01,
 samples=30] 
plot ({\t},{1/sin(180*\t/3.1415)});

\end{scope}

\end{tikzpicture}
\end{figure}


\begin{figure}[htb]
\center
\caption{Graph of cotangent.}
\label{fig:graph of cot}
\begin{tikzpicture}[inner sep=0pt,minimum size=0mm, scale = 0.7]

\node at (3.1415, -4.5) {$\pi$};
\node at (3/2*3.1415, -4.5) {$\frac{3\pi}{2}$};
\node at (2*3.1415, -4.5) {$2\pi$};
\node at (1/2*3.1415, -4.5) {$\frac{\pi}{2}$};
\node at (0, -4.5) {$0$};
\node at (1/2*-3.1415, -4.5) {-$\frac{\pi}{2}$};
\node at (-3.1415, -4.5) {$-\pi$};
\node at (3/2*-3.1415, -4.5) {-$\frac{3\pi}{2}$};
\node at (2*-3.1415, -4.5) {$-2\pi$};

\node[left] at (-7,4) {$4$};
\node[left] at (-7,3) {$3$};
\node[left] at (-7,2) {$2$};
\node[left] at (-7,1) {$1$};
\node[left] at (-7,0) {$0$};
\node[left] at (-7,-1) {$-1$};
\node[left] at (-7,-2) {$-2$};
\node[left] at (-7,-3) {$-3$};
\node[left] at (-7,-4) {$-4$};

\draw[xstep=3.1415/2, dashed] (-2*3.1415,-4) grid (2*3.1415,4);
\AXES{0}{0}{2*3.1415}{4}

\begin{scope}
    \clip(2*-3.1415,-4) rectangle (2*3.1415,4);

\draw[thick,
 variable = \t, 
 domain=4/2*-3.1415+0.01:2/2*-3.1415-0.01,
 samples=30] 
plot ({\t},{1/tan(180*\t/3.1415)});

\draw[thick,
 variable = \t, 
 domain=2/2*-3.1415+0.01:0/2*-3.1415-0.01,
 samples=30] 
plot ({\t},{1/tan(180*\t/3.1415)});

\draw[thick,
 variable = \t, 
 domain=0/2*-3.1415+0.01:2/2*3.1415-0.01,
 samples=30] 
plot ({\t},{1/tan(180*\t/3.1415)});

\draw[thick,
 variable = \t, 
 domain=2/2*3.1415+0.01:4/2*3.1415-0.01,
 samples=30] 
plot ({\t},{1/tan(180*\t/3.1415)});
\end{scope}

\end{tikzpicture}
\end{figure}









\clearpage
\subsection{Chapter 6}

\begin{enumerate}

\item{{\bf Find 3 equivalent angles for each of the following:}\\

\tab a) $30^o \simeq  ..., -690^o, -330^o, 390^o, 750^o, ...$\\

\tab b) $\frac{7\pi}{6}^c \simeq ..., \frac{-17\pi}{6}^c, \frac{-5\pi}{6}^c, \frac{19\pi}{6}^c, \frac{31\pi}{6}^c, ...$\\

\tab c) $115^o \simeq  ..., -605^o, -245^o, 475^o, 835^o, ...$\\

\tab d) $\frac{-\pi}{2}^c \simeq ..., \frac{-9\pi}{2}^c, \frac{-5\pi}{2}^c, \frac{3\pi}{2}^c, \frac{7\pi}{2}^c, ...$\\

\tab e) $0^o  \simeq  ..., -720^o, -360^o, 360^o, 720^o, ...$\\}

\item{{\bf Write whether each of the following trig. functions are even, odd, or neither, and prove it:}\\

\tab a) $tan(\theta)$\\

\tab\tab $tan(\theta) = \frac{sin(\theta)}{cos(\theta)}$\\

\tab\tab$\implies tan(-\theta) = \frac{sin(-\theta)}{cos(-\theta)}$\\

\tab\tab $sin(-\theta) = - sin(\theta)$ and $cos(-\theta)=cos(\theta)$\\

\tab\tab$\implies \frac{sin(-\theta)}{cos(-\theta)}  = \frac{-sin(\theta)}{cos(\theta)} = -tan(\theta)$\\

\tab\tab $\implies tan(-\theta) = - tan(\theta)$\\

\tab\tab Tangent is {\bf odd}.\\

\tab b) $cot(\theta)$\\

\tab\tab $cot(-\theta) = \frac{1}/{tan(-\theta)}$\\

\tab\tab $= - \frac{1}{tan(\theta)}$\\

\tab\tab $= -cot(\theta)$\\

\tab\tab Cotangent is {\bf even}.\\

\tab c) $sec(\theta)$\\

\tab\tab $sec(-\theta) = \frac{1}{cos(-\theta)}$\\

\tab\tab $= \frac{1}{cos(\theta)} = sec(\theta)$\\

\tab\tab Secant is {\bf even}.\\

\tab d) $csc(\theta)$\\

\tab\tab $csc(-theta) = \frac{1}{sin(-\theta)}$\\

\tab\tab $= \frac{1}{-sin(\theta)} = - csc(theta)$\\

\tab\tab Cosecant is {\bf odd}.\\

\tab e) $sin(\theta) - cos(\theta)$\\}

\tab\tab $sin(-\theta) - cos(-\theta) = -sin(\theta) - cos(\theta)$\\

\tab\tab $-sin(\theta) - cos(\theta) \neq sin(\theta) - cos(\theta)$\\

\tab\tab $sin(\theta) - cos(\theta)$ is {\bf neither}.\\	

\item{Simplify the following expressions:\\

\tab a) $cos^2(\theta)tan(\theta)$\\

\tab\tab $cos^2(\theta)tan(\theta) = cos^2(\theta)\frac{sin(\theta)}{cos(\theta)}$\\

\tab\tab $= sin(\theta)cos(\theta)$\\

\tab b) $csc(\theta) - cos(\theta)cot(\theta)$\\

\tab\tab $csc(\theta) - cos(\theta)cot(\theta)$\\

\tab\tab $= \frac{1}{sin(\theta)} - cos(\theta)\frac{cos(\theta)}{sin(\theta)}$\\

\tab\tab $= \frac{1}{sin(\theta)} - \frac{cos^2(\theta)}{sin(\theta)}$\\

\tab\tab $= \frac{1 - cos^2(\theta)}{sin(\theta)}$\\

\tab\tab $= \frac{sin^2(\theta)}{sin(\theta)}$\\

\tab\tab $= sin(\theta)$\\

\tab c) $1 + cot^2(\theta)$\\

\tab\tab $1 + cot^2(\theta)$\\

\tab\tab $= 1 + \frac{cos^2(\theta)}{sin^2(\theta)}$\\

\tab\tab $= \frac{sin^2(\theta)}{sin^2(\theta)} + \frac{cos^2(\theta)}{sin^2(\theta)}$\\

\tab\tab $= \frac{sin^2(\theta) + cos^2(\theta)}{sin^2(\theta)}$\\

\tab\tab $= \frac{1}{sin^2(\theta)}$\\

\tab\tab $= csc^2(\theta)$\\

\tab d) $\frac{sin^2(\theta)tan^2(\theta) + sin^2(\theta)}{tan^2(\theta)}$\\

\tab\tab $\frac{sin^2(\theta)tan^2(\theta) + sin^2(\theta)}{tan^2(\theta)}$\\

\tab\tab $= \frac{sin^2(\theta)tan^2(\theta) + sin^2(\theta)}{tan^2(\theta)}$\\

\tab\tab $= \frac{sin^2(\theta)tan^2(\theta)}{tan^2(\theta)} + \frac{sin^2(\theta)}{tan^2(\theta)}$\\

\tab\tab $= sin^2(\theta) + sin^2(\theta)cot^2(\theta)$\\

\tab\tab $= sin^2(\theta) + sin^2(\theta)\frac{cos^2(\theta)}{sin^2(\theta)}$\\

\tab\tab $= sin^2(\theta) + cos^2(\theta)$\\

\tab\tab $= 1$\\

\tab e) $\frac{(sec(\theta) + tan(\theta))(sec(\theta) - tan(\theta))}{(csc(\theta) - cot(\theta))(csc(\theta) + cot(\theta))}$\\}

\tab\tab $\frac{(sec(\theta) + tan(\theta))(sec(\theta) - tan(\theta))}{(csc(\theta) - cot(\theta))(csc(\theta) + cot(\theta))}$\\

\tab\tab $= \frac{sec^2(\theta) - tan^2(\theta)}{csc^2(\theta) - cot^2(\theta)}$\\

\tab\tab $= \frac{1}{1}$\\

\tab\tab $= 1$\\

\item{\bf Given that the sine of an angle is 0.73, what is the cosine?  the tangent?\\}

\tab $sin(\theta) = 0.73$ and $sin^2(\theta) + cos^2(\theta) = 1$\\

\tab $\implies 0.73^2 + cos^2(\theta) = 1$\\

\tab $\implies 0.5329 + cos^2(\theta) = 1$\\

\tab $\implies cos^2(\theta) = 0.4671$\\

\tab $\implies cos(\theta) = \pm 0.6834$\\

\tab $tan(\theta) = \frac{sin(\theta)}{cos(\theta)}$\\

\tab $= \frac{0.73}{\pm 0.6834}$\\

\tab $= \pm 1.0681$\\

\item{Given that the tangent of an angle is 0.5 and the angle is in the first quadrant, what is the cosine?}

\tab $tan(\theta) = 0.5$\\

\tab $\implies \frac{sin(\theta)}{cos(\theta)} = 0.5$\\

\tab $\implies sin(\theta) = 0.5cos(\theta)$\\

\tab $sin^2(\theta) + cos^2(\theta) = 1$\\

\tab $\implies (0.5cos(\theta))^2 + cos^2(\theta) = 1$\\

\tab $\implies 0.25cos^2(\theta) + cos^2(\theta) = 1$\\

\tab $\implies 1.25cos^2(\theta) = 1$\\

\tab $\implies cos^2(\theta) = \frac{1}{1.25} = \frac{4}{5}$\\

\tab $\implies cos(\theta) = \frac{2}{\sqrt{5}}$\\

\end{enumerate}












\clearpage
\subsection{Chapter 7}

\begin{enumerate}

\item{{\bf What is the definition of the inverse of a function?}\\
{The inverse of a function $f(x)$ is the function $g(x)$, where $g(f(x))=x$}\\}

\item{\bf Write the inverse of each of the following functions:}\\

\tab a) $f(x) = x + 1$\\

\tab \tab $f(x) = x + 1$\\

\tab \tab $\implies x = f^{-1}(x) + 1$\\

\tab \tab $\implies f^{-1}(x) = x - 1$\\

\tab b) $f(x) = x^2$\\

\tab \tab $f(x) = x^2$\\

\tab \tab $\implies x = (f^{-1}(x))^2$\\

\tab \tab $\implies f^{-1}(x) = \sqrt{x}$\\

\tab c) $f(x) = \frac{1}{x}$\\

\tab \tab $f(x) = \frac{1}{x}$\\

\tab \tab $\implies x = \frac{1}{f^{-1}(x)}$\\

\tab \tab $\implies f^{-1}(x) = \frac{1}{x}$\\

\tab \tab Note:  The inverse of $\frac{1}{x}$ is itself.  Any function that has reflective symmetry across the diagonal $y = x$ will be its own inverse.\\

\tab d) $f(x) = sin(x)$\\

\tab \tab $f^{-1}(x) = arcsin(x)$\\

\tab e) $f(x) = 1$\\

\tab \tab $f(x) = 1$\\

\tab \tab $\implies x = ?$\\

\tab \tab This function has no inverse.  The inverse of this function is a relation, not a function.\\

\item{{\bf What is the difference between a relation and a function?}\\
{A function is a mapping where each input value corresponds to exactly one output value.  A relation is also a mapping of inputs to outputs, but each input may map to multiple outputs.}\\}


\item{{\bf Why do inverse trig. functions have a principal-value range?}\\
{Trig functions are periodic, and as a result, the inverse of any trig function is not a function, but a relation.  So what do we do?  We cheat, and constrain each inverse trig. relation to a range of half of its period.  When we do this, there are no repeating values and we can treat them as a functions.}\\}

\item{\bf Draw and label a graph of the arc-sine, arc-cosine, and arc-tangent, without referring back to the text.}\\

\begin{figure}[htb!]
\center
\caption{Arc-sine constrained to its principal-value range.}
\label{fig:principal value arc sine}
\begin{tikzpicture}[inner sep=0pt,minimum size=0mm, scale = 0.7]

\node at (-4.6,3.1415) {$\pi$};
\node at (-4.6,1/2*3.1415) {$\frac{\pi}{2}$};
\node at (-4.6,0) {$0$};
\node at (-4.6,1/2*-3.1415) {-$\frac{\pi}{2}$};
\node at (-4.6,-3.1415) {$-\pi$};

\node[left] at (4, -4) {$4$};
\node[left] at (3, -4) {$3$};
\node[left] at (2, -4) {$2$};
\node[left] at (1, -4) {$1$};
\node[left] at (0, -4) {$0$};
\node[left] at (-1, -4) {$-1$};
\node[left] at (-2, -4) {$-2$};
\node[left] at (-3, -4) {$-3$};
\node[left] at (-4, -4) {$-4$};

\draw[ystep=3.1415/2, densely dashed] (-4,-1*3.1415) grid (4,1*3.1415);
\AXES{0}{0}{4}{1*3.1415}
\draw[thick, variable = \t, domain=1/2*-3.1415:1/2*3.1415,samples=250] plot ({sin(180*\t/3.1415)},{\t});

\end{tikzpicture}
\end{figure}

\begin{figure}[htb!]
\center
\caption{Arc-cosine constrained to its principal-value range.}
\label{fig:principal value arc-cosine }
\begin{tikzpicture}[inner sep=0pt,minimum size=0mm, scale = 0.7]

\node at (-4.6,3.1415) {$\pi$};
\node at (-4.6,1/2*3.1415) {$\frac{\pi}{2}$};
\node at (-4.6,0) {$0$};
\node at (-4.6,1/2*-3.1415) {-$\frac{\pi}{2}$};
\node at (-4.6,-3.1415) {$-\pi$};

\node[left] at (4, -4) {$4$};
\node[left] at (3, -4) {$3$};
\node[left] at (2, -4) {$2$};
\node[left] at (1, -4) {$1$};
\node[left] at (0, -4) {$0$};
\node[left] at (-1, -4) {$-1$};
\node[left] at (-2, -4) {$-2$};
\node[left] at (-3, -4) {$-3$};
\node[left] at (-4, -4) {$-4$};

\draw[ystep=3.1415/2, densely dashed] (-4,-1*3.1415) grid (4,1*3.1415);
\AXES{0}{0}{4}{1*3.1415}
\draw[thick, variable = \t, domain=0:3.1415,samples=250] plot ({cos(180*\t/3.1415)},{\t});

\end{tikzpicture}
\end{figure}


\begin{figure}[htb!]
\center
\caption{Arc-tangent contrained to its principal-value range.}
\label{fig:principal value arc-tan}
\begin{tikzpicture}[inner sep=0pt,minimum size=0mm, scale = 0.7]

\node at (-4.6, 3.1415) {$\pi$};
\node at (-4.6, 1/2*3.1415) {$\frac{\pi}{2}$};
\node at (-4.6, 0) {$0$};
\node at (-4.6, 1/2*-3.1415) {-$\frac{\pi}{2}$};
\node at (-4.6, -3.1415) {$-\pi$};

\node[left] at (4, -4) {$4$};
\node[left] at (3, -4) {$3$};
\node[left] at (2, -4) {$2$};
\node[left] at (1, -4) {$1$};
\node[left] at (0, -4) {$0$};
\node[left] at (-1, -4) {$-1$};
\node[left] at (-2, -4) {$-2$};
\node[left] at (-3, -4) {$-3$};
\node[left] at (-4, -4) {$-4$};

\draw[ystep=3.1415/2, dashed] (-4,-3.1415) grid (4,3.1415);
\AXES{0}{0}{4}{3.1415}

\begin{scope}
    \clip(-4, -3.1415) rectangle (4, 3.1415);

\draw[thick,
 variable = \t, 
 domain=1/2*-3.1415+0.01:1/2*3.1415-0.01,
 samples=30] 
plot ({tan(180*\t/3.1415)}, {\t});

\end{scope}

\end{tikzpicture}
\end{figure}



\end{enumerate}










\clearpage
\subsection{Chapter 8}

\begin{enumerate}

\item{\bf Write down the formulae for the sine, cosine, and tangent of the sum of two angles until you've committed them to memory.}\\

\tab$sin(\alpha \pm \beta) = sin(\alpha)cos(\beta) \pm cos(\alpha)sin(\beta)$\\

\tab$cos(\alpha \pm \beta) = cos(\alpha)cos(\beta) \mp sin(\alpha)sin(\beta)$\\

\tab$tan(\alpha \pm \beta) = \frac{tan(\alpha) \pm tan(\beta)}{1 \mp tan(\alpha)tan(\beta)}$\\



\item{\bf Using the previous formulae, derive the double angle formulae for sine, cosine, and tangent}\\

\tab$sin(\alpha \pm \beta) = sin(\alpha)cos(\beta) \pm cos(\alpha)sin(\beta)$\\

\tab$\implies sin(\alpha + \alpha) = sin(\alpha)cos(\alpha) + cos(\alpha)sin(\alpha)$\\

\tab$\implies sin(2\alpha) = 2sin(\alpha)cos(\alpha)$\\ \\


\tab$cos(\alpha \pm \beta) = cos(\alpha)cos(\beta) \mp sin(\alpha)sin(\beta)$\\

\tab$\implies cos(\alpha + \alpha) = cos(\alpha)cos(\alpha) - sin(\alpha)sin(\alpha)$\\

\tab$\implies cos(2\alpha) = cos^2(\alpha) - sin^2(\alpha)$\\ \\


\tab$tan(\alpha \pm \beta) = \frac{tan(\alpha) \pm tan(\beta)}{1 \mp tan(\alpha)tan(\beta)}$\\

\tab$\implies tan(\alpha + \alpha) = \frac{tan(\alpha) + tan(\alpha)}{1 - tan(\alpha)tan(\alpha)}$\\

\tab$\implies tan(2\alpha) = \frac{2tan(\alpha)}{1 - tan^2(\alpha)}$\\

\item{\bf Using the previous formulae, derive the half angle formulae for sine, cosine, and tangent}\\

Using the double angle formulae for the cosine, and the Pythagorean identitiy, we can also derive equations for the half-angle of the sine and cosine.\\

\tab$cos(2\alpha) = cos^2(\alpha) - sin^2(\alpha)$\\

\tab$\implies cos(\alpha) = cos^2(\frac{\alpha}{2}) - sin^2(\frac{\alpha}{2})$\\

\tab$\implies cos(\alpha) = (1 - sin^2(\frac{\alpha}{2})) - sin^2(\frac{\alpha}{2})$\\

\tab$\implies cos(\alpha) = 1- 2sin^2(\frac{\alpha}{2})$\\

\tab$\implies \frac{1 - cos(\alpha)}{2} = sin^2(\frac{\alpha}{2})$\\

\tab$\implies sin(\frac{\alpha}{2}) = (\frac{1 - cos(\alpha)}{2})^{\frac{1}{2}}$\\ \\



\tab$cos(2\alpha) = cos^2(\alpha) - sin^2(\alpha)$\\

\tab$\implies cos(\alpha) = cos^2(\frac{\alpha}{2}) - sin^2(\frac{\alpha}{2})$\\

\tab$\implies cos(\alpha) = cos^2(\frac{\alpha}{2}) - (1 - cos^2(\frac{\alpha}{2}))$\\

\tab$\implies cos(\alpha) = 2cos^2(\frac{\alpha}{2}) -  1$\\

\tab$\implies \frac{1 + cos(\alpha)}{2} = cos^2(\frac{\alpha}{2})$\\

\tab$\implies cos(\frac{\alpha}{2}) = (\frac{1 + cos(\alpha)}{2})^{\frac{1}{2}}$\\ \\


\tab$tan(\frac{\alpha}{2}) = \frac{sin(\frac{\alpha}{2})}{cos(\frac{\alpha}{2})}$\\

\tab$ = \frac{ (\frac{1 - cos(\alpha)}{2})^{\frac{1}{2}}}{ (\frac{1 + cos(\alpha)}{2})^{\frac{1}{2}}} = (\frac{\frac{1 - cos(\alpha)}{2}}{\frac{1 + cos(\alpha)}{2}})^{\frac{1}{2}}$\\

\tab$ = (\frac{1 - cos(\alpha)}{1 + cos(\alpha)})^{\frac{1}{2}}$\\

\end{enumerate}





\clearpage
\subsection{Chapter 9}

\begin{enumerate}

\item{\bf Write the conversions from rectangular to polar coordinates, and the conversions from polar to rectangular coordinates, until you've committed them to memory.}\\

\tab Polar to rectangular:\\

\tab$y = rsin(\theta)$ \ \ \ and \ \ \ $x = rcos(\theta)$\\ \\

\tab Rectangular to polar:\\

\tab$r = \sqrt{x^2 + y^2}$\\

\tab$\theta = arcsin(\frac{y}{r})$ \ \ or \ \ $\theta = arccos(\frac{x}{r})$ \ \ or \ \ $\theta = arctan(\frac{y}{x})$\\

\item{\bf Convert the following coordinates from polar to rectangular:}\\

\tab a) $(1,0^c)$\\

\tab \tab $x = r cos(\theta) = 1 cos(0) = 1$\\

\tab \tab $y = r sin(\theta) = 1 sin(0) = 0$\\

\tab b) $(2,\frac{\pi}{4}^c)$\\

\tab \tab $x = 2 cos(\frac{\pi}{4}) = 2 \frac{\sqrt{2}}{2} = \sqrt{2}$\\

\tab \tab $y = 2 sin(\frac{\pi}{4}) = 2 \frac{\sqrt{2}}{2} = \sqrt{2}$\\

\tab c) $(1,115^o)$\\

\tab \tab $x = 1 cos(115^o) = -0.4226$\\

\tab \tab $y = 1 sin(115^o) = 0.9063$\\

\tab d) $(2,-30^o)$\\

\tab \tab $x = 2 cos(-30^o) = 2 \frac{\sqrt{3}}{2} = \sqrt{3}$\\

\tab \tab $y = 2 sin(-30^0) = 2 \frac{-1}{2} = -1$\\

\tab e) $(0,0^o)$\\

\tab \tab $x = 0 cos(0) = 0$\\

\tab \tab $y = 0 sin(0) = 0$\\


\item{\bf Convert the following coordinates from rectangular to polar:}\\

\tab a) $(1,0)$\\

\tab \tab $r = \sqrt{x^2 + y^2} = \sqrt{1^2 + 0^2} = 1$\\

\tab \tab $\theta = arctan(\frac{y}{x}) = arctan(\frac{0}{1}) = arctan(0) = 0^o$\\

\tab b) $(1,1)$\\

\tab \tab $r = \sqrt{1^2 + 1^2} = \sqrt{2}$\\

\tab \tab $\theta = arctan(\frac{1}{1}) = arctan(1) = 45^o$\\

\tab c) $(0,1)$\\

\tab \tab $r = \sqrt{0^2 + 1^2} = 1$\\

\tab \tab $\theta = arctan(\frac{1}{0}) = ? \implies \theta = 90^o$\\

\tab d) $(2,3)$\\

\tab \tab $r = \sqrt{2^2 + 3^2} = \sqrt{13}$\\

\tab \tab $\theta = arctan(\frac{3}{2})  = 35.78^o$\\

\tab e) $(0,0)$\\

\tab \tab $r = \sqrt{0^2 + 0^2} = 0$\\

\tab \tab $\theta = arctan(\frac{0}{0}) = ?$\\

\tab \tab The origin is an unusual point in polar coordinates.  Any point in polar coordinates with a radius of $0$ converts to $(0,0)$ in rectangular coordinates, which means the rectangular coordinates $(0,0)$ map to an infinite set of polar coordinates with radius 0.  In other words, the angle could be anything.\\

\end{enumerate}

\clearpage
\subsection{Chapter 10}

{\begin{figure}[htb]
\center
\caption{A triangle.}
\label{fig:A triangle}
\begin{tikzpicture}[inner sep=0pt,minimum size=0mm]

\node () at (0,2.5) {};

\draw[] (0,0) -- (2,0);
\draw[] (0,0) -- (1,1.866);
\draw[] (2,0) -- (1,1.866);

\node () at (0.25,0.125){$a$};
\node () at (1.0,1.5) {$b$};
\node () at (1.75,0.125) {$c$};

\node () at (1.75,0.9){$A$};
\node () at (0.25,0.9) {$C$};
\node () at (1.0,-0.25) {$B$};

\end{tikzpicture}
\end{figure}
}

\begin{enumerate}

\item{\bf Given $a=b=c=60\degree$ and $A=1$, find $B$ and $C$.}\\

\tab \tab Using the law of sines:\\

\tab \tab $\frac{sin(a)}{A} = \frac{sin(b)}{B}$\\

\tab \tab $=\frac{sin(60\degree)}{1} = \frac{sin(60\degree)}{B}$\\

\tab \tab $\implies B = 1$\\

\tab \tab $\frac{sin(a)}{A} = \frac{sin(c)}{C}$\\

\tab \tab $=\frac{sin(60\degree)}{1} = \frac{sin(60\degree)}{C}$\\

\tab \tab $\implies C = 1$\\

\item{\bf Given $a=35\degree$, $b=55\degree$, and $C=2$, find $c$, $A$, and $B$.}\\

\tab \tab $a+b+c=180\degree$\\

\tab \tab $35\degree + 55\degree + c = 180\degree$\\

\tab \tab $\implies c = 90\degree$\\

\tab \tab Using the law of sines,\\

\tab \tab $\frac{sin(c)}{C} = \frac{sin(a)}{A}$\\

\tab \tab $\frac{sin(90\degree)}{2} = \frac{sin(35)}{A}$\\

\tab \tab $\frac{1}{2} = \frac{0.574}{A}$\\

\tab \tab $A = 1.147$\\

\tab \tab $\frac{sin(c)}{C} = \frac{sin(b)}{B}$\\

\tab \tab $\frac{sin(90\degree)}{2} = \frac{sin(55)}{B}$\\

\tab \tab $\frac{1}{2} = \frac{0.819}{B}$\\

\tab \tab $B =1.638$\\

\item{\bf Given $a=45\degree$, $B=2$, and $C=3$, find $A$, $b$, and $c$.}\\

\tab \tab Using the law of cosines,\\

\tab \tab $A^2 = B^2 + C^2 - 2BCcos(a)$\\

\tab \tab $A^2 = 2^2 + 3^2 -2(2)(3)cos(45\degree)$\\

\tab \tab $A^2 = 4 + 9 -12\frac{\sqrt(2)}{2}$\\

\tab \tab $A^2 = 13 - 6\sqrt{2}$\\

\tab \tab $A^2 = 13 - 6\sqrt{2}$\\

\tab \tab $A = \sqrt{13 - 6\sqrt{2}}$\\

\tab \tab $A = \sqrt{13 - 6\sqrt{2}}$\\

\tab \tab $A = 2.125$\\

\tab \tab Using the law of sines,\\

\tab \tab $\frac{sin(a)}{A} = \frac{sin(b)}{B}$\\

\tab \tab $\frac{sin(45\degree)}{2.125} = \frac{sin(b)}{2}$\\

\tab \tab $\frac{\frac{\sqrt{2}}{2}}{2.125} = \frac{sin(b)}{2}$\\

\tab \tab $sin(b) = \frac{\sqrt{2}}{2.125}$\\

\tab \tab $sin(b) = 0.665$\\

\tab \tab $b = asin(0.665) = 41.72\degree$\\

\tab \tab $\frac{sin(a)}{A} = \frac{sin(c)}{C}$\\

\tab \tab $\frac{sin(45\degree)}{2.125} = \frac{sin(c)}{3}$\\

\tab \tab $\frac{\frac{\sqrt{2}}{2}}{2.125} = \frac{sin(c)}{3}$\\

\tab \tab $sin(c) = 0.998$\\

\tab \tab $b = asin(0.998) = 93.23\degree$ Note: $sin(x) = sin(180\degree - x)$\\

\item{\bf Given $A=5$, $B=7$, $C=11$, find $a$, $b$, and $c$.}\\

\tab \tab Using the law of cosines,\\

\tab \tab $A^2 = B^2 + C^2 - 2BCcos(a)$\\

\tab \tab $25 = 49 + 121 - 154cos(a)$\\

\tab \tab $cos(a) = \frac{49 + 121 -25}{154}$\\

\tab \tab $cos(a) = \frac{145}{154}$\\

\tab \tab $a = acos(0.9416)$\\

\tab \tab $a = 19.7\degree$\\

\tab \tab $B^2 = A^2 + C^2 - 2ACcos(b)$\\

\end{enumerate}

\clearpage
\subsection{Chapter 11}

\begin{enumerate}

\item{\bf Convert these complex numbers to polar form: \\ $1,\  i ,\  -1,\  -i,\  1+i,\  i-1,\  -i-1,\  1-i,\  2+i,\  2+2i$}\\

\tab \tab $1 = 1\angle0\degree$\\

\tab \tab $i = 1\angle90\degree$\\

\tab \tab $-1 = 1\angle180\degree$\\

\tab \tab $-i = 1\angle270\degree$\\

\tab \tab $1+i = \frac{\sqrt{2}}{2}\angle45\degree$\\

\tab \tab $i-1 = \frac{\sqrt{2}}{2}\angle135\degree$\\

\tab \tab $-1-i = \frac{\sqrt{2}}{2}\angle225\degree$\\

\tab \tab $1-i = \frac{\sqrt{2}}{2}\angle315\degree$\\

\tab \tab $2+i = \sqrt{5}\angle26.57\degree$\\

\tab \tab $2+i = \sqrt{2}\angle45\degree$\\

\item{\bf Convert these complex numbers to rectangular form: \\ $1\angle0,\   1\angle90\degree,\  2\angle\frac{\pi}{6}\rad,\   3\angle-60\degree,\  1\angle\pi\rad,\  -2\angle\frac{\pi}{4},\  0\angle90\degree,\  0\angle20\degree$}\\

\tab \tab $1\angle0 = 1$\\

\tab \tab $1\angle90 = i$\\

\tab \tab $2\angle\frac{\pi}{6}\rad = \sqrt{3} + i$\\

\tab \tab $3\angle-60\degree = \frac{3/2} - i\frac{3\sqrt{3}}{2}$\\

\tab \tab $1\angle\pi\rad = -1$\\

\tab \tab $-2\angle\frac{\pi}{4} = -\sqrt{2} - i\sqrt{2}$\\

\tab \tab $0\angle90\degree = 0$\\

\tab \tab $0\angle20\degree = 0$\\

\item{\bf Simplify the following: \\ $(1+i)+(2+2i),\ (3+3i)-(1-i),\ (1\angle\frac{3\pi}{4}\rad)-i,\  (1\angle45\degree)+(1\angle-45\degree), $}\\

\tab \tab $(1+i)+(2+2i) = 3 + 3i$\\

\tab \tab $(3+3i)-(1-i) = 2 + 4i$\\

\tab \tab $(1\angle\frac{3\pi}{4}\rad)-i = -\frac{\sqrt{2}}{2} + \frac{\sqrt{2}-2}{2}i$\\

\tab \tab $(1\angle45\degree)+(1\angle-45\degree) = \sqrt{2}$\\

\item{\bf Simplify the following: \\ $(1+i)(1-i),\ (2+i)(3+3i),\ (1+i)(1\angle\frac{\pi}{4}\rad),\ (2\angle135\degree)(2\angle210\degree)$}\\

\tab \tab $(1+i)(1-i) = 2$\\

\tab \tab $(2+i)(3+3i) = 3 + 9i$\\

\tab \tab $(1+i)(1\angle\frac{\pi}{4}\rad) = \frac{2+\sqrt{2}}{2} + \frac{2+\sqrt{2}}{2}i$\\

\tab \tab $(2\angle135\degree)(2\angle210\degree) = 4\angle345\degree$\\

\item{\bf Simplify the following: \\ $\frac{1+i}{1-i},\ \frac{2+2i}{-3-i},\ \frac{1\angle\frac{\pi}{3}\rad}{1+i},\ \frac{1\angle30\degree}{1\angle-60\degree}$}\\

\tab \tab $\frac{1+i}{1-i} = i$\\

\tab \tab $\frac{2+2i}{-3-i} = \frac{-4-2i}{5}$\\

\tab \tab $\frac{1\angle\frac{\pi}{3}\rad}{1+i} = \frac{(\sqrt{3} + 1) + (\sqrt{3} - 1)i}{4}$\\

\tab \tab $\frac{1\angle30\degree}{1\angle-60\degree} = i$\\

\end{enumerate}

\clearpage
\subsection{Chapter 12}

\begin{enumerate}

\item{\bf What is Euler's formula?}

\tab \tab $e^{i\theta} = cos(\theta) + i sin(\theta)$\\

\item{\bf Use Euler's formula to derive the double-angle sine formula: \\ $sin(2\theta)=?$}

Starting with Euler's formula:\\

\tab \tab $e^{i\theta} = cos(\theta) + i sin(\theta)$\\

Using rearrangements from chapter 12,\\

\tab \tab $sin(\theta) = \frac{1}{2i}(e^{i\theta} - e^{-i\theta})$ \  and \  $cos(\theta) = \frac{1}{2}(e^{i\theta} + e^{-i\theta})$\\

So,\\

\tab \tab $sin(2\theta) = \frac{1}{2i}(e^{i2\theta} - e^{-i2\theta})$\\

\tab \tab $\implies sin(2\theta) = \frac{1}{2i}((e^{i\theta})^{2} - (e^{-i\theta})^{2})$\\

\tab \tab $\implies sin(2\theta) = \frac{1}{2i} (e^{i\theta} - e^{-i\theta}) (e^{i\theta} + e^{-i\theta}) $\\

\tab \tab $\implies sin(2\theta) = (sin(\theta))(2cos(\theta))$\\

\tab \tab $\implies sin(2\theta) = 2sin(\theta)cos(\theta)$\\

\item{\bf Use Euler's formula to find the phase offset between sine and cosine: \\ $sin(\theta + x) = cos(\theta) \implies ?$}

Starting with Euler's formula:\\

\tab \tab $e^{i\theta} = cos(\theta) + i sin(\theta)$\\

We are looking for $x$ such that $sin(\theta + x) = cos(\theta)$\\

So, \\

\tab \tab $sin(\theta + x) = cos(\theta)$\\

\tab \tab $\implies \frac{1}{2i}(e^{i(\theta + x)} - e^{-i(\theta + x)})= \frac{1}{2}(e^{i\theta} + e^{-i\theta})$\\

\tab \tab $\implies (e^{i(\theta + x)} - e^{-i(\theta + x)})= i(e^{i\theta} + e^{-i\theta})$\\

\tab \tab $\implies (e^{i\theta}e^{ix} - e^{-i\theta}e^{-ix})= ie^{i\theta} + ie^{-i\theta}$\\

\tab \tab $\implies e^{ix} = i$ \ and \ $-e^{-ix} = i$\\

\tab \tab $\implies e^{ix} = i$ \ and \ $\frac{1}{e^{ix}} = -i$\\

\tab \tab $\implies e^{ix} = i$ \ and \ $e^{ix} = -\frac{1}{i}$\\

\tab \tab $\implies e^{ix} = i$ \ and \ $e^{ix} = i$\\

\tab \tab $\implies cos(x) + isin(x) = i$\\

\tab \tab $\implies x = 90\degree$\\

\end{enumerate}

\section{Appendix B:  Study Guide}

\subsection{Chapter 2}
\begin{enumerate}

\item{Can you define an angle?  Can you draw an angle?}

\item{Can you define a radian?  A degree?}

\item{Do you know the symbol for radians?  For degrees?}

\item{Can you convert radians to degrees?  Degrees to radians?  Radians to fractions of a circle?  Degrees to fractions of a circle?}

\item{Can you draw a positive angle?  A negative angle?}

\item {Can you define and draw a right angle?}

\item{Can you define and find equivalent angles?  Do you understand why some angles are equivalent?}

\item{Can you define and draw complementary angles?  Supplementary angles?}

\item{Can you list and draw the different classifications of angles?}

\end{enumerate}
\subsection{Chapter 3}
\begin{enumerate}

\item{Can you define a trigonometric function?}

\item{Can you define Cartesean coordinate system?}

\item{Can you define a projection?  Do you understand why a trigonometric projection is at a right angle?}

\item{Can you define the sine of an angle?  The cosine? The tangent?}

\item{Can you define the reciprocal trigonometric functions:  the secant, the cosecant, the cotangent?}

\item{Can you define the trigonometric functions in terms of a right triangle instead of a projections?  Do you understand why you can?}

\end{enumerate}
\subsection{Chapter 4}
\begin{enumerate}


\item{Can you draw a complete unit circle with all the simple angles?}

\item{Can you list all the simple angles in degrees, radians, and fractions of a circle?}

\item{Can you list the sine of all the simple angles?  The cosine?  The tangent?}

\item{Do you understand why the tangent of $\frac{\pi}{2}$ is undefined?}

\end{enumerate}
\subsection{Chapter 5}
\begin{enumerate}


\item{Can you define a periodic function?  The period of a function?}

\item{Can you define phase?  Do you know the phase difference between sine and cosine?}

\item{Can you define asymptotes?}

\item{Can you draw and label a graph of all the trig functions?}\\


\end{enumerate}
\subsection{Chapter 6}
\begin{enumerate}

\item{Can you explain why trig. functions are periodic?  Why every angle has equivalent angles?  Can you find and draw equivalent angles for any angle?}

\item{Can you define even and odd functions?  Can you list which trig. functions are even and which are odd?}

\item{Can you define the Pythagorean Identity?  Explain its relationship to the Pythagorean Theorem?}

\item{Can you rearrange and simplify any expression composed of trig. functions?}

\end{enumerate}
\subsection{Chapter 7}
\begin{enumerate}

\item{Can you define an inverse function? Can you find the inverse of a function? Can you explain why not every function has an inverse function?} 

\item{Can you define a relation? Explain the difference between a function and a relation?}

\item{Can you define principle-value range?  Can you explain why inverse trig. functions have a principal value range?}

\item{Can you draw and label a graph of arc-sine, arc-cos, and arc-tangent?}

\end{enumerate}
\subsection{Chapter 8}
\begin{enumerate}

\item{Can you recite the formula for the sine of the sum of angles?  The cosine?  The tangent?}

\item{Can you derive the double-angle formulae for sine, cosine, and tangent?}

\item{Can you derive the half-angle formulae for sine, cosine, and tangent?}

\end{enumerate}
\subsection{Chapter 9}
\begin{enumerate}

\item{Can you draw and label rectangular coordinates and polar coordinates for a point?}

\item{Can you convert from polar coordinates to rectangular coordinates?}

\item{Can you convert from rectangular coordinates to polar coordinates?}

\end{enumerate}
\subsection{Chapter 10}
\begin{enumerate}

\item{Can you write the law of sines?  The law of cosines?}

\end{enumerate}
\subsection{Chapter 11}
\begin{enumerate}

\item{Can you define a real number?  An imaginary number?  A complex number?}

\item{Can you convert a complex number from rectangular form to polar form?  From polar form to rectangular form?}

\item{Can you add two complex numbers in rectangular form?  In polar form?}

\item{Can you subtract two complex numbers in rectangular form?  In polar form?}

\item{Can you multiply two complex numbers in rectangular form?  In polar form?}

\item{Can you divide two complex numbers in rectangular form?  In polar form?}

\end{enumerate}
\subsection{Chapter 12}
\begin{enumerate}

\item{What is Euler's formula?  Do you understand what it means with respect to rectangular and polar coordinates?}

\item{Can you use Euler's formula to derive different properties of sine and cosinem, such as $sin(-\theta) = -sin(\theta)$, or $cos(\theta) = \frac{1}{2}(e^{i\theta} + e^{-i\theta})$?}

\end{enumerate}

\section{Appendix C:  Additional Material}

\subsection{Derivation of the Law of Sines}

The Law of Sines shows a fundamental relationship between the interior angles of a triangle and the lengths of its sides.  For a triangle with sides of length $A$, $B$, and $C$, with opposing interior angles $a$, $b$, and $c$.\\

\begin{figure}[htb]
\center
\caption{A triangle.}
\label{fig:A triangle}
\begin{tikzpicture}[inner sep=0pt,minimum size=0mm]

\node () at (0,3.5) {};

\draw[] (0,0) -- (2,1);
\draw[] (2,1) -- (1,3);
\draw[] (1,3) -- (0,0);

\node () at (0.4,0.5){$a$};
\node () at (1.6,1.15) {$b$};
\node () at (1.05,2.4) {$c$};

\node () at (1.9,2.1){$A$};
\node () at (0.15,1.5) {$B$};
\node () at (1.25,0.25) {$C$};

\end{tikzpicture}
\end{figure} 

The Law of Sines states that:\\


\tab$\frac{A}{sin(a)} = \frac{B}{sin(b)} = \frac{C}{sin(c)}$\\

This can be derived readily for acute angles using the definition of sine.  Given a triangle, draw a line from one vertex perpendicular to the opposing side.  This line splits the triangle into two right triangles.  We assume this line has a length of $Y$.\\

\begin{figure}[htb]
\center
\caption{A triangle with acute angles.}
\label{fig:A triangle}
\begin{tikzpicture}[inner sep=0pt,minimum size=0mm]

\node () at (0,3.5) {};

\draw[] (0,0) -- (6,0);
\draw[] (6,0) -- (4,3);
\draw[] (4,3) -- (0,0);

\draw[] (4,3) -- (4,0);

\draw[] (4,0.25) -- (3.75,0.25) -- (3.75,0);


\node () at (0.75,0.25){$a$};
\node () at (5.5,0.25) {$b$};
\node () at (3.85,2.5) {$c$};

\node () at (5.5,1.5){$A$};
\node () at (1.4,1.5) {$B$};
\node () at (3,-0.5) {$C$};

\node () at (4.3,1) {$Y$};

\end{tikzpicture}
\end{figure}

Using the definition of sine, we can state:\\

\tab $sin(a) = \frac{Y}{B}$  and  $sin(b) = \frac{Y}{A}$\\

We can rearrange our definitions to be equal:\\

\tab $B sin(a) = Y$  and $A sin(b) = Y$ \\

\tab $\implies B sin(a) = A sin(b)$\\

\tab $\implies \frac{sin(a)}{A} = \frac{sin(b)}{B}$\\

A similar, derivation is possible when comparing an acute angle and an obtuse angle.  We still project a line of length $Y$ down at a right angle to its opposing side:\\

\begin{figure}[htb]
\center
\caption{A triangle with and obtuse angle.}
\label{fig:A triangle}
\begin{tikzpicture}[inner sep=0pt,minimum size=0mm]

\node () at (0,3.5) {};

\draw[] (0,0) -- (4,0);
\draw[] (4,0) -- (6,3);
\draw[] (6,3) -- (0,0);

\draw[] (6,3) -- (6,0);

\draw[dashed] (6,0) -- (4,0);

\draw[] (6,0.25) -- (5.75,0.25) -- (5.75,0);


\node () at (1.25,0.25){$a$};
\node () at (3.75,0.25) {$b$};
\node () at (5.1,2.25) {$c$};
\node () at (4.85,0.25) {$180-b$};

\node () at (5,1){$A$};
\node () at (1.4,1.5) {$B$};
\node () at (3,-0.5) {$C$};

\node () at (6.3,1.5) {$Y$};

\end{tikzpicture}
\end{figure}

Using the definition of sine, we can state:\\

\tab $sin(a) = \frac{Y}{B}$  and  $sin(180-b) = \frac{Y}{A}$\\

As well, from the sum of two angles, we can show that $sin(180-x) = sin(x)$:\\

\tab $sin(180-x) = sin(180)cos(x) - cos(180)sin(x)$\\

\tab $= (0)cos(x) - (-1)sin(x)$\\

\tab $= sin(x)$\\

Thus:\\

\tab $B sin(a) = Y$  and $A sin(b) = Y$ \\

\tab $\implies B sin(a) = A sin(b)$\\

\tab $\implies \frac{sin(a)}{A} = \frac{sin(b)}{B}$\\

When the angle $b$ is a right angle, we know that $Y = A$, thus:\\

\tab $sin(b) = 1 = \frac{A}{A}$ and $sin(a) = \frac{A}{B}$\\

\tab $\implies A sin(b) = A$ and $B sin(a) = A$\\

\tab $\implies A sin(b) = B sin(a)$\\

\tab $\implies \frac{sin(b)}{B} = \frac{sin(a)}{A}$\\

This can also be derived using the assumptions made for the the case of an obtuse angle, the result is the same either way.\\

From these three proofs, we can show equaltiy for any sets of sides and angles, thus:\\

\tab$\frac{A}{sin(a)} = \frac{B}{sin(b)} = \frac{C}{sin(c)}$\\


\clearpage
\subsection{Derivation of the Law of Cosines}

To derive the law of cosines, we use a triangle oriented in the same way as the law of sines.  We define $Y$ as the length of the projected line, and $X$ as the distance from that line to the vertex associated with angle $b$.  By doing this, we constrain the problem so that $A^2 = X^2 + Y^2$, by the Pythagorean Theorem.  We can then rewrite $X$ and $Y$ in terms of $a$:\\

\begin{figure}[htb]
\center
\caption{A triangle with acute angles.}
\label{fig:A triangle}
\begin{tikzpicture}[inner sep=0pt,minimum size=0mm]

\node () at (0,3.5) {};

\draw[] (0,0) -- (6,0);
\draw[] (6,0) -- (4,3);
\draw[] (4,3) -- (0,0);

\draw[] (4,3) -- (4,0);

\draw[] (4,0.25) -- (3.75,0.25) -- (3.75,0);


\node () at (0.75,0.25){$a$};
\node () at (5.5,0.25) {$b$};
\node () at (3.85,2.5) {$c$};

\node () at (5.5,1.5){$A$};
\node () at (1.4,1.5) {$B$};
\node () at (3,-0.5) {$C$};

\node () at (4.3,1) {$Y$};

\node () at (5,-0.5) {$X$};

\end{tikzpicture}
\end{figure}

\begin{figure}[htb]
\center
\caption{A triangle with an obtuse angle.}
\label{fig:A triangle}
\begin{tikzpicture}[inner sep=0pt,minimum size=0mm]

\node () at (0,3.5) {};

\draw[] (0,0) -- (4,0);
\draw[] (4,0) -- (6,3);
\draw[] (6,3) -- (0,0);

\draw[] (6,3) -- (6,0);

\draw[dashed] (6,0) -- (4,0);

\draw[] (6,0.25) -- (5.75,0.25) -- (5.75,0);


\node () at (1.25,0.25){$a$};
\node () at (3.75,0.25) {$b$};
\node () at (5.1,2.25) {$c$};
\node () at (4.85,0.25) {$180-b$};

\node () at (5,1){$A$};
\node () at (1.4,1.5) {$B$};
\node () at (3,-0.5) {$C$};

\node () at (6.3,1.5) {$Y$};
\node () at (5,-.5) {$X$};


\end{tikzpicture}
\end{figure}


\tab $X = B cos(a) - C$ or $C - B cos(a)$\\

\tab $\implies X^2 = B^2 cos^2(a) + C^2 - 2BC cos(a)$\\

\tab $Y = B sin(a)$\\

\tab $\implies Y^2 = B^2 sin^2(a)$\\

\tab $A^2 = X^2 + Y^2$\\

\tab $\implies A^2 = B^2 cos^2(a) + C^2 - 2BC cos(a) + B^2 sin^2(a)$\\

\tab $\implies A^2 = B^2(sin^2(a) + cos^2(a)) + C^2 - 2BC cos(a)$\\

\tab $\implies A^2 = B^2(1) + C^2 - 2BC cos(a)$\\

\tab $\implies A^2 = B^2 + C^2 - 2BC cos(a)$\\

Again, note that because of our definition of $X$, we can show the law of cosines holds for any angle, regardless of whether it is acute, right, or obtuse.  We do not need a separate proof for each of these three cases.\\



\end{document}

